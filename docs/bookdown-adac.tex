\documentclass[]{book}
\usepackage{lmodern}
\usepackage{amssymb,amsmath}
\usepackage{ifxetex,ifluatex}
\usepackage{fixltx2e} % provides \textsubscript
\ifnum 0\ifxetex 1\fi\ifluatex 1\fi=0 % if pdftex
  \usepackage[T1]{fontenc}
  \usepackage[utf8]{inputenc}
\else % if luatex or xelatex
  \ifxetex
    \usepackage{mathspec}
  \else
    \usepackage{fontspec}
  \fi
  \defaultfontfeatures{Ligatures=TeX,Scale=MatchLowercase}
\fi
% use upquote if available, for straight quotes in verbatim environments
\IfFileExists{upquote.sty}{\usepackage{upquote}}{}
% use microtype if available
\IfFileExists{microtype.sty}{%
\usepackage{microtype}
\UseMicrotypeSet[protrusion]{basicmath} % disable protrusion for tt fonts
}{}
\usepackage[margin=1in]{geometry}
\usepackage{hyperref}
\hypersetup{unicode=true,
            pdftitle={Análise de Dados Amostrais Complexos},
            pdfauthor={Djalma Pessoa e Pedro Nascimento Silva},
            pdfborder={0 0 0},
            breaklinks=true}
\urlstyle{same}  % don't use monospace font for urls
\usepackage{natbib}
\bibliographystyle{apalike}
\usepackage{color}
\usepackage{fancyvrb}
\newcommand{\VerbBar}{|}
\newcommand{\VERB}{\Verb[commandchars=\\\{\}]}
\DefineVerbatimEnvironment{Highlighting}{Verbatim}{commandchars=\\\{\}}
% Add ',fontsize=\small' for more characters per line
\usepackage{framed}
\definecolor{shadecolor}{RGB}{248,248,248}
\newenvironment{Shaded}{\begin{snugshade}}{\end{snugshade}}
\newcommand{\KeywordTok}[1]{\textcolor[rgb]{0.13,0.29,0.53}{\textbf{#1}}}
\newcommand{\DataTypeTok}[1]{\textcolor[rgb]{0.13,0.29,0.53}{#1}}
\newcommand{\DecValTok}[1]{\textcolor[rgb]{0.00,0.00,0.81}{#1}}
\newcommand{\BaseNTok}[1]{\textcolor[rgb]{0.00,0.00,0.81}{#1}}
\newcommand{\FloatTok}[1]{\textcolor[rgb]{0.00,0.00,0.81}{#1}}
\newcommand{\ConstantTok}[1]{\textcolor[rgb]{0.00,0.00,0.00}{#1}}
\newcommand{\CharTok}[1]{\textcolor[rgb]{0.31,0.60,0.02}{#1}}
\newcommand{\SpecialCharTok}[1]{\textcolor[rgb]{0.00,0.00,0.00}{#1}}
\newcommand{\StringTok}[1]{\textcolor[rgb]{0.31,0.60,0.02}{#1}}
\newcommand{\VerbatimStringTok}[1]{\textcolor[rgb]{0.31,0.60,0.02}{#1}}
\newcommand{\SpecialStringTok}[1]{\textcolor[rgb]{0.31,0.60,0.02}{#1}}
\newcommand{\ImportTok}[1]{#1}
\newcommand{\CommentTok}[1]{\textcolor[rgb]{0.56,0.35,0.01}{\textit{#1}}}
\newcommand{\DocumentationTok}[1]{\textcolor[rgb]{0.56,0.35,0.01}{\textbf{\textit{#1}}}}
\newcommand{\AnnotationTok}[1]{\textcolor[rgb]{0.56,0.35,0.01}{\textbf{\textit{#1}}}}
\newcommand{\CommentVarTok}[1]{\textcolor[rgb]{0.56,0.35,0.01}{\textbf{\textit{#1}}}}
\newcommand{\OtherTok}[1]{\textcolor[rgb]{0.56,0.35,0.01}{#1}}
\newcommand{\FunctionTok}[1]{\textcolor[rgb]{0.00,0.00,0.00}{#1}}
\newcommand{\VariableTok}[1]{\textcolor[rgb]{0.00,0.00,0.00}{#1}}
\newcommand{\ControlFlowTok}[1]{\textcolor[rgb]{0.13,0.29,0.53}{\textbf{#1}}}
\newcommand{\OperatorTok}[1]{\textcolor[rgb]{0.81,0.36,0.00}{\textbf{#1}}}
\newcommand{\BuiltInTok}[1]{#1}
\newcommand{\ExtensionTok}[1]{#1}
\newcommand{\PreprocessorTok}[1]{\textcolor[rgb]{0.56,0.35,0.01}{\textit{#1}}}
\newcommand{\AttributeTok}[1]{\textcolor[rgb]{0.77,0.63,0.00}{#1}}
\newcommand{\RegionMarkerTok}[1]{#1}
\newcommand{\InformationTok}[1]{\textcolor[rgb]{0.56,0.35,0.01}{\textbf{\textit{#1}}}}
\newcommand{\WarningTok}[1]{\textcolor[rgb]{0.56,0.35,0.01}{\textbf{\textit{#1}}}}
\newcommand{\AlertTok}[1]{\textcolor[rgb]{0.94,0.16,0.16}{#1}}
\newcommand{\ErrorTok}[1]{\textcolor[rgb]{0.64,0.00,0.00}{\textbf{#1}}}
\newcommand{\NormalTok}[1]{#1}
\usepackage{longtable,booktabs}
\usepackage{graphicx,grffile}
\makeatletter
\def\maxwidth{\ifdim\Gin@nat@width>\linewidth\linewidth\else\Gin@nat@width\fi}
\def\maxheight{\ifdim\Gin@nat@height>\textheight\textheight\else\Gin@nat@height\fi}
\makeatother
% Scale images if necessary, so that they will not overflow the page
% margins by default, and it is still possible to overwrite the defaults
% using explicit options in \includegraphics[width, height, ...]{}
\setkeys{Gin}{width=\maxwidth,height=\maxheight,keepaspectratio}
\IfFileExists{parskip.sty}{%
\usepackage{parskip}
}{% else
\setlength{\parindent}{0pt}
\setlength{\parskip}{6pt plus 2pt minus 1pt}
}
\setlength{\emergencystretch}{3em}  % prevent overfull lines
\providecommand{\tightlist}{%
  \setlength{\itemsep}{0pt}\setlength{\parskip}{0pt}}
\setcounter{secnumdepth}{5}
% Redefines (sub)paragraphs to behave more like sections
\ifx\paragraph\undefined\else
\let\oldparagraph\paragraph
\renewcommand{\paragraph}[1]{\oldparagraph{#1}\mbox{}}
\fi
\ifx\subparagraph\undefined\else
\let\oldsubparagraph\subparagraph
\renewcommand{\subparagraph}[1]{\oldsubparagraph{#1}\mbox{}}
\fi

%%% Use protect on footnotes to avoid problems with footnotes in titles
\let\rmarkdownfootnote\footnote%
\def\footnote{\protect\rmarkdownfootnote}

%%% Change title format to be more compact
\usepackage{titling}

% Create subtitle command for use in maketitle
\newcommand{\subtitle}[1]{
  \posttitle{
    \begin{center}\large#1\end{center}
    }
}

\setlength{\droptitle}{-2em}
  \title{Análise de Dados Amostrais Complexos}
  \pretitle{\vspace{\droptitle}\centering\huge}
  \posttitle{\par}
  \author{Djalma Pessoa e Pedro Nascimento Silva}
  \preauthor{\centering\large\emph}
  \postauthor{\par}
  \predate{\centering\large\emph}
  \postdate{\par}
  \date{2017-09-06}

\usepackage{booktabs}

\usepackage{amsthm}
\newtheorem{theorem}{Teorema}[chapter]
\newtheorem{lemma}{Lema}[chapter]
\theoremstyle{definition}
\newtheorem{definition}{Definição}[chapter]
\newtheorem{corollary}{Corolário}[chapter]
\newtheorem{proposition}{Proposição}[chapter]
\theoremstyle{definition}
\newtheorem{example}{Example}[chapter]
\theoremstyle{definition}
\newtheorem{exercise}{Exercise}[chapter]
\theoremstyle{remark}
\newtheorem*{remark}{Observação }
\newtheorem*{solution}{Solution}
\let\BeginKnitrBlock\begin \let\EndKnitrBlock\end
\begin{document}
\maketitle

{
\setcounter{tocdepth}{1}
\tableofcontents
}
\chapter*{Prefácio}\label{prefacio}
\addcontentsline{toc}{chapter}{Prefácio}

\section*{Agradecimentos}\label{agradecimentos}
\addcontentsline{toc}{section}{Agradecimentos}

\chapter{Introdução}\label{introduc}

\section{Motivação}\label{motivacao}

\section{Objetivos do Livro}\label{objetivos-do-livro}

\section{Laboratório de R do Capítulo 1.}\label{epa}

\section{total SE DEff}\label{total-se-deff}

\section{analf1 1174220 127982
2.0543}\label{analf1-1174220-127982-2.0543}

\section{total SE DEff}\label{total-se-deff-1}

\section{analf2 4792344 318877
3.3237}\label{analf2-4792344-318877-3.3237}

\section{Ratio estimator:
svyratio.survey.design2(\textasciitilde{}analf1,
\textasciitilde{}faixa1,
ppv.se.des)}\label{ratio-estimator-svyratio.survey.design2analf1-faixa1-ppv.se.des}

\section{Ratios=}\label{ratios}

\section{faixa1}\label{faixa1}

\section{analf1 0.118689}\label{analf1-0.118689}

\section{SEs=}\label{ses}

\section{faixa1}\label{faixa1-1}

\section{analf1 0.01178896}\label{analf1-0.01178896}

\section{Ratio estimator:
svyratio.survey.design2(\textasciitilde{}analf2,
\textasciitilde{}faixa2,
ppv.se.des)}\label{ratio-estimator-svyratio.survey.design2analf2-faixa2-ppv.se.des}

\section{Ratios=}\label{ratios-1}

\section{faixa2}\label{faixa2}

\section{analf2 0.1086871}\label{analf2-0.1086871}

\section{SEs=}\label{ses-1}

\section{faixa2}\label{faixa2-1}

\section{analf2 0.006732254}\label{analf2-0.006732254}

\section{regiao analf1 se
DEff.analf1}\label{regiao-analf1-se-deff.analf1}

\section{1 1 3512866 352619.5 9.660561}\label{section}

\section{2 2 1174220 127982.2 2.054345}\label{section-1}

\subsection{Estimativa do efeito de plano amostral
(EPA)}\label{estimativa-do-efeito-de-plano-amostral-epa}

\section{analf1}\label{analf1}

\section{analf1 2.054049}\label{analf1-2.054049}

\section{analf2}\label{analf2}

\section{analf2 3.32324}\label{analf2-3.32324}

\section{total SE DEff}\label{total-se-deff-2}

\section{{[}1,{]} 1174220 127982 2.6426}\label{section-2}

\section{total SE DEff}\label{total-se-deff-3}

\section{{[}1,{]} 4792344 318877 4.1667}\label{section-3}

\section{total SE DEff}\label{total-se-deff-4}

\section{I(ifelse(regiao == 2, analf1, 0)) 1174220 127982
2.6426}\label{iifelseregiao-2-analf1-0-1174220-127982-2.6426}

\section{total SE DEff}\label{total-se-deff-5}

\section{I(ifelse(regiao == 2, analf2, 0)) 4792344 318877
4.1667}\label{iifelseregiao-2-analf2-0-4792344-318877-4.1667}

\section{Taxa Erro\_Padrao CV}\label{taxa-erro_padrao-cv}

\section{1 11.80303 0.1174791 0.9953299}\label{section-4}

\section{{[}1{]} 60202 1019}\label{section-5}

\section{diag\_dep}\label{diag_dep}

\section{7.6}\label{section-6}

\section{diag\_dep}\label{diag_dep-1}

\section{diag\_dep 0.2}\label{diag_dep-0.2}

\section{{[}1{]} 7.2 8.1}\label{section-7}

\section{diag\_dep}\label{diag_dep-2}

\section{7.6 7.2 8.1}\label{section-8}

\chapter{Referencial para Inferência}\label{refinf}

\section{Modelagem - Primeiras Idéias}\label{classic}

\subsection{Abordagem 1 - Modelagem
Clássica}\label{abordagem-1---modelagem-classica}

\subsection{Abordagem 2 - Amostragem
Probabilística}\label{abordagem-2---amostragem-probabilistica}

\subsection{Discussão das Abordagens 1 e
2}\label{discussao-das-abordagens-1-e-2}

\subsection{Abordagem 3 - Modelagem de
Superpopulação}\label{modelsuperpop}

\section{Fontes de Variação}\label{fontes-de-variacao}

\section{Modelos de Superpopulação}\label{modelos-de-superpopulacao}

\section{Planejamento Amostral}\label{planamo}

\section{Planos Amostrais Informativos e Ignoráveis}\label{inform}

\chapter{Estimação Baseada no Plano Amostral}\label{capplanamo}

\section{Estimação de Totais}\label{estimatotais}

\section{Por que Estimar Variâncias}\label{por-que-estimar-variancias}

\section{Linearização de Taylor para Estimar variâncias}\label{taylor}

\section{Método do Conglomerado
Primário}\label{metodo-do-conglomerado-primario}

\section{Métodos de Replicação}\label{metodos-de-replicacao}

\section{Laboratório de R}\label{laboratorio-de-r}

\section{faixa}\label{faixa}

\section{0.01178896}\label{section-9}

\section{Ratio estimator:
svyratio.survey.design2(\textasciitilde{}analf.faixa,
\textasciitilde{}faixa,
ppv.se.des)}\label{ratio-estimator-svyratio.survey.design2analf.faixa-faixa-ppv.se.des}

\section{Ratios=}\label{ratios-2}

\section{faixa}\label{faixa-1}

\section{analf.faixa 0.118689}\label{analf.faixa-0.118689}

\section{SEs=}\label{ses-2}

\section{faixa}\label{faixa-2}

\section{analf.faixa 0.01178896}\label{analf.faixa-0.01178896}

\section{Ratio estimator:
svyratio.svyrep.design(\textasciitilde{}analf.faixa,
\textasciitilde{}faixa,
ppv.se.des.jkn)}\label{ratio-estimator-svyratio.svyrep.designanalf.faixa-faixa-ppv.se.des.jkn}

\section{Ratios=}\label{ratios-3}

\section{faixa}\label{faixa-3}

\section{analf.faixa 0.118689}\label{analf.faixa-0.118689-1}

\section{SEs=}\label{ses-3}

\section{{[},1{]}}\label{section-10}

\section{{[}1,{]} 0.01181434}\label{section-11}

\section{Ratio estimator:
svyratio.svyrep.design(\textasciitilde{}analf.faixa,
\textasciitilde{}faixa,
ppv.se.des.boot)}\label{ratio-estimator-svyratio.svyrep.designanalf.faixa-faixa-ppv.se.des.boot}

\section{Ratios=}\label{ratios-4}

\section{faixa}\label{faixa-4}

\section{analf.faixa 0.118689}\label{analf.faixa-0.118689-2}

\section{SEs=}\label{ses-4}

\section{{[},1{]}}\label{section-12}

\section{{[}1,{]} 0.01256654}\label{section-13}

\section{\texorpdfstring{{[}1{]}
``svyrep.design''}{{[}1{]} svyrep.design}}\label{svyrep.design}

\section{\texorpdfstring{{[}1{]} ``repweights'' ``pweights''
``type''}{{[}1{]} repweights pweights type}}\label{repweights-pweights-type}

\section{\texorpdfstring{{[}4{]} ``rho'' ``scale''
``rscales''}{{[}4{]} rho scale rscales}}\label{rho-scale-rscales}

\section{\texorpdfstring{{[}7{]} ``call'' ``combined.weights''
``selfrep''}{{[}7{]} call combined.weights selfrep}}\label{call-combined.weights-selfrep}

\section{\texorpdfstring{{[}10{]} ``mse'' ``variables''
``degf''}{{[}10{]} mse variables degf}}\label{mse-variables-degf}

\section{{[}1{]} 8903}\label{section-14}

\section{{[}1{]} 276}\label{section-15}

\section{num {[}1:8903, 1:276{]} 0 0 1.06 1.06 1.06
\ldots{}}\label{num-18903-1276-0-0-1.06-1.06-1.06}

\section{{[}1{]} 0.01181}\label{section-16}

\section{theta SE}\label{theta-se}

\section{{[}1,{]} 0.11869 0.0118}\label{section-17}

\section{theta SE}\label{theta-se-1}

\section{{[}1,{]} 0.50623 0.0494}\label{section-18}

\chapter{Efeitos do Plano Amostral}\label{epa}

\section{Introdução}\label{introducao}

\section{Efeito do Plano Amostral (EPA) de
Kish}\label{efeito-do-plano-amostral-epa-de-kish}

\section{Efeito do Plano Amostral
Ampliado}\label{efeito-do-plano-amostral-ampliado}

\section{Intervalos de Confiança e Testes de
Hipóteses}\label{intervalos-de-confianca-e-testes-de-hipoteses}

\section{Efeitos Multivariados de Plano
Amostral}\label{efeitos-multivariados-de-plano-amostral}

\section{Laboratório de R}\label{laboratorio-de-r-1}

\section{\texorpdfstring{Warning: package `survey' was built under R
version
3.4.1}{Warning: package survey was built under R version 3.4.1}}\label{warning-package-survey-was-built-under-r-version-3.4.1}

\section{Carregando pacotes exigidos:
grid}\label{carregando-pacotes-exigidos-grid}

\section{Carregando pacotes exigidos:
methods}\label{carregando-pacotes-exigidos-methods}

\section{Carregando pacotes exigidos:
Matrix}\label{carregando-pacotes-exigidos-matrix}

\section{Carregando pacotes exigidos:
survival}\label{carregando-pacotes-exigidos-survival}

\section{}\label{section-19}

\section{\texorpdfstring{Attaching package:
`survey'}{Attaching package: survey}}\label{attaching-package-survey}

\section{\texorpdfstring{The following object is masked from
`package:graphics':}{The following object is masked from package:graphics:}}\label{the-following-object-is-masked-from-packagegraphics}

\section{}\label{section-20}

\section{dotchart}\label{dotchart}

\section{{[}1{]} 0.140409}\label{section-21}

\section{{[}1{]} 0.3963556}\label{section-22}

\chapter{Ajuste de Modelos Paramétricos}\label{ajmodpar}

\section{Introdução}\label{modpar1}

\section{Método de Máxima Verossimilhança
(MV)}\label{metodo-de-maxima-verossimilhanca-mv}

\section{Ponderação de Dados
Amostrais}\label{ponderacao-de-dados-amostrais}

\section{Método de Máxima Pseudo-Verossimilhança}\label{modpar3}

\section{Robustez do Procedimento
MPV}\label{robustez-do-procedimento-mpv}

\section{Desvantagens da Inferência de
Aleatorização}\label{desvantagens-da-inferencia-de-aleatorizacao}

\section{Laboratório de R}\label{laboratorio-de-r-2}

\chapter{Modelos de Regressão}\label{modreg}

\section{Modelo de Regressão Linear Normal}\label{modlinear}

\subsection{Especificação do Modelo}\label{especificacao-do-modelo}

\subsection{Pseudo-parâmetros do
Modelo}\label{pseudo-parametros-do-modelo}

\subsection{Estimadores de MPV dos Parâmetros do
Modelo}\label{estimadores-de-mpv-dos-parametros-do-modelo}

\subsection{Estimação da Variância de Estimadores de
MPV}\label{estimacao-da-variancia-de-estimadores-de-mpv}

\section{Modelo de Regressão Logística}\label{modlogist}

\section{Warning in 1.96 * sqrt(var\_raz112) * c(-1, 1): Recycling array
of length 1 in array-vector arithmetic is
deprecated.}\label{warning-in-1.96-sqrtvar_raz112-c-1-1-recycling-array-of-length-1-in-array-vector-arithmetic-is-deprecated.}

\section{Use c() or as.vector()
instead.}\label{use-c-or-as.vector-instead.}

\section{Warning in 1.96 * sqrt(var\_raz123) * c(-1, 1): Recycling array
of length 1 in array-vector arithmetic is
deprecated.}\label{warning-in-1.96-sqrtvar_raz123-c-1-1-recycling-array-of-length-1-in-array-vector-arithmetic-is-deprecated.}

\section{Use c() or as.vector()
instead.}\label{use-c-or-as.vector-instead.-1}

\section{Warning in 1.96 * sqrt(var\_raz212) * c(-1, 1): Recycling array
of length 1 in array-vector arithmetic is
deprecated.}\label{warning-in-1.96-sqrtvar_raz212-c-1-1-recycling-array-of-length-1-in-array-vector-arithmetic-is-deprecated.}

\section{Use c() or as.vector()
instead.}\label{use-c-or-as.vector-instead.-2}

\section{Warning in 1.96 * sqrt(var\_raz223) * c(-1, 1): Recycling array
of length 1 in array-vector arithmetic is
deprecated.}\label{warning-in-1.96-sqrtvar_raz223-c-1-1-recycling-array-of-length-1-in-array-vector-arithmetic-is-deprecated.}

\section{Use c() or as.vector()
instead.}\label{use-c-or-as.vector-instead.-3}

\section{Warning in 1.96 * sqrt(var\_raz312) * c(-1, 1): Recycling array
of length 1 in array-vector arithmetic is
deprecated.}\label{warning-in-1.96-sqrtvar_raz312-c-1-1-recycling-array-of-length-1-in-array-vector-arithmetic-is-deprecated.}

\section{Use c() or as.vector()
instead.}\label{use-c-or-as.vector-instead.-4}

\section{Warning in 1.96 * sqrt(var\_raz323) * c(-1, 1): Recycling array
of length 1 in array-vector arithmetic is
deprecated.}\label{warning-in-1.96-sqrtvar_raz323-c-1-1-recycling-array-of-length-1-in-array-vector-arithmetic-is-deprecated.}

\section{Use c() or as.vector()
instead.}\label{use-c-or-as.vector-instead.-5}

\section{Teste de Hipóteses}\label{teste-de-hipoteses}

\section{Laboratório de R}\label{laboratorio-de-r-3}

\section{\texorpdfstring{{[}1{]} ``stra'' ``psu'' ``pesopes''
``informal'' ``sx''
``id''}{{[}1{]} stra psu pesopes informal sx id}}\label{stra-psu-pesopes-informal-sx-id}

\section{\texorpdfstring{{[}7{]} ``ae'' ``ht'' ``re''
``um''}{{[}7{]} ae ht re um}}\label{ae-ht-re-um}

\section{stra psu pesopes informal sx id
ae}\label{stra-psu-pesopes-informal-sx-id-ae}

\section{\texorpdfstring{``numeric'' ``numeric'' ``numeric'' ``numeric''
``numeric'' ``numeric''
``numeric''}{numeric numeric numeric numeric numeric numeric numeric}}\label{numeric-numeric-numeric-numeric-numeric-numeric-numeric}

\section{ht re um}\label{ht-re-um}

\section{\texorpdfstring{``numeric'' ``numeric''
``numeric''}{numeric numeric numeric}}\label{numeric-numeric-numeric}

\section{Wald test for ht:re}\label{wald-test-for-htre}

\section{in svyglm(formula = informal \textasciitilde{} sx + ae + ht +
id + re + sx * id
+}\label{in-svyglmformula-informal-sx-ae-ht-id-re-sx-id}

\section{sx * ht + ae * ht + ht * id + ht * re, design =
pnad.des,}\label{sx-ht-ae-ht-ht-id-ht-re-design-pnad.des}

\section{family = binomial)}\label{family-binomial}

\section{F = 6.742662 on 4 and 616 df: p=
2.58e-05}\label{f-6.742662-on-4-and-616-df-p-2.58e-05}

\chapter{Testes de Qualidade de Ajuste}\label{testqualajust}

\section{Introdução}\label{introducao-1}

Tabelas de distribuições de frequências ocorrem comumente na análise de
dados de pesquisas complexas. Tais tabelas são formadas pela
classificação e cálculo de frequências dos dados da amostra disponível
segundo níveis de uma variável categórica - tabelas de uma entrada - ou
segundo celas de uma classificação cruzada de duas (ou mais) variáveis
categóricas - tabelas de duas (ou mais) entradas. Neste capítulo
concentraremos a atenção em tabelas de uma entrada, ou equivalentemente
nas frequências absolutas e relativas (ou proporções) correspondentes.

Em muitos casos, o objetivo da análise é testar hipóteses de bondade de
ajuste de modelos para descrever essas distribuições de frequências. Sob
a hipótese de observações IID (distribuição Multinomial) ou
equivalentemente, de amostragem aleatória simples, inferências válidas
para testar tais hipóteses podem ser baseadas na estatística padrão de
teste qui-quadrado de Pearson. Tais testes podem ser facilmente
executados usando procedimentos prontos em pacotes estatísticos padrões
tais como o SAS, S-Plus, SPSS, GLIM e outros.

No caso de planos amostrais complexos, entretanto, os procedimentos de
teste precisam ser ajustados devido aos efeitos de conglomeração,
estratificação e/ou pesos desiguais. Neste capítulo examinaremos o
impacto do plano amostral sobre as estatísticas de teste usuais notando
que, em alguns casos, os valores observados dessas estatísticas de teste
podem ser muito grandes, acarretando inferências incorretas, conforme já
ilustrado no Exemplo @ref\{ex:exebin\}. Isto ocorre porque a
probabilidade de erros do tipo I (rejeitar a hipótese nula quando esta é
verdadeira) é muito maior que o nível nominal de significância
\(\alpha\) especificado.

Para obter inferências válidas usando amostras complexas podemos
introduzir correções na estatística de teste de Pearson, tais como os
ajustes de Rao-Scott, ou alternativamente usar outras estatísticas de
teste que já incorporem o plano amostral, tais como a estatística de
Wald. Os dois enfoques serão ilustrados através de um exemplo
introdutório simples de teste de bondade de ajuste. Os resultados
discutidos neste capítulo são adequados tanto para uma abordagem de
aleatorização, em que os parâmetros se referem à população finita em
questão, quanto para uma abordagem baseada em modelos, em que os
parâmetros especificam algum modelo de superpopulação.

\section{Teste para uma Proporção}\label{teste-para-uma-proporcao}

\subsection{Correção de Estatísticas
Clássicas}\label{correcao-de-estatisticas-classicas}

No Exemplo @ref\{ex:exebin\} a estatística de teste \(Z_{bin},\) que foi
utilizada para comparar com um valor hipotético pré-fixado a proporção
de empregados cobertos por plano de saúde, resultou num teste mais
liberal do que o teste que empregou a estatística \(Z_{p}\), baseada no
plano amostral efetivamente adotado. A causa disto foi o fato de
\(Z_{bin}\) não considerar o efeito de conglomeração existente. Vamos
examinar com mais detalhes o comportamento assintótico da estatística de
teste \(Z_{bin}\), construindo a estatística de teste \(X_{P}^{2}\) de
Pearson para o exemplo correspondente. Para isto, consideremos a Tabela
\ref{tab:prophip} contendo a distribuição de frequências, onde \(n_{j}\)
e \(p_{oj}\) são as frequências (absolutas) observadas na amostra e as
proporções hipotéticas nas categorias de interesse, respectivamente.

\begin{longtable}[]{@{}llll@{}}
\caption{\label{tab:prophip} Frequências observadas e proporções
hipotéticas}\tabularnewline
\toprule
\begin{minipage}[b]{0.23\columnwidth}\raggedright\strut
Categoria\strut
\end{minipage} & \begin{minipage}[b]{0.05\columnwidth}\raggedright\strut
\(j\)\strut
\end{minipage} & \begin{minipage}[b]{0.07\columnwidth}\raggedright\strut
\(n_j\)\strut
\end{minipage} & \begin{minipage}[b]{0.11\columnwidth}\raggedright\strut
\(p_{0j}\)\strut
\end{minipage}\tabularnewline
\midrule
\endfirsthead
\toprule
\begin{minipage}[b]{0.23\columnwidth}\raggedright\strut
Categoria\strut
\end{minipage} & \begin{minipage}[b]{0.05\columnwidth}\raggedright\strut
\(j\)\strut
\end{minipage} & \begin{minipage}[b]{0.07\columnwidth}\raggedright\strut
\(n_j\)\strut
\end{minipage} & \begin{minipage}[b]{0.11\columnwidth}\raggedright\strut
\(p_{0j}\)\strut
\end{minipage}\tabularnewline
\midrule
\endhead
\begin{minipage}[t]{0.23\columnwidth}\raggedright\strut
Cobertos por planos de saúde\strut
\end{minipage} & \begin{minipage}[t]{0.05\columnwidth}\raggedright\strut
1\strut
\end{minipage} & \begin{minipage}[t]{0.07\columnwidth}\raggedright\strut
840\strut
\end{minipage} & \begin{minipage}[t]{0.11\columnwidth}\raggedright\strut
0.8\strut
\end{minipage}\tabularnewline
\begin{minipage}[t]{0.23\columnwidth}\raggedright\strut
Não cobertos\strut
\end{minipage} & \begin{minipage}[t]{0.05\columnwidth}\raggedright\strut
2\strut
\end{minipage} & \begin{minipage}[t]{0.07\columnwidth}\raggedright\strut
160\strut
\end{minipage} & \begin{minipage}[t]{0.11\columnwidth}\raggedright\strut
0.2\strut
\end{minipage}\tabularnewline
\begin{minipage}[t]{0.23\columnwidth}\raggedright\strut
Todos empregados\strut
\end{minipage} & \begin{minipage}[t]{0.05\columnwidth}\raggedright\strut
-\strut
\end{minipage} & \begin{minipage}[t]{0.07\columnwidth}\raggedright\strut
1000\strut
\end{minipage} & \begin{minipage}[t]{0.11\columnwidth}\raggedright\strut
1.0\strut
\end{minipage}\tabularnewline
\bottomrule
\end{longtable}

As proporções populacionais desconhecidas nas categorias são
\(p_{j}=N_{j}/N\), onde \(N\) é o tamanho total da população de
empregados e \(N_{j}\) é o número de elementos da população na categoria
\(j,\ j=1,2\). Os parâmetros populacionais \(p_{j}\) poderiam também ser
considerados como pseudo-parâmetros, se vistos como estimativas de censo
para as probabilidades desconhecidas (\(\pi _{j}\), digamos) no contexto
de um modelo de superpopulação.

A estatística de teste de Pearson para a hipótese simples de bondade de
ajuste \(H_{0}:p_{j}=p_{0j},\ j=1,2\), é dada por

\begin{equation}
X_{P}^{2}=\sum\limits_{j=1}^{2}\left( n_{j}-n\,p_{0j}\right)
^{2}/\left( n\,p_{0j}\right) =n\sum\limits_{j=1}^{2}\left( 
\widehat{p}_{j}-p_{0j}\right) ^{2}/p_{0j},  
\label{eq:qual1}
\end{equation}

onde as proporções \(\widehat{p}_{j}=n_{j}/n\) são estimativas amostrais
usuais das proporções populacionais \(p_{j}\), para \(j=1,2\).

\begin{Shaded}
\begin{Highlighting}[]
\NormalTok{p0 <-}\StringTok{ }\KeywordTok{c}\NormalTok{(.}\DecValTok{8}\NormalTok{, .}\DecValTok{2}\NormalTok{)}
\NormalTok{obs<-}\StringTok{ }\KeywordTok{c}\NormalTok{(}\DecValTok{840}\NormalTok{, }\DecValTok{160}\NormalTok{)}
\NormalTok{n<-}\StringTok{ }\KeywordTok{sum}\NormalTok{(obs)}
\NormalTok{phat <-}\StringTok{ }\NormalTok{obs}\OperatorTok{/}\NormalTok{n}
\CommentTok{# Estatística de Pearson}
\NormalTok{x2p <-}\StringTok{ }\KeywordTok{sum}\NormalTok{((obs}\OperatorTok{-}\NormalTok{n}\OperatorTok{*}\NormalTok{p0)}\OperatorTok{^}\DecValTok{2}\OperatorTok{/}\NormalTok{(n}\OperatorTok{*}\NormalTok{p0))}
\NormalTok{x2p}
\end{Highlighting}
\end{Shaded}

\begin{verbatim}
## [1] 10
\end{verbatim}

Como há apenas duas categorias e as proporções devem somar \(1\),
observa-se que \(p_{2}=1-p_{1},\) \(\widehat{p}_{2}=1-\widehat{p}_{1}\)
e \(p_{02}=1-p_{01}\). Isto acarreta na equivalência entre as
estatísticas \(Z_{bin}\) e \(X_{P}^{2}\) demonstrada pela relação

\begin{equation}
X_{P}^{2}=n\sum\limits_{j=1}^{2}\left( \widehat{p}%
_{j}-p_{0j}\right) ^{2}/p_{0j}=\frac{\left( \widehat{p}-p_{0}\right) ^{2}}{%
p_{0}\left( 1-p_{0}\right) /n}=Z_{bin}^{2}  
\label{eq:qual2}
\end{equation}

onde \(\widehat{p}=\widehat{p}_{1}\) e \(p_{0}=p_{01}\) para
simplicidade e coerência com a notação do Exemplo @ref\{ex:exebin\}.

Sob a hipótese de observações IID, a distribuição assintótica da
estatística \(X_{P}^{2}\) é qui-quadrado (\(\chi^{2}\)). Neste caso, em
que há apenas duas categorias e uma restrição (soma das proporções igual
a \(1\)), a distribuição da estatística \(X_{P}^{2}\) em \eqref{eq:qual2}
tem apenas um grau de liberdade.

Rao e Scott(1981) obtiveram resultados gerais para a distribuição
assintótica da estatística de teste \(X_{P}^{2}\) de Pearson sob planos
amostrais complexos. Com apenas duas celas, a distribuição assintótica
da estatística de teste \(X_{P}^{2}\) é a distribuição da variável
aleatória \(dW\), onde \(W\) tem distribuição
\(\chi ^{2}\left( 1\right)\) (qui-quadrado com um grau de liberdade) e
\(d\) é o efeito de plano amostral (EPA) da estimativa \(\widehat{p}\)
da proporção \(p\). O efeito de plano amostral nesse caso é dado por
\(d=V_{p}\left( \widehat{p}\right) /V_{bin}\left( \widehat{p}\right)\).

Para uma amostra de empregados selecionada por amostragem aleatória
simples, teríamos \(d=1\) pois \(V_{p}\left( \widehat{p}\right)\) e
\(V_{bin}\left( \widehat{p}\right)\) seriam iguais. Neste caso, a
estatística \(X_{P}^{2}\) de teste seria assintoticamente
\(\chi ^{2}\left(1\right)\). Como a amostra foi efetivamente selecionada
por amostragem de conglomerados, devido à correlação intraclasse
positiva o efeito de plano amostral \(d\) é maior que um, e portanto a
distribuição assintótica da estatística de teste \(X_{P}^{2}\) não é
mais \(\chi ^{2}\left( 1\right)\).

Considerando que o impacto da correlação intraclasse positiva na
distribuição assintótica da estatística \(X_{P}^{2}\) de Pearson pode
levar a inferências incorretas caso se utilize a distribuição
assintótica usual, o próximo passo é derivar um procedimento de teste
válido. Isto é feito introduzindo uma correção em \(X_{P}^{2}.\) Para
isto, observe que a esperança assintótica de \(X_{P}^{2}\) é
\(E_{p}\left( X_{P}^{2}\right) =d\). Como
\(E_{p}\left( X_{P}^{2}/d\right) =\)
\(E\left( \chi ^{2}\left( 1\right) \right)=1\), obtemos então a correção
simples de Rao-Scott para \(X_{P}^{2}\) dividindo o valor observado da
estatística de teste pelo efeito do plano amostral \(d\), isto é,

\begin{equation}
X_{P}^{2}(d)=X_{P}^{2}/d\mbox{,}  
\label{eq:qual3}
\end{equation}

que tem, no caso de duas celas, distribuição assintótica
\(\chi^{2}\left( 1\right)\).

Outra estatística comumente usada para testar a mesma hipótese de
bondade de ajuste no caso de proporções é a estatística do teste da
Razão de Verossimilhança (RV), dada por

\begin{equation}
X_{RV}^{2}=2n\sum\limits_{j=1}^{2}\widehat{p}_{j}\log \left( 
\widehat{p}_{j}/p_{0j}\right) =2n\log \left( \frac{\widehat{p}\left( 1-
\widehat{p}\right) }{p_{0}\left( 1-p_{0}\right) }\right) .
\label{eq:qual4}
\end{equation}

No caso de amostragem aleatória simples, a estatística \(X_{RV}^{2}\) é
também distribuída assintoticamente como \(\chi ^{2}\left(1\right)\),
quando a hipótese nula é verdadeira. Para planos amostrais complexos, a
estatística corrigida correspondente é

\begin{equation}
X_{RV}^{2}(d)=X_{RV}^{2}/d\;\;.  
\label{eq:qual5}
\end{equation}

Vamos calcular os valores das estatísticas de Pearson e de RV, com suas
correções de Rao-Scott, para os dados do Exemplo @ref\{ex:exebin\}. Para
as correções, primeiro é preciso calcular o efeito do plano amostral

\begin{Shaded}
\begin{Highlighting}[]
\NormalTok{p <-}\StringTok{ }\NormalTok{p0[}\DecValTok{1}\NormalTok{]}
\NormalTok{m <-}\StringTok{ }\DecValTok{50}
\CommentTok{# Efeito do plano amostral}
\NormalTok{d <-}\StringTok{ }\NormalTok{(p}\OperatorTok{*}\NormalTok{(}\DecValTok{1}\OperatorTok{-}\NormalTok{p)}\OperatorTok{/}\NormalTok{m)}\OperatorTok{/}\StringTok{ }\NormalTok{(p}\OperatorTok{*}\NormalTok{(}\DecValTok{1}\OperatorTok{-}\NormalTok{p)}\OperatorTok{/}\NormalTok{n)}
\NormalTok{d}
\end{Highlighting}
\end{Shaded}

\begin{verbatim}
## [1] 20
\end{verbatim}

\begin{Shaded}
\begin{Highlighting}[]
\CommentTok{# Estatística de Peason corrigida}
\NormalTok{x2pd <-}\StringTok{ }\NormalTok{x2p}\OperatorTok{/}\NormalTok{d}
\NormalTok{x2pd}
\end{Highlighting}
\end{Shaded}

\begin{verbatim}
## [1] 0.5
\end{verbatim}

\begin{Shaded}
\begin{Highlighting}[]
\CommentTok{# valor-p do teste}
\KeywordTok{pchisq}\NormalTok{(x2pd, }\DecValTok{1}\NormalTok{, }\DataTypeTok{lower.tail =}\NormalTok{ F)}
\end{Highlighting}
\end{Shaded}

\begin{verbatim}
## [1] 0.4795001
\end{verbatim}

\[
\begin{array}{llcll}
d & = & V_{p}\left( \widehat{p}\right) /V_{bin}\left( \widehat{p}\right) & =
& \frac{p\left( 1-p\right) /m}{p\left( 1-p\right) /n} \\ 
& = & \frac{0,0032}{0,00016} & = & 20
\end{array}
\] onde \(m=50\) é o número de empregados por empresa (tamanho do
conglomerado) e \(n=1.000\) é o número de empregados na amostra.

O valor da estatística de teste de Pearson é \[
X_{P}^{2}=\frac{\left( 0,84-0,80\right) ^{2}}{\left( 0,80\times 0,20\right)
/1.000}=10 
\] com \(p\)valor \(0,0016\). O valor da estatística de teste de Pearson
com a correção de Rao-Scott \(X_{P}^{2}(d)\) é então dado por \[
X_{P}^{2}(d)=X_{P}^{2}/d=10/20=0,5 
\] com \(p\)valor \(0,4795\). Observe que \(Z_{p}^{2}=0,707^{2}=0,5\) ,
e também que \(X_{P}^{2}(d)=Z_{bin}^{2}/d= 3,162^{2}/20=0,5\) ou seja,
\(Z_{p}^{2}=X_{P}^{2}(d)\) conforme esperado. Os valores da estatística
do teste da Razão de Verossimilhança e sua correção de Rao-Scott são
dados respectivamente por

\begin{Shaded}
\begin{Highlighting}[]
\CommentTok{# Estatística da RV}
\NormalTok{x2rv <-}\StringTok{ }\DecValTok{2}\OperatorTok{*}\NormalTok{n}\OperatorTok{*}\KeywordTok{sum}\NormalTok{(phat}\OperatorTok{*}\KeywordTok{log}\NormalTok{(phat}\OperatorTok{/}\NormalTok{p0))}
\NormalTok{x2rv}
\end{Highlighting}
\end{Shaded}

\begin{verbatim}
## [1] 10.56154
\end{verbatim}

\[
X_{RV}^{2}=2\times 1.000\times \log \left( \frac{0,84\times 0,16}{0,80\times
0,20}\right) =10,56\;, 
\] com \(p\)valor \(0,0012\), e

\begin{Shaded}
\begin{Highlighting}[]
\CommentTok{# Estatística da RV corrigida}
\NormalTok{x2rvd <-}\StringTok{ }\NormalTok{x2rv}\OperatorTok{/}\NormalTok{d}
\NormalTok{x2rvd}
\end{Highlighting}
\end{Shaded}

\begin{verbatim}
## [1] 0.528077
\end{verbatim}

\begin{Shaded}
\begin{Highlighting}[]
\CommentTok{# valor-p do teste}
\KeywordTok{pchisq}\NormalTok{(x2rvd, }\DecValTok{1}\NormalTok{, }\DataTypeTok{lower.tail =} \OtherTok{FALSE}\NormalTok{)}
\end{Highlighting}
\end{Shaded}

\begin{verbatim}
## [1] 0.4674165
\end{verbatim}

\[
X_{RV}^{2}(d)=X_{LR}^{2}/d=10,56/20=0,528\;\;, 
\] com \(p\)valor de \(0,4675\).

Como se pode notar, as estatísticas baseadas na Razão de Verossimilhança
oferecem resultados semelhantes às versões correspondentes baseadas na
estatística de Pearson. Em ambos os casos, as decisões baseadas nas
estatísticas sem correção seriam incorretas no sentido de rejeitar a
hipótese nula. Também em ambos os casos a correção de Rao-Scott produziu
efeito semelhante.

O efeito de plano amostral \(d=20\) observado neste exemplo é muito
grande e pouco comum na prática. Isto ocorreu neste caso porque o
coeficiente de correlação intraclasse assume o valor máximo \(\rho =1\)
(todos os valores dentro de um conglomerado são iguais, e portanto a
homogeneidade é máxima). Na prática, as correlações intraclasse
observadas são usualmente positivas mas menores que um, e portanto as
estimativas de efeito de plano amostral \(\widehat{d}\) correspondentes
são maiores que um. Para conglomerados de tamanho médio igual a \(20\)
(\(m=20\)) como neste exemplo, os valores típicos de \(\widehat{d}\) são
menores que \(3\), tendo em correspondência correlações intraclasse
estimadas positivas \(\widehat{\rho }<0,1\).

Os resultados do exemplo discutido nesta seção ilustram bem a
importância de considerar o plano amostral na construção de estatísticas
de teste para proporções simples, embora num caso um tanto extremo.
Ilustram também um dos enfoques existentes para tratar do problema, a
saber a correção de estatísticas de teste usuais (de Pearson e da Razão
de Verossimilhança).

\subsection{Estatística de Wald}\label{estatistica-de-wald}

Como alternativa à estatística de teste de Pearson, podemos usar a
estatística de bondade de ajuste \(X_{N}^{2}\) de Neyman. No caso de
duas celas, ela se reduz a

\begin{equation}
X_{N}^{2}=n\sum\limits_{j=1}^{2}\left( \hat{p}_{j}-p_{0j}\right) ^{2}/\hat{p}%
_{j}=\frac{\left( \widehat{p}-p_{0}\right) ^{2}}{\hat{p}\left( 1-\hat{p}
\right) /n}\;\;\mbox{.}  
\label{eq:qual6}
\end{equation}

Note que a expressão de \(X_{N}^{2}\) em \eqref{eq:qual6} pode ser obtida
substituindo-se no denominador de \(X_{P}^{2}\) em \eqref{eq:qual2} a
proporção hipotética \(p_{0}\) pela proporção estimada \(\hat{p}\).

A estatística de Neyman é um caso particular da estatística de bondade
de ajuste de Wald. Esta última estatística difere das estatísticas de
Pearson , da Razão de Verossimilhança e de Neyman por incorporar
automaticamente o plano amostral. Para o caso de duas celas, ela se
reduz a

\begin{equation}
X_{W}^{2}=\left( \widehat{p}-p_{0}\right) ^{2}/\hat{V}_{p}\left( \widehat{p}
\right) \mbox{,}  
\label{eq:qual7}
\end{equation}

onde \(\hat{V}_{p}\left( \widehat{p}\right)\) é uma estimativa da
variância de aleatorização de \(\hat{p}\), correspondente ao plano
amostral efetivamente utilizado.

O efeito do termo \(\hat{V}_{p}\left( \widehat{p}\right)\), que aparece
no denominador de \(X_{W}^{2}\), é incorporar na estatística de bondade
de ajuste o efeito do plano amostral utilizado. No caso particular de
amostragem aleatória simples, usamos no lugar de
\(\widehat{V}_{p}\left( \widehat{p}\right)\) a variância
\(\widehat{V}_{bin}\left( \widehat{p}\right) =\)
\(\widehat{p}\left( 1-\widehat{p}\right) /n\). Neste caso, estatística
resultante \(X_{bin}^{2}\) coincide com a estatística \(X_{N}^{2}\) de
Neyman.

Para o plano amostral de conglomerados considerado no Exemplo
@ref\{ex:exebin\}, a estatística \(X_{W}^{2}\), sem qualquer ajuste
auxiliar, já é distribuída assintoticamente como qui-quadrado com um
grau de liberdade. O valor da estatística de Wald para esse exemplo é

\begin{Shaded}
\begin{Highlighting}[]
\NormalTok{x2n <-}\StringTok{ }\DecValTok{1000}\OperatorTok{*}\StringTok{ }\KeywordTok{sum}\NormalTok{((phat}\OperatorTok{-}\NormalTok{p0)}\OperatorTok{^}\DecValTok{2}\OperatorTok{/}\NormalTok{phat)}
\NormalTok{vhatp <-}\StringTok{ }\NormalTok{(phat[}\DecValTok{1}\NormalTok{]}\OperatorTok{*}\NormalTok{phat[}\DecValTok{2}\NormalTok{])}\OperatorTok{/}\NormalTok{m }\CommentTok{# variância usa número efetivo}
\CommentTok{# Estatística de Wald}
\NormalTok{x2w <-}\StringTok{ }\NormalTok{((phat[}\DecValTok{1}\NormalTok{]}\OperatorTok{-}\NormalTok{p0[}\DecValTok{1}\NormalTok{])}\OperatorTok{^}\DecValTok{2}\NormalTok{)}\OperatorTok{/}\NormalTok{vhatp}
\NormalTok{x2w}
\end{Highlighting}
\end{Shaded}

\begin{verbatim}
## [1] 0.5952381
\end{verbatim}

\begin{Shaded}
\begin{Highlighting}[]
\KeywordTok{pchisq}\NormalTok{(x2w,}\DecValTok{1}\NormalTok{, }\DataTypeTok{lower.tail =} \OtherTok{FALSE}\NormalTok{)}
\end{Highlighting}
\end{Shaded}

\begin{verbatim}
## [1] 0.4404007
\end{verbatim}

\[
X_{W}^{2}=\left( 0,84-0,80\right) ^{2}/0,002743=0,583\;\;\mbox{.} 
\] Observe que o valor desta estatística é bem próximo dos valores das
estatísticas de Pearson e da Razão de Verossimilhança com a correção de
Rao-Scott.

A estatística de Wald, pelo uso de uma estimativa apropriada da
variância, reflete a complexidade do plano amostral e fornece uma
estatística de teste assintoticamente válida, não necessitando que seja
feito qualquer ajuste auxiliar. Esta pode ser considerada uma vantagem
em relação às estatísticas com correção de Rao-Scott. Entretanto, no
caso de mais de duas celas, pode haver desvantagens no uso da
estatística de Wald baseada no plano amostral, devido à instabilidade
nas estimativas de variância em pequenas amostras.

Reproduzimos na Tabela \ref{tab:testprop} os resultados para todas as
estatísticas de teste consideradas até agora, para facilidade de
comparação.

\begin{table}

\caption{\label{tab:testprop}Valores observados e valores-p de estatísticas de testes para os dados do Exemplo 4.4}
\centering
\begin{tabular}[t]{lccclccclccclccc}
\toprule
Estatística & gl & vobs & valorp\\
\midrule
Pearson & 1 & 10.00 & 0.0016\\
Pearson ajustada & 1 & 0.50 & 0.4795\\
RV & 1 & 10.56 & 0.0012\\
RV ajustada & 1 & 0.53 & 0.4674\\
Wald & 1 & 0.60 & 0.4404\\
\bottomrule
\end{tabular}
\end{table}

Nesta seção foram apresentadas as duas principais abordagens para
incorporar o efeito do plano amostral na estatística de teste:

\begin{enumerate}
\def\labelenumi{\arabic{enumi}.}
\item
  a metodologia de ajuste de Rao-Scott para as estatísticas de teste de
  Pearson e da Razão de Verossimilhança;
\item
  e a estatística de Wald baseada no plano amostral.
\end{enumerate}

Ambas as abordagens são facilmente generalizáveis para tabelas de uma ou
duas entradas com número de linhas e colunas maior que dois. Vamos
considerar na próxima seção o caso geral de testes de bondade de ajuste
e apresentar mais detalhes sobre as estatísticas de teste alternativas.
Depois, introduziremos os testes de independência e de homogeneidade
para tabelas de duas entradas. A ênfase será dada nos procedimentos
baseados na estatísticas de teste de Wald baseadas no plano amostral e
nas estatísticas de Pearson e da RV com os vários ajustes de Rao-Scott.

\section{Teste para Várias
Proporções}\label{teste-para-varias-proporcoes}

Neste seção vamos considerar extensões do problema de testes de bondade
de ajuste, aumentando o número de proporções envolvidas. O caso de
tabelas de duas entradas será considerado no capítulo seguinte.

A hipótese de bondade de ajuste para \(J\geq 2\) celas pode ser escrita
como \(H_{0}:p_{j}=p_{0j}\;,\;j=1,\ldots ,J\), onde \(p_{j}=N_{j}/N\)
são as proporções populacionais desconhecidas nas celas e \(p_{0j}\) são
as proporções hipotéticas das celas. Essa hipótese pode também ser
escrita, usando notação vetorial, como \(H_{0}:\mathbf{p=p}_{0}\) , onde
\(\mathbf{p=}\left( p_{1},\ldots ,p_{J-1}\right) ^{^{\prime}}\)é o vetor
de proporções populacionais desconhecidas e
\(\mathbf{p}_{0}\mathbf{=}\left( p_{01},\ldots ,p_{0\,J-1}\right) ^{^{\prime}}\)é
o vetor de proporções hipotéticas.

O vetor de estimativas consistentes das proporções das celas, baseado em
\(n\) observações, é denotado por
\(\widehat{\mathbf{p}}\mathbf{=}\left(\widehat{p}_{1},\ldots,\widehat{p}_{J-1}\right) ^{^{\prime}}\),
onde \(\widehat{p}_{j}=\widehat{n}_{j}/n\). Os \(\widehat{n}_{j}\) são
as frequências ponderadas nas celas, considerando as diferentes
probabilidades de inclusão dos elementos e ajustes por não-resposta,
onde os pesos amostrais são normalizados de modo que
\(\sum_{j=1}^{J}\widehat{n}_{j}=n\). Se \(n\) não for fixado de antemão,
os \(\widehat{p}\) serão estimadores de razões, o que é comum quando
trabalhamos com subgrupos da população. Observe que apenas \(J-1\)
componentes são incluídos em cada um dos vetores \(\mathbf{p,\;p}_{0}\)
e \(\widehat{\mathbf{p}}\), pois a soma das proporções nas \(J\)
categorias é igual a \(1\), e portanto a proporção na \(J\)-ésima
categoria é obtida por diferença.

\subsection{Estatística de Wald Baseada no Plano
Amostral}\label{estatistica-de-wald-baseada-no-plano-amostral}

A estatística de Wald baseada no plano amostral \(X_{W}^{2}\), para o
teste da hipótese simples de bondade de ajuste, foi anteriormente
introduzida no caso de duas celas como uma alternativa à estatística de
Pearson ajustada. No caso de mais de duas celas, a estatística de
bondade de ajuste de Wald é dada por

\begin{equation}
X_{W}^{2}=\left( \widehat{\mathbf{p}}\mathbf{-p}_{0}\right) ^{^{\prime }}
\widehat{\mathbf{V}}_{p}^{-1}\left( \widehat{\mathbf{p}}-\mathbf{p}_{0}\right) \;\mbox{,} 
\label{eq:qual8}
\end{equation}

onde \(\widehat{\mathbf{V}}_{p}\) denota um estimador consistente da
matriz de covariância de aleatorização verdadeira \(\mathbf{V}_{p}\) do
estimador \(\widehat{\mathbf{p}}\) do vetor de proporções
\(\mathbf{p}\). Uma estimativa \(\widehat{\mathbf{V}}_{p}\) pode ser
obtida pelo método de linearização, usando-se por exemplo o pacote
\textbf{SUDAAN}.

Sob a hipótese nula \(H_{0}\), a estatística \(X_{W}^{2}\) tem
distribuição assintótica qui-quadrado com \(J-1\) graus de liberdade,
fornecendo assim um procedimento de teste válido no caso de amostras
complexas. Na prática, espera-se que \(X_{W}^{2}\) funcione
adequadamente se o número de unidades primárias de amostragem
selecionadas for grande e o número de celas componentes do vetor
\(\mathbf{p}\) for relativamente pequeno. Neste caso, podemos obter um
estimador estável de \(\mathbf{V}_{p}\). Observe que \eqref{eq:qual7} é um
caso particular de \eqref{eq:qual8}.

\subsection{Situações Instáveis}\label{situacoes-instaveis}

Se o número \(m\) de unidades primárias de amostragem disponíveis for
pequeno, pode ocorrer um problema de instabilidade na estimativa
\(\widehat{\mathbf{V}}_{p}\), devido ao pequeno número de graus de
liberdade \(f=m-H\) disponível para a estimação da variância. A
instabilidade da estimativa \(\widehat{\mathbf{V}}_{p}\) pode tornar a
estatística de Wald muito liberal.

é comum contornar esta instabilidade corrigindo a estatística de Wald,
mediante emprego da chamada \texttt{estatística\ de\ Wald\ F-corrigida}
. Há duas propostas alternativas de estatísticas \textbf{F}-corrigidas
de Wald. A primeira é dada por

\begin{equation}
F_{1.p}=\frac{f-J+2}{f\left( J-1\right) }X_{W}^{2}\;\;,  
\label{eq:qual9}
\end{equation}

que tem distribuição assintótica de referência \(F\) com \(J-1\) e
\(f-J+2\) graus de liberdade. A segunda é dada por

\begin{equation}
F_{2.p}=\frac{X_{W}^{2}}{\left( J-1\right) },  
\label{eq:qual10}
\end{equation}

que tem distribuição assintótica de referência \(F\) com \(J-1\) e \(f\)
graus de liberdade. No caso \(J=2\), as duas correções reproduzem a
estatística original.

O efeito de uma correção \(\mathbf{F}\) à estatística \(X_{W}^{2}\) pode
ser visualizado facilmente no caso de duas celas. Se \(f\) for pequeno,
então o \(p\)valor de \(X_{W}^{2}\), obtido a partir de uma distribuição
\textbf{F} com \(1\) e \(f\) graus, é maior que o \(p\) valor obtido
numa distribuição qui-quadrado com um grau de liberdade. Quando \(f\)
aumenta a diferença diminui, tornando a correção desprezível, quando
\(f\) for grande.

\citep{TR87} analisaram o desempenho das diferentes estatísticas de
teste de bondade de ajuste, no caso de instabilidade. Eles verificaram
que a estatística de Wald F-corrigida \(F_{1.p}\) não apresentou, em
geral, o melhor desempenho nesta comparação, contudo, comportou-se
relativamente bem nos casos padrões, onde a instabilidade não era muito
grave. As estatísticas \textbf{F}-corrigidas de Wald são bastante
utilizadas na prática, e estão implementadas em pacotes para análise de
dados de pesquisas amostrais complexas.

\subsection{Estatística de Pearson com Ajuste de
Rao-Scott}\label{raoscott}

O exemplo introdutório serviu para mostrar que, na presença de efeitos
de plano amostral importantes, as estatísticas clássicas de teste
precisam ser ajustadas para terem a mesma distribuição assintótica de
referência que a obtida para o caso de amostragem aleatória simples.
Inicialmente, vamos considerar a estatística de teste \(X_{P}^{2}\) de
Pearson\(.\) Essa estatística pode ser escrita em forma matricial como

\begin{equation}
X_{P}^{2}=n\sum\limits_{j=1}^{J}\left( \widehat{p}_{j}-p_{0j}\right)
^{2}/p_{0j}=n\left( \widehat{\mathbf{p}}-\mathbf{p}_{0}\right) ^{^{\prime }}
\mathbf{P}_{0}^{-1}\left( \widehat{\mathbf{p}}-\mathbf{p}_{0}\right) 
\label{eq:qual11}
\end{equation}

onde
\(\mathbf{P}_{0}=diag\left( \mathbf{p}_{0}\right) -\mathbf{p}_{0}\mathbf{p}_{0}^{^{\prime }}\)
e \(\mathbf{P}_{0}/n\) é a matriz
\(\left( J-1\right)\times \left( J-1\right)\) de covariância multinomial
de \(\widehat{\mathbf{p}}\) sob a hipótese nula, e
\(diag\left( \mathbf{p}_{0}\right)\) representa uma matriz diagonal com
elementos \(p_{0j}\) na diagonal.

A matriz de covariância \(\mathbf{P}_{0}/n\) é uma generalização do caso
\(J=2\) celas para o caso de mais de duas celas (\(J>2\)). Observe que a
expressão de \(X_{P}^{2}\) tem a mesma forma da estatística de Wald, com
\(\mathbf{P}_{0}/n\) no lugar de \(\widehat{\mathbf{V}}_{p}\). No caso
de apenas duas celas, \(X_{P}^{2}\) reduz-se à fórmula simples antes
considerada
\(X_{P}^{2}=\left( \widehat{p}_{1}-p_{01}\right) ^{2}/\left[ p_{01}\left( 1-p_{01}\right) /n\right]\),
onde o denominador corresponde à variância da binomial sob a hipótese
nula.

Para examinar a distribuição assintótica da estatística \(X_{P}^{2}\) de
Pearson, vamos generalizar os resultados anteriores, do caso de duas
celas para o caso \(J>2\). Neste caso, \(X_{P}^{2}\) é assintoticamente
distribuído como uma soma ponderada
\(\delta_{1}W_{1}+\delta _{2}W_{2}+\ldots +\delta _{J-1}W_{J-1}\) de
\(J-1\) variáveis aleatórias independentes \(W_{j}\) , cada uma tendo
distribuição qui-quadrado com um grau de liberdade. Os pesos
\(\delta _{j}\) são os autovalores da matriz de efeito multivariado de
plano amostral \(\mathbf{\Delta }=\mathbf{P}_{0}^{-1}\mathbf{V}_{p}\),
onde \(\mathbf{V}_{p}\mathbf{/}n\) é a matriz de covariância do
estimador \(\widehat{\mathbf{p}}\) do vetor de proporção \(\mathbf{p}\)
baseada no plano amostral verdadeiro. Tais autovalores são também
chamados efeitos generalizados de plano amostral. Observe que, em geral,
eles não coincidem com os efeitos univariados de plano amostral
\(d_{j}\).

No caso de amostragem aleatória simples, os efeitos generalizados de
plano amostral \(\delta_{j}\) são todos iguais a um, pois neste caso
\(\mathbf{\Delta =I}\), matriz identidade. Neste caso, a soma
\(\sum_{j=1}^{J-1}\delta _{j}W_{j}\) se reduz a
\(\sum_{j=1}^{J-1}W_{j},\) cuja distribuição é \(\chi ^{2}\) com \(J-1\)
graus de liberdade. Assim, sob amostragem aleatória simples, a
estatística \(X_{P}^{2}\) é distribuída assintoticamente como
qui-quadrado com \(J-1\) graus de liberdade.

No caso de plano amostral mais complexo, envolvendo estratificação e/ou
conglomeração, os efeitos generalizados de plano amostral não são iguais
a um. Devido aos efeitos de conglomeração, os \(\delta _{j}\) tendem a
ser maiores que um, e assim a distribuição assintótica da variável
aleatória \(\sum_{j=1}^{J-1}\delta_{j}W_{j}\) será diferente de uma
qui-quadrado com \(J-1\) graus de liberdade. Desta forma, a estatística
\(X_{P}^{2}\) requer correções semelhantes às introduzidas no caso de
duas celas. No caso geral, há mais de uma possibilidade de correção e
consideraremos as
\texttt{correções\ de\ primeira\ ordem\ e\ de\ segunda\ ordem\ de\ Rao-Scott}
, desenvolvidas por \citep{Rao}. A correção de primeira ordem tem por
objetivo corrigir a esperança assintótica da estatística \(X_{P}^{2}\)
de Pearson, e a de segunda ordem também envolve correção da variância.
Tecnicamente, os dois ajustes são baseados nos autovalores da matriz de
efeito multivariado de plano amostral estimada
\(\widehat{\mathbf{\Delta }}\).

Inicialmente, consideramos um \texttt{ajuste\ simples\ de\ EPA\ médio} à
estatística \(X_{P}^{2}\), devido a \citep{fellegi} e \citep{holt80a}, e
o ajuste de primeira ordem de Rao-Scott. Estes ajustes são úteis nos
casos em que não é possível obter uma estimativa adequada
\(\mathbf{\hat{V}}_{p}\) para a matriz de covariância de aleatorização.
Quando esta estimativa está disponível, deve-se usar o ajuste mais
preciso de segunda ordem.

O ajuste de EPA médio é baseado nos efeitos univariados de plano
amostral estimados \(\hat{d}_{j}\) das estimativas \(\hat{p}_{j}\). O
ajuste da estatística \eqref{eq:qual11} é feito dividindo o valor
observado da estatística \(X_{P}^{2}\) de Pearson pela média
\(\hat{d}_{.}\) dos efeitos univariados de plano amostral:

\begin{equation}
X_{P}^{2}\left( \hat{d}_{.}\right) =X_{P}^{2}/\hat{d}_{.}  
\label{eq:qual12}
\end{equation}

onde \(\hat{d}_{.}=\sum_{j=1}^{J}\hat{d}_{j}/J\) é um estimador da média
\(\bar{d}\) dos efeitos de plano amostral desconhecidos.

Estimamos os efeitos do plano amostral por
\(\hat{d}_{j}=\hat{V}_{p}\left(\hat{p}_{j}\right) /\left( \hat{p}_{j}\left( 1-\hat{p}_{j}\right) /n\right)\),
onde \(\hat{V}_{p}\left( \hat{p}_{j}\right)\) é a estimativa da
variância de aleatorização do estimador de proporção \(\hat{p}_{j}\) .
Este ajustamento requer que estejam disponíveis as estimativas dos
efeitos de plano amostral dos estimadores das proporções das \(J\)
celas. A correlação intraclasse positiva fornece uma média
\(\hat{d}_{.}\) maior que \(1\) e, portanto, o ajuste do EPA médio tende
a remover a liberalidade de \(X_{P}^{2}\).

O ajuste do EPA médio não corrige exatamente a esperança assintótica de
\(X_{P}^{2}\) , pois a média dos efeitos univariados de plano amostral
não é igual à média dos efeitos generalizados de plano amostral. Sob a
hipótese nula, a esperança assintótica de \(X_{P}^{2}\) é
\(E\left( X_{P}^{2}\right) =\sum_{j=1}^{J-1}\delta _{j}\) , logo
\(E\left( X_{P}^{2}/\bar{\delta}\right) =E\left( \chi ^{2}\left(J-1\right) \right) =J-1\),
onde a média dos autovalores é
\(\bar{\delta}=\sum_{j=1}^{J-1}\delta _{j}/\left( J-1\right)\). Este
raciocínio conduz ao ajuste de primeira ordem de Rao-Scott para
\(X_{P}^{2}\), dado por

\begin{equation}
X_{P}^{2}\left( \hat{\delta}_{.}\right) =X_{P}^{2}/\hat{\delta}_{.}\;\;\mbox{,}
\label{eq:qual13}
\end{equation}

onde \(\hat{\delta}_{.}\) é um estimador da média \(\bar{\delta}\) dos
autovalores desconhecidos da matriz de efeitos multivariados de plano
amostral \(\mathbf{\Delta }\).

Podemos estimar a média dos efeitos generalizados usando os efeitos
univariados de plano amostral estimados, pela equação

\[
\left( J-1\right) \hat{\delta}_{.}=\sum_{j=1}^J\frac{\hat{p}_{j}}{p_{0j}}\left( 1-\hat{p}_{0j}\right)
 \hat{d}_{j}\mbox{,} 
\] sem estimar os próprios autovalores. Alternativamente,
\(\hat{\delta}_{.}\) pode ser obtido a partir da estimativa da matriz de
efeitos multivariados
\(\mathbf{\hat{\Delta}}=n\mathbf{P}_{0}^{-1}\mathbf{\hat{V}}_{p}\), pela
equação
\(\hat{\delta}_{.}=tr\left( \mathbf{\hat{\Delta}}\right) /\left(J-1\right)\),
isto é, dividindo o traço de \(\mathbf{\hat{\Delta}}\) pelo número de
graus de liberdade.

A estatística ajustada \(X_{P}^{2}\left( \hat{\delta}_{.}\right)\) só
tem distribuição assintoticamente qui-quadrado com \(\left(J-1\right)\)
graus de liberdade se os autovalores forem iguais. Na prática, esta
estatística funciona bem se a variação dos autovalores estimados for
pequena. No cálculo de \(X_{P}^{2}\left( \hat{\delta}_{.}\right)\) só
são necessários os efeitos multivariados de plano amostral dos
\(\hat{p}_{j}\) que aparecem na diagonal da matriz
\(\mathbf{\hat{\Delta}}\). Assim, esta estatística é adequada em
análises secundárias de tabelas de contingência, se forem divulgadas as
estimativas de efeito de plano amostral correspondentes. O ajuste de
primeira ordem de Rao-Scott \(X_{P}^{2}\left( \hat{\delta}_{.}\right)\)
é mais exato do que o ajuste do EPA médio da estatística
\(X_{P}^{2}\left( \hat{d}_{.}\right)\), que é considerada uma
alternativa conservadora de \(X_{P}^{2}\left( \hat{\delta}_{.}\right)\).

A correção de primeira ordem de Rao-Scott \eqref{eq:qual13} é introduzida
na estatística de Pearson com o objetivo de tornar a média assintótica
da estatística ajustada igual ao número de graus de liberdade da
distribuição de referência. Se a variação dos autovalores estimados
\(\hat{\delta}_{j}\) for grande, então será também necessária uma
correção da variância de \(X_{P}^{2}\). Isto é obtido através de uma
\texttt{correção\ de\ segunda\ ordem\ de\ Rao-Scott}, baseada no método
de \citep{sat}. A estatística de Pearson com ajuste de Rao-Scott de
segunda ordem é dada por

\begin{equation}
X_{P}^{2}\left( \hat{\delta}_{.},\hat{a}^{2}\right) =
X_{P}^{2}\left( \hat{\delta}_{.}\right) /\left( 1+\hat{a}^{2}\right) , 
 \label{eq:qual14}
\end{equation}

onde \(\hat{a}^{2}\) é um estimador do quadrado do coeficiente de
variação \(a^{2}\) dos autovalores desconhecidos dado por \[
\hat{a}^{2}=\sum_{j=1}^{J-1}\hat{\delta}_{j}^{2}/\left( \left( J-1\right)
 \hat{\delta}_{.}^{2}\right) -1\;\;. 
\]

Um estimador da soma dos quadrados dos autovalores é dado por \[
\sum_{j=1}^{J-1}\hat{\delta}_{j}^{2}=tr\left( \mathbf{\hat{\Delta}}^{2}\right) =n^{2}\sum\limits_{j=1}^{J}\sum\limits_{k=1}^{J}
\hat{V}_{p}^{2}\left( \hat{p}_{j},\hat{p}_{k}\right) /p_{0j}p_{0k}, 
\] onde \(\hat{V}_{p}\left( \hat{p}_{j},\hat{p}_{k}\right)\) são os
estimadores das covariâncias de aleatorização de \(\hat{p}_{j}\) e
\(\hat{p}_{k}\). Os graus de liberdade também devem ser corrigidos. A
estatística \(X_{P}^{2}\left( \hat{\delta}_{.},\hat{a}^{2}\right)\) é
assintoticamente qui-quadrado com graus de liberdade com ajuste de
Satterthwaite dados por
\(gl_{S}=\left( J-1\right) /\left( 1+\hat{a}^{2}\right)\).

Observe que, para o ajuste de segunda ordem, é necessária estimativa
completa da matriz de variância \(\mathbf{\hat{V}}_{p}\), enquanto que
para o ajuste de primeira ordem só precisamos conhecer estimativas das
variâncias \(\hat{V}_{p}\).

Em situações instáveis, pode ser necessário fazer uma correção F ao
ajuste de primeira ordem de Rao-Scott \eqref{eq:qual13}. A estatística
F-corrigida é definida por

\begin{equation}
FX_{P}^{2}\left( \hat{\delta}_{.}\right) =X_{W}^{2}/\left( \left( J-1\right) 
\hat{\delta}_{.}\right) \;\;.  
\label{eq:qual15}
\end{equation}

A estatística \(FX_{P}^{2}\left( \hat{\delta}_{.}\right)\) tem
distribuição de referência \(F\) com \(J-1\) e \(f\) graus de liberdade.
\citep{TR87} observaram que esta estatística, em situações instáveis, é
melhor que a estatística sem correção de primeira ordem.

\BeginKnitrBlock{example}
\protect\hypertarget{exm:unnamed-chunk-6}{}{\label{exm:unnamed-chunk-6}
}Teste de bondade de ajuste para a distribuição etária da PPV 96-97 na
Região Sudeste.
\EndKnitrBlock{example}

Vamos considerar um teste da bondade de ajuste da distribuição das
idades para a Pesquisa sobre Padrões de Vida (PPV) 96/97, para os
subgrupos de 0 a 14; de 15 a 29; de 30 a 44; de 45 a 59 e de 60 e mais
anos de idade. As proporções correspondentes para a população foram
obtidas da Contagem Populacional de 96. Na Região Sudeste, o número de
estratos é \(H=15\) e o número total de conglomerados (setores) na
amostra da PPV é \(m=276\) e portanto \(f=m-H=261\). As informações
utilizadas neste exemplo são apresentadas na Tabela \ref{tab:ppcount}.

\begin{Shaded}
\begin{Highlighting}[]
\KeywordTok{library}\NormalTok{(survey)}
\NormalTok{ppv1 <-}\StringTok{ }\KeywordTok{readRDS}\NormalTok{(}\StringTok{"~}\CharTok{\textbackslash{}\textbackslash{}}\StringTok{GitHub}\CharTok{\textbackslash{}\textbackslash{}}\StringTok{adac}\CharTok{\textbackslash{}\textbackslash{}}\StringTok{data}\CharTok{\textbackslash{}\textbackslash{}}\StringTok{ppv.rds"}\NormalTok{)}
\CommentTok{# cria idade categorizada}
\NormalTok{ppv1<-}\KeywordTok{transform}\NormalTok{(ppv1,}\DataTypeTok{idatab=}\KeywordTok{cut}\NormalTok{(v02a08,}
\KeywordTok{c}\NormalTok{(}\DecValTok{0}\NormalTok{,}\DecValTok{14}\NormalTok{,}\DecValTok{29}\NormalTok{,}\DecValTok{44}\NormalTok{,}\DecValTok{59}\NormalTok{,}\DecValTok{200}\NormalTok{), }\DataTypeTok{include.lowest=}\NormalTok{T), }\DataTypeTok{one=}\DecValTok{1}\NormalTok{)}

\CommentTok{# Objeto de desenho da PPV}
\NormalTok{ppv.des<-}\KeywordTok{svydesign}\NormalTok{(}\DataTypeTok{id=}\OperatorTok{~}\NormalTok{nsetor,}\DataTypeTok{strat=}\OperatorTok{~}\NormalTok{estratof,}\DataTypeTok{weights=}\OperatorTok{~}\NormalTok{pesof,}
\DataTypeTok{data=}\NormalTok{ppv1,}\DataTypeTok{nest=}\OtherTok{TRUE}\NormalTok{)}
\CommentTok{# Considera região sudeste}
\NormalTok{ppv.se.des<-}\KeywordTok{subset}\NormalTok{(ppv.des,regiao}\OperatorTok{==}\DecValTok{2}\NormalTok{)}
\CommentTok{# estima proporções nas classes de IDATAB}
\NormalTok{ppv.id<-}\KeywordTok{svymean}\NormalTok{(}\OperatorTok{~}\NormalTok{idatab,ppv.se.des,}\DataTypeTok{deff=}\NormalTok{T)}
\NormalTok{vhat<-}\KeywordTok{vcov}\NormalTok{(ppv.id)}
\NormalTok{ppv.id <-}\StringTok{ }\KeywordTok{data.frame}\NormalTok{(ppv.id)}
\NormalTok{freq <-}\StringTok{ }\KeywordTok{svyby}\NormalTok{(}\OperatorTok{~}\NormalTok{one, }\OperatorTok{~}\NormalTok{idatab, ppv.se.des, unwtd.count)}\OperatorTok{$}\NormalTok{counts}
\CommentTok{# matriz de variância-covariância estimada}
\NormalTok{idad_tab <-}\StringTok{ }\KeywordTok{c}\NormalTok{(}\StringTok{"0-14"}\NormalTok{, }\StringTok{"15-29"}\NormalTok{, }\StringTok{"30-44"}\NormalTok{,}\StringTok{"45-59"}\NormalTok{,}\StringTok{"60+"}\NormalTok{)}
\NormalTok{tab73 <-}\StringTok{ }\KeywordTok{data.frame}\NormalTok{(}\DataTypeTok{idade=}\NormalTok{ idad_tab, }\DataTypeTok{prop_contagem=} \KeywordTok{c}\NormalTok{(}\FloatTok{0.2842}\NormalTok{, }\FloatTok{0.2747}\NormalTok{, }\FloatTok{0.2263}\NormalTok{, }\FloatTok{0.1261}\NormalTok{, }\FloatTok{0.0860}\NormalTok{), frequência= freq, }\DataTypeTok{prop_est_ppv=}\NormalTok{ ppv.id}\OperatorTok{$}\NormalTok{mean,}\DataTypeTok{epa =}\NormalTok{ ppv.id}\OperatorTok{$}\NormalTok{deff) }
\NormalTok{knitr}\OperatorTok{::}\KeywordTok{kable}\NormalTok{(tab73, }\DataTypeTok{booktabs=} \OtherTok{TRUE}\NormalTok{, }\DataTypeTok{digits=}\KeywordTok{c}\NormalTok{(}\DecValTok{0}\NormalTok{,}\DecValTok{4}\NormalTok{,}\DecValTok{0}\NormalTok{,}\DecValTok{4}\NormalTok{,}\DecValTok{4}\NormalTok{), }\DataTypeTok{align =} \StringTok{"lcccc"}\NormalTok{,}
 \DataTypeTok{caption =} \StringTok{"Vetores de proporções por classes de idade da PPV 96/97 e Contagem}
\StringTok{ 96 e EPAs calculados para a PPV - Região Sudeste"}\NormalTok{)}
\end{Highlighting}
\end{Shaded}

\begin{table}

\caption{\label{tab:ppcount}Vetores de proporções por classes de idade da PPV 96/97 e Contagem
 96 e EPAs calculados para a PPV - Região Sudeste}
\centering
\begin{tabular}[t]{lcccclcccclcccclcccclcccc}
\toprule
idade & prop\_contagem & frequência & prop\_est\_ppv & epa\\
\midrule
0-14 & 0.2842 & 2516 & 0.2845 & 2.2862\\
15-29 & 0.2747 & 2360 & 0.2678 & 2.1867\\
30-44 & 0.2263 & 2018 & 0.2225 & 2.2542\\
45-59 & 0.1261 & 1177 & 0.1316 & 1.9906\\
60+ & 0.0860 & 832 & 0.0935 & 3.1632\\
\bottomrule
\end{tabular}
\end{table}

Os valores dos EPAs observados na PPV ( coluna 5 da Tabela
\ref{tab:ppcount} mostram que o plano amostral não pode ser ignorado na
análise. Queremos testar a hipótese \(H_{0}:\mathbf{p=p}_{0}\) usando as
estimativas de proporções obtidas pela amostra da PPV. O vetor de
proporções populacionais \(\mathbf{p}_{0}\) foi obtido dos resultados da
Contagem Populacional de 96, que é uma pesquisa censitária. Neste
exemplo, vamos calcular a estatística de Pearson e suas correções, e
também a estatística de Wald baseada no plano amostral. Calculamos a
matriz \(\widehat{\mathbf{V}}_{p}\) pela aplicação do método de
linearização de Taylor descrito na Seção \ref{taylor} através da fórmula
\eqref{eq:estpa22} obtendo

\begin{tabular}{l|r|r|r|r|r}
\hline
  & 0-14 & 15-29 & 30-44 & 45-59 & 60+\\
\hline
0-14 & 52.274 & -3.899 & -5.672 & -19.292 & -23.411\\
\hline
15-29 & -3.899 & 48.164 & -29.346 & -3.399 & -11.520\\
\hline
30-44 & -5.672 & -29.346 & 43.799 & -8.226 & -0.556\\
\hline
45-59 & -19.292 & -3.399 & -8.226 & 25.551 & 5.366\\
\hline
60+ & -23.411 & -11.520 & -0.556 & 5.366 & 30.120\\
\hline
\end{tabular}

Para obter a estatística de Pearson \eqref{eq:qual11}, vamos calcular a
matriz de covariância populacional e uma estimativa dessa matriz de
covariância sob suposição de distribuição multinomial, dada por
\(\mathbf{P}_{0}/n=\frac{diag\left( \mathbf{p}_{0}\right) -\mathbf{p}_{0}\mathbf{p}_{0}^{^{\prime }}}{8.903}\),
resultando em

\begin{tabular}{r|r|r|r|r}
\hline
22.850 & -8.855 & -7.224 & -4.025 & -2.745\\
\hline
-8.855 & 22.515 & -7.051 & -3.929 & -2.680\\
\hline
-7.224 & -7.051 & 19.666 & -3.205 & -2.186\\
\hline
-4.025 & -3.929 & -3.205 & 12.378 & -1.218\\
\hline
-2.745 & -2.680 & -2.186 & -1.218 & 8.829\\
\hline
\end{tabular}

Para obter os diversos ajustes desta estatística precisamos usar os
valores dos EPAs, listados na coluna 5 da Tabela \ref{tab:ppcount}.
Estes valores foram obtidos através do pacote SUDAAN. Para obter as
diferentes correções da estatística de Pearson, precisamos calcular as
seguintes quantidades:

\begin{Shaded}
\begin{Highlighting}[]
\NormalTok{dhat <-}\StringTok{ }\NormalTok{tab73}\OperatorTok{$}\NormalTok{epa}
\NormalTok{dhatdot <-}\StringTok{ }\KeywordTok{mean}\NormalTok{(dhat)}
\NormalTok{dhatdot}
\end{Highlighting}
\end{Shaded}

\begin{verbatim}
## [1] 2.376168
\end{verbatim}

\[
\hat{d}_{.}=\sum_{j=1}^{5}\hat{d}_{j}/5=2,376\;\;\mbox{,} 
\]

\begin{Shaded}
\begin{Highlighting}[]
\NormalTok{phat <-}\StringTok{ }\NormalTok{tab73}\OperatorTok{$}\NormalTok{prop_est_ppv}
\NormalTok{deltahatdot <-}\StringTok{ }\KeywordTok{sum}\NormalTok{(phat}\OperatorTok{*}\NormalTok{(}\DecValTok{1}\OperatorTok{-}\NormalTok{p0)}\OperatorTok{*}\NormalTok{dhat}\OperatorTok{/}\NormalTok{(}\DecValTok{4}\OperatorTok{*}\NormalTok{p0))}
\NormalTok{deltahatdot}
\end{Highlighting}
\end{Shaded}

\begin{verbatim}
## [1] 2.459607
\end{verbatim}

\[
\hat{\delta}_{.}=\sum_{j=1}^5\frac{\hat{p}_{j}}{4p_{0j}}
\left( 1-\hat{p}_{0j}\right) \hat{d}_{j}=2,457\;\;, 
\]

\begin{Shaded}
\begin{Highlighting}[]
\NormalTok{nppv <-}\StringTok{ }\KeywordTok{sum}\NormalTok{ (freq)}
\NormalTok{raz <-}\StringTok{ }\KeywordTok{matrix}\NormalTok{(}\OtherTok{NA}\NormalTok{,}\DecValTok{5}\NormalTok{,}\DecValTok{5}\NormalTok{)}
\ControlFlowTok{for}\NormalTok{(i }\ControlFlowTok{in} \DecValTok{1}\OperatorTok{:}\DecValTok{5}\NormalTok{)\{}
  \ControlFlowTok{for}\NormalTok{(j }\ControlFlowTok{in} \DecValTok{1}\OperatorTok{:}\DecValTok{5}\NormalTok{)\{}
\NormalTok{    raz[i,j]<-}\StringTok{ }\NormalTok{vhat[i,j]}\OperatorTok{^}\DecValTok{2}\OperatorTok{/}\NormalTok{(p0[i]}\OperatorTok{*}\NormalTok{p0[j])}
\NormalTok{  \}}
\NormalTok{\}}
\NormalTok{ahat2_}\DecValTok{1}\NormalTok{ <-}\StringTok{ }\NormalTok{nppv}\OperatorTok{^}\DecValTok{2}\OperatorTok{*}\StringTok{ }\KeywordTok{sum}\NormalTok{(raz)}\OperatorTok{/}\NormalTok{(}\DecValTok{4}\OperatorTok{*}\NormalTok{deltahatdot}\OperatorTok{^}\DecValTok{2}\NormalTok{)}
\NormalTok{ahat2_}\DecValTok{1}
\end{Highlighting}
\end{Shaded}

\begin{verbatim}
## [1] 1.249524
\end{verbatim}

\[
1+\hat{a}^{2}=8903^{2}\sum\limits_{j=1}^{5}\sum\limits_{k=1}^{5}\left( \hat{V}_{p}^{2}\left( \hat{p}_{j},\hat{p}_{k}\right) /p_{0j}p_{0k}\right) /\left(
4\times 2,457^{2}\right) =1,253\;\;. 
\]

Podemos então calcular a estatística \(X_{P}^{2}\) de Pearson usando
\eqref{eq:qual11}, resultando em

\begin{Shaded}
\begin{Highlighting}[]
\NormalTok{x2p <-nppv}\OperatorTok{*}\StringTok{ }\KeywordTok{sum}\NormalTok{((phat }\OperatorTok{-}\NormalTok{p0)}\OperatorTok{^}\DecValTok{2}\OperatorTok{/}\NormalTok{p0)}
\NormalTok{x2p}
\end{Highlighting}
\end{Shaded}

\begin{verbatim}
## [1] 11.50719
\end{verbatim}

\[
X_{P}^{2}=11,64 
\] com \(4\) g.l. e um \(p\)valor \(0,020\) .

A estatística de Pearson com ajustamento de EPA médio é calculada usando
\eqref{eq:qual12}, resultando em

\begin{Shaded}
\begin{Highlighting}[]
\NormalTok{x2p}\OperatorTok{/}\NormalTok{dhatdot}
\end{Highlighting}
\end{Shaded}

\begin{verbatim}
## [1] 4.842751
\end{verbatim}

\[
X_{P}^{2}\left( \hat{d}_{.}\right) =11,64/2,376=4,901 
\] com \(4\) g.l. e um \(p\)valor \(0,298\) .

A estatística de Pearson com ajustamento de Rao-Scott de primeira ordem,
dada por \eqref{eq:qual13}, resulta em

\begin{Shaded}
\begin{Highlighting}[]
\NormalTok{x2p}\OperatorTok{/}\NormalTok{deltahatdot }
\end{Highlighting}
\end{Shaded}

\begin{verbatim}
## [1] 4.678467
\end{verbatim}

\[
X_{P}^{2}\left( \hat{\delta}_{.}\right) =11,64/2,457=4,74 
\] com \(4\) g.l. e um \(p\)valor \(0,315\) .

O ajustamento de Rao-Scott de primeira ordem F-corrigido para a
estatística de Pearson, dado por \eqref{eq:qual15}, resulta em

\begin{Shaded}
\begin{Highlighting}[]
\NormalTok{J <-}\StringTok{ }\KeywordTok{length}\NormalTok{(phat)}
\NormalTok{x2p}\OperatorTok{/}\StringTok{ }\NormalTok{((J}\OperatorTok{-}\DecValTok{1}\NormalTok{)}\OperatorTok{*}\NormalTok{deltahatdot)}
\end{Highlighting}
\end{Shaded}

\begin{verbatim}
## [1] 1.169617
\end{verbatim}

\[
FX_{P}^{2}\left( \hat{\delta}_{.}\right) =4,74/4=1,85
\] com \(4\) e \(261\) g.l e um \(p\)valor \(0,318\) .

O ajustamento de Rao-Scott de segunda ordem para a estatística de
Pearson, dado por \eqref{eq:qual14}, resulta em

\begin{Shaded}
\begin{Highlighting}[]
\NormalTok{(x2p}\OperatorTok{/}\NormalTok{deltahatdot)}\OperatorTok{/}\NormalTok{ahat2_}\DecValTok{1}
\end{Highlighting}
\end{Shaded}

\begin{verbatim}
## [1] 3.744199
\end{verbatim}

\[
X_{P}^{2}\left( \hat{\delta}_{.},\hat{a}^{2}\right) =4,74/1,253=3,784 
\] com \(4\)/\(1,253=3,19\) g.l. e \(p\)valor \(0,314\) .

A estatística de Wald baseada no plano amostral (veja equação
\eqref{eq:qual8} resulta em

\begin{Shaded}
\begin{Highlighting}[]
\NormalTok{phatv <-}\StringTok{ }\KeywordTok{matrix}\NormalTok{(phat[}\OperatorTok{-}\DecValTok{5}\NormalTok{], }\DataTypeTok{ncol=}\DecValTok{1}\NormalTok{)}
\NormalTok{p0v <-}\StringTok{ }\KeywordTok{matrix}\NormalTok{(p0[}\OperatorTok{-}\DecValTok{5}\NormalTok{], }\DataTypeTok{ncol=}\DecValTok{1}\NormalTok{)}
\NormalTok{vhat0 <-}\StringTok{ }\NormalTok{vhat[}\OperatorTok{-}\DecValTok{5}\NormalTok{,}\OperatorTok{-}\DecValTok{5}\NormalTok{]}
\NormalTok{x2w <-}\StringTok{  }\KeywordTok{t}\NormalTok{(phatv}\OperatorTok{-}\NormalTok{p0v)}\OperatorTok\StringTok{ }\KeywordTok{solve}\NormalTok{(vhat0)}\OperatorTok\NormalTok{(phatv}\OperatorTok{-}\NormalTok{p0v)}
\NormalTok{x2w}
\end{Highlighting}
\end{Shaded}

\begin{verbatim}
##          [,1]
## [1,] 5.742022
\end{verbatim}

\[
X_{W}^{2}=5,691 
\] com \(4\) g.l. e um \(p\)valor \(0,223\) .

As estatísticas F-corrigidas de Wald, definidas em \eqref{eq:qual9} e
\eqref{eq:qual10}, resultam em

\begin{Shaded}
\begin{Highlighting}[]
\CommentTok{# número de estratos da PPV-sudeste}
\NormalTok{nestrat <-}\StringTok{ }\KeywordTok{length}\NormalTok{(}\KeywordTok{unique}\NormalTok{(ppv1}\OperatorTok{$}\NormalTok{estratof[ppv1}\OperatorTok{$}\NormalTok{regiao }\OperatorTok{==}\StringTok{ }\DecValTok{2}\NormalTok{]))}
\CommentTok{# número de setores da PPV-sudeste}
\NormalTok{nsetor <-}\StringTok{ }\KeywordTok{length}\NormalTok{(}\KeywordTok{unique}\NormalTok{(ppv1}\OperatorTok{$}\NormalTok{nsetor[ppv1}\OperatorTok{$}\NormalTok{regiao }\OperatorTok{==}\StringTok{ }\DecValTok{2}\NormalTok{]))}
\NormalTok{f <-}\StringTok{ }\NormalTok{nsetor }\OperatorTok{-}\NormalTok{nestrat}
\NormalTok{f1p <-}\StringTok{ }\NormalTok{((f}\OperatorTok{-}\NormalTok{J}\OperatorTok{+}\DecValTok{2}\NormalTok{)}\OperatorTok{/}\NormalTok{(f}\OperatorTok{*}\NormalTok{(J}\OperatorTok{-}\DecValTok{1}\NormalTok{)))}\OperatorTok{*}\NormalTok{x2w}
\NormalTok{f1p}
\end{Highlighting}
\end{Shaded}

\begin{verbatim}
##          [,1]
## [1,] 1.419005
\end{verbatim}

\[
F_{1.p}=\frac{261-5+2}{261\times 4}\times 5,690661=1,406 
\] com \(4\) e \(259\) g.l. e um \(p\)valor \(0,232\) , e

\begin{Shaded}
\begin{Highlighting}[]
\NormalTok{f2p <-}\StringTok{ }\NormalTok{x2w}\OperatorTok{/}\NormalTok{(J}\OperatorTok{-}\DecValTok{1}\NormalTok{)}
\NormalTok{f2p}
\end{Highlighting}
\end{Shaded}

\begin{verbatim}
##          [,1]
## [1,] 1.435505
\end{verbatim}

\[
F_{2.p}=5,691/4\mbox{ }=1,423 
\] com 4 e 261 gl e um \(p\)valor \(0,228\) .

A Tabela \ref{tab:estaltteste} resume os valores das diversas
estatísticas de teste calculadas, bem como das informações comparativas
com as respectivas distribuições de referência.

\begin{longtable}[]{@{}cclcr@{}}
\caption{\label{tab:estaltteste}Valores e valores-p de estatísticas
alternativas de teste}\tabularnewline
\toprule
\begin{minipage}[b]{0.40\columnwidth}\centering\strut
Estatística\strut
\end{minipage} & \begin{minipage}[b]{0.15\columnwidth}\centering\strut
Tipo\strut
\end{minipage} & \begin{minipage}[b]{0.05\columnwidth}\raggedright\strut
Valor\strut
\end{minipage} & \begin{minipage}[b]{0.19\columnwidth}\centering\strut
Distribuição\strut
\end{minipage} & \begin{minipage}[b]{0.07\columnwidth}\raggedleft\strut
valor-\(p\)\strut
\end{minipage}\tabularnewline
\midrule
\endfirsthead
\toprule
\begin{minipage}[b]{0.40\columnwidth}\centering\strut
Estatística\strut
\end{minipage} & \begin{minipage}[b]{0.15\columnwidth}\centering\strut
Tipo\strut
\end{minipage} & \begin{minipage}[b]{0.05\columnwidth}\raggedright\strut
Valor\strut
\end{minipage} & \begin{minipage}[b]{0.19\columnwidth}\centering\strut
Distribuição\strut
\end{minipage} & \begin{minipage}[b]{0.07\columnwidth}\raggedleft\strut
valor-\(p\)\strut
\end{minipage}\tabularnewline
\midrule
\endhead
\begin{minipage}[t]{0.40\columnwidth}\centering\strut
\(X_{P}^{2}\)\strut
\end{minipage} & \begin{minipage}[t]{0.15\columnwidth}\centering\strut
Adequada para IID\strut
\end{minipage} & \begin{minipage}[t]{0.05\columnwidth}\raggedright\strut
11.640\strut
\end{minipage} & \begin{minipage}[t]{0.19\columnwidth}\centering\strut
\(\chi ^{2}(4)\)\strut
\end{minipage} & \begin{minipage}[t]{0.07\columnwidth}\raggedleft\strut
0.020\strut
\end{minipage}\tabularnewline
\begin{minipage}[t]{0.40\columnwidth}\centering\strut
\(X_{P}^{2}\left( \hat{d}_{.}\right)\)\strut
\end{minipage} & \begin{minipage}[t]{0.15\columnwidth}\centering\strut
Ajustes e\strut
\end{minipage} & \begin{minipage}[t]{0.05\columnwidth}\raggedright\strut
4.901\strut
\end{minipage} & \begin{minipage}[t]{0.19\columnwidth}\centering\strut
\(\chi ^{2}(4)\)\strut
\end{minipage} & \begin{minipage}[t]{0.07\columnwidth}\raggedleft\strut
0.298\strut
\end{minipage}\tabularnewline
\begin{minipage}[t]{0.40\columnwidth}\centering\strut
\(X_{P}^{2}\left( \hat{\delta}_{.}\right)\)\strut
\end{minipage} & \begin{minipage}[t]{0.15\columnwidth}\centering\strut
correções da\strut
\end{minipage} & \begin{minipage}[t]{0.05\columnwidth}\raggedright\strut
4.740\strut
\end{minipage} & \begin{minipage}[t]{0.19\columnwidth}\centering\strut
\(\chi ^{2}(4)\)\strut
\end{minipage} & \begin{minipage}[t]{0.07\columnwidth}\raggedleft\strut
0.315\strut
\end{minipage}\tabularnewline
\begin{minipage}[t]{0.40\columnwidth}\centering\strut
\(FX_{P}^{2}\left( \hat{\delta}_{.}\right)\)\strut
\end{minipage} & \begin{minipage}[t]{0.15\columnwidth}\centering\strut
Estatística\strut
\end{minipage} & \begin{minipage}[t]{0.05\columnwidth}\raggedright\strut
1.850\strut
\end{minipage} & \begin{minipage}[t]{0.19\columnwidth}\centering\strut
\(F\left( 4;261\right)\)\strut
\end{minipage} & \begin{minipage}[t]{0.07\columnwidth}\raggedleft\strut
0.318\strut
\end{minipage}\tabularnewline
\begin{minipage}[t]{0.40\columnwidth}\centering\strut
\(X_{P}^{2}\left( \hat{\delta}_{.},\hat{a}^{2}\right)\)\strut
\end{minipage} & \begin{minipage}[t]{0.15\columnwidth}\centering\strut
\(X_{P}^{2}\)\strut
\end{minipage} & \begin{minipage}[t]{0.05\columnwidth}\raggedright\strut
3.784\strut
\end{minipage} & \begin{minipage}[t]{0.19\columnwidth}\centering\strut
\(\chi ^{2}(3,19)\)\strut
\end{minipage} & \begin{minipage}[t]{0.07\columnwidth}\raggedleft\strut
0.314\strut
\end{minipage}\tabularnewline
\begin{minipage}[t]{0.40\columnwidth}\centering\strut
\(X_{W}^{2}\)\strut
\end{minipage} & \begin{minipage}[t]{0.15\columnwidth}\centering\strut
Baseadas no\strut
\end{minipage} & \begin{minipage}[t]{0.05\columnwidth}\raggedright\strut
5.691\strut
\end{minipage} & \begin{minipage}[t]{0.19\columnwidth}\centering\strut
\(\chi ^{2}(4)\)\strut
\end{minipage} & \begin{minipage}[t]{0.07\columnwidth}\raggedleft\strut
0.223\strut
\end{minipage}\tabularnewline
\begin{minipage}[t]{0.40\columnwidth}\centering\strut
\(F_{1.p}\)\strut
\end{minipage} & \begin{minipage}[t]{0.15\columnwidth}\centering\strut
plano\strut
\end{minipage} & \begin{minipage}[t]{0.05\columnwidth}\raggedright\strut
1.406\strut
\end{minipage} & \begin{minipage}[t]{0.19\columnwidth}\centering\strut
\(F\left( 4;259\right)\)\strut
\end{minipage} & \begin{minipage}[t]{0.07\columnwidth}\raggedleft\strut
0.232\strut
\end{minipage}\tabularnewline
\begin{minipage}[t]{0.40\columnwidth}\centering\strut
\(F_{2.p}\)\strut
\end{minipage} & \begin{minipage}[t]{0.15\columnwidth}\centering\strut
amostral\strut
\end{minipage} & \begin{minipage}[t]{0.05\columnwidth}\raggedright\strut
1.423\strut
\end{minipage} & \begin{minipage}[t]{0.19\columnwidth}\centering\strut
\(F\left( 4;261\right)\)\strut
\end{minipage} & \begin{minipage}[t]{0.07\columnwidth}\raggedleft\strut
0.228\strut
\end{minipage}\tabularnewline
\bottomrule
\end{longtable}

Examinando os resultados da Tabela \ref{tab:estaltteste}, verificamos
que o teste clássico de Pearson rejeita a hipótese nula \(H_{0}\) no
nível \(\alpha =5\%\), diferentemente de todos os outros testes. Os
valores das estatísticas com ajustes de Rao-Scott (com ou sem correção
F) são semelhantes e parecem corrigir exageradamente o \(p\)-valor dos
testes. A estatística de Wald baseada no plano amostral e suas correções
F, que têm valores quase iguais, produzem uma correção menor no
\(p\)-valor do teste. Nesse exemplo, como o número de graus de liberdade
(dado pelo número de unidades primárias na amostra menos o número de
estratos)\(f=m-H=261\) é grande, a correção F tem pouco efeito, tanto
nas estatísticas com ajustes de primeira e segunda ordem de Rao-Scott,
como na estatística Wald.

\section{Laboratório de R}\label{laboratorio-de-r-4}

Exemplo 7.1 pode ser substituído por: Criar variável ITAB (Não aparece)

\begin{Shaded}
\begin{Highlighting}[]
\NormalTok{ppv.des<-}\KeywordTok{svydesign}\NormalTok{(}\DataTypeTok{id=}\OperatorTok{~}\NormalTok{nsetor,}\DataTypeTok{strat=}\OperatorTok{~}\NormalTok{estratof,}\DataTypeTok{weights=}\OperatorTok{~}\NormalTok{pesof,}
\DataTypeTok{data=}\NormalTok{ppv1,}\DataTypeTok{nest=}\OtherTok{TRUE}\NormalTok{)}
\NormalTok{ppv.se.des<-}\KeywordTok{subset}\NormalTok{(ppv.des,regiao}\OperatorTok{==}\DecValTok{2}\NormalTok{)}
\end{Highlighting}
\end{Shaded}

\begin{Shaded}
\begin{Highlighting}[]
\NormalTok{ppv.id<-}\KeywordTok{svymean}\NormalTok{(}\OperatorTok{~}\NormalTok{idatab,ppv.se.des,}\DataTypeTok{deff=}\NormalTok{T)}
\NormalTok{vhvat<-}\KeywordTok{vcov}\NormalTok{(ppv.id)}
\end{Highlighting}
\end{Shaded}

\begin{Shaded}
\begin{Highlighting}[]
\KeywordTok{library}\NormalTok{(xtable)}
\NormalTok{fr_ppv_id<-}\StringTok{ }\KeywordTok{data.frame}\NormalTok{(ppv.id)}
\KeywordTok{row.names}\NormalTok{(fr_ppv_id) <-}\StringTok{ }\OtherTok{NULL}
\NormalTok{fr_ppv_id <-}\StringTok{ }\KeywordTok{cbind}\NormalTok{(}\DataTypeTok{idade=} \KeywordTok{c}\NormalTok{(}\StringTok{"0 a 14 anos"}\NormalTok{,}\StringTok{"15 a 29 anos"}\NormalTok{, }\StringTok{"30 a 44 anos"}\NormalTok{,}\StringTok{"45 a 59 anos"}\NormalTok{, }\StringTok{"60 anos e mais"}\NormalTok{), fr_ppv_id)}
\NormalTok{knitr}\OperatorTok{::}\KeywordTok{kable}\NormalTok{(fr_ppv_id, }\DataTypeTok{booktabs=} \OtherTok{TRUE}\NormalTok{, }\DataTypeTok{digits=}\KeywordTok{c}\NormalTok{(}\DecValTok{0}\NormalTok{,}\DecValTok{4}\NormalTok{,}\DecValTok{4}\NormalTok{,}\DecValTok{3}\NormalTok{), }\DataTypeTok{caption=}\StringTok{"Estimativas das proporções nas classes"}\NormalTok{) }
\end{Highlighting}
\end{Shaded}

\begin{table}

\caption{\label{tab:estpropclas}Estimativas das proporções nas classes}
\centering
\begin{tabular}[t]{lrrr}
\toprule
idade & mean & SE & deff\\
\midrule
0 a 14 anos & 0.2845 & 0.0072 & 2.286\\
15 a 29 anos & 0.2678 & 0.0069 & 2.187\\
30 a 44 anos & 0.2225 & 0.0066 & 2.254\\
45 a 59 anos & 0.1316 & 0.0051 & 1.991\\
60 anos e mais & 0.0935 & 0.0055 & 3.163\\
\bottomrule
\end{tabular}
\end{table}

Estatística de Wald calculada a partir da fórmula \eqref{eq:qual8}

\begin{Shaded}
\begin{Highlighting}[]
\CommentTok{#Vetor de proporções estimadas}
\NormalTok{phat<-}\KeywordTok{coefficients}\NormalTok{(ppv.id)}
\CommentTok{# Vetor de proporções obtido na Contagem Populacional de 1996}
\NormalTok{p0<-}\KeywordTok{c}\NormalTok{(.}\DecValTok{2842}\NormalTok{,.}\DecValTok{2774}\NormalTok{,.}\DecValTok{2263}\NormalTok{,.}\DecValTok{1261}\NormalTok{,.}\DecValTok{086}\NormalTok{)}
\CommentTok{# Estatística de Wald}
\NormalTok{x2_w<-}\KeywordTok{matrix}\NormalTok{((phat}\OperatorTok{-}\NormalTok{p0)[}\OperatorTok{-}\DecValTok{5}\NormalTok{],}\DataTypeTok{nrow=}\DecValTok{1}\NormalTok{)}\OperatorTok\KeywordTok{solve}\NormalTok{(vhat[}\OperatorTok{-}\DecValTok{5}\NormalTok{,}\OperatorTok{-}\DecValTok{5}\NormalTok{])}\OperatorTok
\KeywordTok{matrix}\NormalTok{((phat}\OperatorTok{-}\NormalTok{p0)[}\OperatorTok{-}\DecValTok{5}\NormalTok{],}\DataTypeTok{ncol=}\DecValTok{1}\NormalTok{)}
\NormalTok{x2_w}
\end{Highlighting}
\end{Shaded}

\begin{verbatim}
##          [,1]
## [1,] 5.742022
\end{verbatim}

\begin{Shaded}
\begin{Highlighting}[]
\CommentTok{#Cálculo do p-valor}
\KeywordTok{round}\NormalTok{(}\KeywordTok{pchisq}\NormalTok{(x2_w,}\DecValTok{4}\NormalTok{,}\DataTypeTok{lower.tail=}\OtherTok{FALSE}\NormalTok{),}\DataTypeTok{digits=}\DecValTok{3}\NormalTok{)}
\end{Highlighting}
\end{Shaded}

\begin{verbatim}
##       [,1]
## [1,] 0.219
\end{verbatim}

Estatística de Pearson calculada a partir da fórmula 7.11

\begin{Shaded}
\begin{Highlighting}[]
\NormalTok{n<-}\DecValTok{8903}
\NormalTok{P0<-}\KeywordTok{diag}\NormalTok{(p0)}\OperatorTok{-}\KeywordTok{matrix}\NormalTok{(p0,}\DataTypeTok{ncol=}\DecValTok{1}\NormalTok{)}\OperatorTok\KeywordTok{matrix}\NormalTok{(p0,}\DataTypeTok{nrow=}\DecValTok{1}\NormalTok{)}
\NormalTok{x2_p<-n}\OperatorTok{*}\KeywordTok{matrix}\NormalTok{((phat}\OperatorTok{-}\NormalTok{p0)[}\OperatorTok{-}\DecValTok{5}\NormalTok{],}\DataTypeTok{nrow=}\DecValTok{1}\NormalTok{)}\OperatorTok\KeywordTok{solve}\NormalTok{(P0[}\OperatorTok{-}\DecValTok{5}\NormalTok{,}\OperatorTok{-}\DecValTok{5}\NormalTok{])}\OperatorTok
\KeywordTok{matrix}\NormalTok{((phat}\OperatorTok{-}\NormalTok{p0)[}\OperatorTok{-}\DecValTok{5}\NormalTok{],}\DataTypeTok{ncol=}\DecValTok{1}\NormalTok{)}
\NormalTok{x2_p}
\end{Highlighting}
\end{Shaded}

\begin{verbatim}
##          [,1]
## [1,] 11.50719
\end{verbatim}

Cálculo do valor-p:

\begin{Shaded}
\begin{Highlighting}[]
\KeywordTok{round}\NormalTok{(}\KeywordTok{pchisq}\NormalTok{(x2_p,}\DecValTok{4}\NormalTok{,}\DataTypeTok{lower.tail=}\OtherTok{FALSE}\NormalTok{),}\DataTypeTok{digits=}\DecValTok{3}\NormalTok{)}
\end{Highlighting}
\end{Shaded}

\begin{verbatim}
##       [,1]
## [1,] 0.021
\end{verbatim}

\chapter{Testes em Tabelas de Duas Entradas}\label{testetab2}

\section{Introdução}\label{introducao-2}

\section{Tabelas 2x2}\label{tabelas22}

\subsection{Teste de Independência}\label{teste-de-independencia}

\subsection{Teste de Homogeneidade}\label{teste-de-homogeneidade}

\subsection{Efeitos de Plano Amostral nas
Celas}\label{efeitos-de-plano-amostral-nas-celas}

\section{Tabelas de Duas Entradas (Caso
Geral)}\label{tabelas-de-duas-entradas-caso-geral}

\subsection{Teste de Homogeneidade}\label{teste-de-homogeneidade-1}

\subsection{Teste de Independência}\label{teste-de-independencia-1}

\subsection{Estatística de Wald Baseada no Plano
Amostral}\label{estatistica-de-wald-baseada-no-plano-amostral-1}

\subsection{Estatística de Pearson com Ajuste de
Rao-Scott}\label{estatistica-de-pearson-com-ajuste-de-rao-scott}

\section{Laboratório de R}\label{laboratorio-de-r-5}

\section{\texorpdfstring{{[}1{]} ``stra'' ``psu'' ``pesopes''
``informal'' ``sx''
``id''}{{[}1{]} stra psu pesopes informal sx id}}\label{stra-psu-pesopes-informal-sx-id-1}

\section{\texorpdfstring{{[}7{]} ``ae'' ``ht'' ``re''
``um''}{{[}7{]} ae ht re um}}\label{ae-ht-re-um-1}

\section{stra psu pesopes informal sx id
ae}\label{stra-psu-pesopes-informal-sx-id-ae-1}

\section{\texorpdfstring{``numeric'' ``numeric'' ``numeric'' ``numeric''
``numeric'' ``numeric''
``numeric''}{numeric numeric numeric numeric numeric numeric numeric}}\label{numeric-numeric-numeric-numeric-numeric-numeric-numeric-1}

\section{ht re um}\label{ht-re-um-1}

\section{\texorpdfstring{``numeric'' ``numeric''
``numeric''}{numeric numeric numeric}}\label{numeric-numeric-numeric-1}

\section{mean SE}\label{mean-se}

\section{sx1 0.65708 0.0056}\label{sx1-0.65708-0.0056}

\section{sx2 0.34292 0.0056}\label{sx2-0.34292-0.0056}

\section{mean SE}\label{mean-se-1}

\section{re1 0.15546 0.0069}\label{re1-0.15546-0.0069}

\section{re2 0.58356 0.0082}\label{re2-0.58356-0.0082}

\section{re3 0.26098 0.0096}\label{re3-0.26098-0.0096}

\section{mean SE}\label{mean-se-2}

\section{ae1 0.31304 0.0095}\label{ae1-0.31304-0.0095}

\section{ae2 0.31972 0.0071}\label{ae2-0.31972-0.0071}

\section{ae3 0.36725 0.0105}\label{ae3-0.36725-0.0105}

\section{ht1 ht2 ht3}\label{ht1-ht2-ht3}

\section{0.2103714 0.6148881 0.1747405}\label{section-23}

\section{\$names}\label{names}

\section{\texorpdfstring{{[}1{]} ``ht1'' ``ht2''
``ht3''}{{[}1{]} ht1 ht2 ht3}}\label{ht1-ht2-ht3-1}

\section{}\label{section-24}

\section{\$var}\label{var}

\section{ht1 ht2 ht3}\label{ht1-ht2-ht3-2}

\section{ht1 3.666206e-05 -3.322546e-05
-3.436592e-06}\label{ht1-3.666206e-05--3.322546e-05--3.436592e-06}

\section{ht2 -3.322546e-05 6.758652e-05
-3.436106e-05}\label{ht2--3.322546e-05-6.758652e-05--3.436106e-05}

\section{ht3 -3.436592e-06 -3.436106e-05
3.779765e-05}\label{ht3--3.436592e-06--3.436106e-05-3.779765e-05}

\section{}\label{section-25}

\section{\$statistic}\label{statistic}

\section{\texorpdfstring{{[}1{]} ``mean''}{{[}1{]} mean}}\label{mean}

\section{}\label{section-26}

\section{\$class}\label{class}

\section{\texorpdfstring{{[}1{]}
``svystat''}{{[}1{]} svystat}}\label{svystat}

\section{ht1 ht2 ht3}\label{ht1-ht2-ht3-3}

\section{ht1 3.666206e-05 -3.322546e-05
-3.436592e-06}\label{ht1-3.666206e-05--3.322546e-05--3.436592e-06-1}

\section{ht2 -3.322546e-05 6.758652e-05
-3.436106e-05}\label{ht2--3.322546e-05-6.758652e-05--3.436106e-05-1}

\section{ht3 -3.436592e-06 -3.436106e-05
3.779765e-05}\label{ht3--3.436592e-06--3.436106e-05-3.779765e-05-1}

\section{ht1 ht2 ht3}\label{ht1-ht2-ht3-4}

\section{ht1 3.666206e-05 -3.322546e-05
-3.436592e-06}\label{ht1-3.666206e-05--3.322546e-05--3.436592e-06-2}

\section{ht2 -3.322546e-05 6.758652e-05
-3.436106e-05}\label{ht2--3.322546e-05-6.758652e-05--3.436106e-05-2}

\section{ht3 -3.436592e-06 -3.436106e-05
3.779765e-05}\label{ht3--3.436592e-06--3.436106e-05-3.779765e-05-2}

\section{mean SE DEff}\label{mean-se-deff}

\section{re1 0.1554555 0.0068977
2.3611}\label{re1-0.1554555-0.0068977-2.3611}

\section{re2 0.5835630 0.0082001
1.8027}\label{re2-0.5835630-0.0082001-1.8027}

\section{re3 0.2609815 0.0096130
3.1216}\label{re3-0.2609815-0.0096130-3.1216}

\section{re}\label{re}

\section{sx 1 2 3}\label{sx-1-2-3}

\section{1 0.073 0.388 0.196}\label{section-27}

\section{2 0.082 0.195 0.065}\label{section-28}

\section{sx re1 re2 re3 se.re1 se.re2
se.re3}\label{sx-re1-re2-re3-se.re1-se.re2-se.re3}

\section{1 1 0.1110831 0.5908215 0.2980955 0.005726888 0.01025759
0.01112131}\label{section-29}

\section{2 2 0.2404788 0.5696548 0.1898663 0.012502636 0.01193753
0.01114102}\label{section-30}

\section{mean SE DEff}\label{mean-se-deff-1}

\section{I((sx == 1 \& re == 1) * 1) 0.0729904 0.0038343
1.4156}\label{isx-1-re-1-1-0.0729904-0.0038343-1.4156}

\section{re}\label{re-1}

\section{sx 1 2 3}\label{sx-1-2-3-1}

\section{1 0.07299044 0.38821684 0.19587250}\label{section-31}

\section{2 0.08246505 0.19534616 0.06510900}\label{section-32}

\section{re}\label{re-2}

\section{sx 1 2 3}\label{sx-1-2-3-2}

\section{1 7.299044 38.821684 19.587250}\label{section-33}

\section{2 8.246505 19.534616 6.510900}\label{section-34}

\section{{[}1{]} 1.642161}\label{section-35}

\section{}\label{section-36}

\section{Pearson's Chi-squared test}\label{pearsons-chi-squared-test}

\section{}\label{section-37}

\section{data: tab.amo}\label{data-tab.amo}

\section{X-squared = 227.03, df = 2, p-value \textless{}
2.2e-16}\label{x-squared-227.03-df-2-p-value-2.2e-16}

\chapter{Estimação de densidades}\label{estimacao-de-densidades}

\section{Introdução}\label{introducao-3}

\chapter{Modelos Hierárquicos}\label{modelos-hierarquicos}

\section{Introdução}\label{introducao-4}

\chapter{Não-Resposta}\label{nao-resposta}

\section{Introdução}\label{introducao-5}

\chapter{Diagnóstico de ajuste de
modelo}\label{diagnostico-de-ajuste-de-modelo}

\section{Introdução}\label{introducao-6}

\chapter{Agregação vs.~Desagregação}\label{agregdesag}

\section{Introdução}\label{introducao-7}

\section{Modelagem da Estrutura
Populacional}\label{modelagem-da-estrutura-populacional}

\section{Modelos Hierárquicos}\label{modelos-hierarquicos-1}

\section{Análise Desagregada: Prós e
Contras}\label{analise-desagregada-pros-e-contras}

\chapter{Pacotes para Analisar Dados Amostrais}\label{pacotes}

\section{Introdução}\label{introducao-8}

\section{Pacotes Computacionais}\label{pacotes-computacionais}

\chapter*{Referências}\label{referencias}
\addcontentsline{toc}{chapter}{Referências}

\bibliography{packages,book}


\end{document}
