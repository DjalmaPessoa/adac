\documentclass[]{book}
\usepackage{lmodern}
\usepackage{amssymb,amsmath}
\usepackage{ifxetex,ifluatex}
\usepackage{fixltx2e} % provides \textsubscript
\ifnum 0\ifxetex 1\fi\ifluatex 1\fi=0 % if pdftex
  \usepackage[T1]{fontenc}
  \usepackage[utf8]{inputenc}
\else % if luatex or xelatex
  \ifxetex
    \usepackage{mathspec}
  \else
    \usepackage{fontspec}
  \fi
  \defaultfontfeatures{Ligatures=TeX,Scale=MatchLowercase}
\fi
% use upquote if available, for straight quotes in verbatim environments
\IfFileExists{upquote.sty}{\usepackage{upquote}}{}
% use microtype if available
\IfFileExists{microtype.sty}{%
\usepackage{microtype}
\UseMicrotypeSet[protrusion]{basicmath} % disable protrusion for tt fonts
}{}
\usepackage[margin=1in]{geometry}
\usepackage{hyperref}
\hypersetup{unicode=true,
            pdftitle={Análise de Dados Amostrais Complexos},
            pdfauthor={Djalma Pessoa e Pedro Nascimento Silva},
            pdfborder={0 0 0},
            breaklinks=true}
\urlstyle{same}  % don't use monospace font for urls
\usepackage{natbib}
\bibliographystyle{apalike}
\usepackage{color}
\usepackage{fancyvrb}
\newcommand{\VerbBar}{|}
\newcommand{\VERB}{\Verb[commandchars=\\\{\}]}
\DefineVerbatimEnvironment{Highlighting}{Verbatim}{commandchars=\\\{\}}
% Add ',fontsize=\small' for more characters per line
\usepackage{framed}
\definecolor{shadecolor}{RGB}{248,248,248}
\newenvironment{Shaded}{\begin{snugshade}}{\end{snugshade}}
\newcommand{\KeywordTok}[1]{\textcolor[rgb]{0.13,0.29,0.53}{\textbf{#1}}}
\newcommand{\DataTypeTok}[1]{\textcolor[rgb]{0.13,0.29,0.53}{#1}}
\newcommand{\DecValTok}[1]{\textcolor[rgb]{0.00,0.00,0.81}{#1}}
\newcommand{\BaseNTok}[1]{\textcolor[rgb]{0.00,0.00,0.81}{#1}}
\newcommand{\FloatTok}[1]{\textcolor[rgb]{0.00,0.00,0.81}{#1}}
\newcommand{\ConstantTok}[1]{\textcolor[rgb]{0.00,0.00,0.00}{#1}}
\newcommand{\CharTok}[1]{\textcolor[rgb]{0.31,0.60,0.02}{#1}}
\newcommand{\SpecialCharTok}[1]{\textcolor[rgb]{0.00,0.00,0.00}{#1}}
\newcommand{\StringTok}[1]{\textcolor[rgb]{0.31,0.60,0.02}{#1}}
\newcommand{\VerbatimStringTok}[1]{\textcolor[rgb]{0.31,0.60,0.02}{#1}}
\newcommand{\SpecialStringTok}[1]{\textcolor[rgb]{0.31,0.60,0.02}{#1}}
\newcommand{\ImportTok}[1]{#1}
\newcommand{\CommentTok}[1]{\textcolor[rgb]{0.56,0.35,0.01}{\textit{#1}}}
\newcommand{\DocumentationTok}[1]{\textcolor[rgb]{0.56,0.35,0.01}{\textbf{\textit{#1}}}}
\newcommand{\AnnotationTok}[1]{\textcolor[rgb]{0.56,0.35,0.01}{\textbf{\textit{#1}}}}
\newcommand{\CommentVarTok}[1]{\textcolor[rgb]{0.56,0.35,0.01}{\textbf{\textit{#1}}}}
\newcommand{\OtherTok}[1]{\textcolor[rgb]{0.56,0.35,0.01}{#1}}
\newcommand{\FunctionTok}[1]{\textcolor[rgb]{0.00,0.00,0.00}{#1}}
\newcommand{\VariableTok}[1]{\textcolor[rgb]{0.00,0.00,0.00}{#1}}
\newcommand{\ControlFlowTok}[1]{\textcolor[rgb]{0.13,0.29,0.53}{\textbf{#1}}}
\newcommand{\OperatorTok}[1]{\textcolor[rgb]{0.81,0.36,0.00}{\textbf{#1}}}
\newcommand{\BuiltInTok}[1]{#1}
\newcommand{\ExtensionTok}[1]{#1}
\newcommand{\PreprocessorTok}[1]{\textcolor[rgb]{0.56,0.35,0.01}{\textit{#1}}}
\newcommand{\AttributeTok}[1]{\textcolor[rgb]{0.77,0.63,0.00}{#1}}
\newcommand{\RegionMarkerTok}[1]{#1}
\newcommand{\InformationTok}[1]{\textcolor[rgb]{0.56,0.35,0.01}{\textbf{\textit{#1}}}}
\newcommand{\WarningTok}[1]{\textcolor[rgb]{0.56,0.35,0.01}{\textbf{\textit{#1}}}}
\newcommand{\AlertTok}[1]{\textcolor[rgb]{0.94,0.16,0.16}{#1}}
\newcommand{\ErrorTok}[1]{\textcolor[rgb]{0.64,0.00,0.00}{\textbf{#1}}}
\newcommand{\NormalTok}[1]{#1}
\usepackage{longtable,booktabs}
\usepackage{graphicx,grffile}
\makeatletter
\def\maxwidth{\ifdim\Gin@nat@width>\linewidth\linewidth\else\Gin@nat@width\fi}
\def\maxheight{\ifdim\Gin@nat@height>\textheight\textheight\else\Gin@nat@height\fi}
\makeatother
% Scale images if necessary, so that they will not overflow the page
% margins by default, and it is still possible to overwrite the defaults
% using explicit options in \includegraphics[width, height, ...]{}
\setkeys{Gin}{width=\maxwidth,height=\maxheight,keepaspectratio}
\IfFileExists{parskip.sty}{%
\usepackage{parskip}
}{% else
\setlength{\parindent}{0pt}
\setlength{\parskip}{6pt plus 2pt minus 1pt}
}
\setlength{\emergencystretch}{3em}  % prevent overfull lines
\providecommand{\tightlist}{%
  \setlength{\itemsep}{0pt}\setlength{\parskip}{0pt}}
\setcounter{secnumdepth}{5}
% Redefines (sub)paragraphs to behave more like sections
\ifx\paragraph\undefined\else
\let\oldparagraph\paragraph
\renewcommand{\paragraph}[1]{\oldparagraph{#1}\mbox{}}
\fi
\ifx\subparagraph\undefined\else
\let\oldsubparagraph\subparagraph
\renewcommand{\subparagraph}[1]{\oldsubparagraph{#1}\mbox{}}
\fi

%%% Use protect on footnotes to avoid problems with footnotes in titles
\let\rmarkdownfootnote\footnote%
\def\footnote{\protect\rmarkdownfootnote}

%%% Change title format to be more compact
\usepackage{titling}

% Create subtitle command for use in maketitle
\newcommand{\subtitle}[1]{
  \posttitle{
    \begin{center}\large#1\end{center}
    }
}

\setlength{\droptitle}{-2em}
  \title{Análise de Dados Amostrais Complexos}
  \pretitle{\vspace{\droptitle}\centering\huge}
  \posttitle{\par}
  \author{Djalma Pessoa e Pedro Nascimento Silva}
  \preauthor{\centering\large\emph}
  \postauthor{\par}
  \predate{\centering\large\emph}
  \postdate{\par}
  \date{2018-01-09}

\usepackage{booktabs}
\usepackage[brazil]{babel}

\usepackage{amsthm}
\newtheorem{theorem}{Teorema}[chapter]
\newtheorem{lemma}{Lema}[chapter]
\theoremstyle{definition}
\newtheorem{definition}{Definição}[chapter]
\newtheorem{corollary}{Corolário}[chapter]
\newtheorem{proposition}{Proposição}[chapter]
\theoremstyle{definition}
\newtheorem{example}{Exemplo}[chapter]
\theoremstyle{definition}
\newtheorem{exercise}{Exercise}[chapter]
\theoremstyle{remark}
\newtheorem*{remark}{Observação}
\newtheorem*{solution}{Solution}
\let\BeginKnitrBlock\begin \let\EndKnitrBlock\end
\begin{document}
\maketitle

{
\setcounter{tocdepth}{1}
\tableofcontents
}
\chapter*{Prefácio}\label{prefacio}
\addcontentsline{toc}{chapter}{Prefácio}

\section*{Agradecimentos}\label{agradecimentos}
\addcontentsline{toc}{section}{Agradecimentos}

\chapter{Introdução}\label{introduc}

\section{Motivação}\label{motivacao}

Este livro trata de problema de grande importância para os analistas de
dados obtidos através de pesquisas amostrais, tais como as conduzidas
por agências produtoras de informações estatísticas oficiais ou
públicas. Tais dados são comumente utilizados em análises descritivas
envolvendo a obtenção de estimativas para totais, médias, proporções e
razões. Nessas análises, em geral, são devidamente incorporados os pesos
distintos das observações e a estrutura do plano amostral empregado para
obter os dados considerados.

Nas três últimas décadas tem se tornado mais frequente um outro tipo de
uso de dados de pesquisas amostrais. Tal uso, denominado secundário e/ou
analítico, envolve a construção e ajuste de modelos, geralmente feitos
por analistas que trabalham fora das agências produtoras dos dados.
Neste caso, o foco da análise busca estabelecer a natureza de relações
ou associações entre variáveis ou testar hipóteses. Para tais fins, a
estatística clássica conta com um vasto arsenal de ferramentas de
análise, já incorporado aos principais pacotes estatísticos disponíveis
(tais como MINITAB, R, SAS, SPSS, etc).

As ferramentas de análise convencionais disponíveis nesses pacotes
estatísticos geralmente partem de hipóteses básicas que só são válidas
quando os dados foram obtidos através de Amostras Aleatórias Simples Com
Reposição (AASC). Tais hipóteses são geralmente inadequadas para modelar
observações provenientes de amostras de populações finitas, pois
desconsideram os seguintes aspectos relevantes dos planos amostrais
usualmente empregados nas pesquisas amostrais:

i.) \textbf{probabilidades distintas de seleção das unidades};

ii.) \textbf{conglomeração das unidades};

iii.) \textbf{estratificação};

iv.) \textbf{calibração ou imputação para não-resposta e outros
ajustes}.

As estimativas pontuais de parâmetros descritivos da população ou de
modelos são influenciadas por pesos distintos das observações. Além
disso, as estimativas de variância (ou da precisão dos estimadores) são
influenciadas pela conglomeração, estratificação e pesos, ou no caso de
não resposta, também por eventual imputação de dados faltantes ou
reponderação das observações disponíveis. Ao ignorar estes aspectos, os
pacotes tradicionais de análise podem produzir estimativas incorretas
das variâncias das estimativas pontuais.

O exemplo a seguir considera o uso de dados de uma pesquisa amostral
real conduzida pelo IBGE para ilustrar como os pontos i) a iv) acima
mencionados afetam a inferência sobre quantidades descritivas
populacionais tais como totais, médias, proporções e razões.

\BeginKnitrBlock{example}
\protect\hypertarget{exm:distppv}{}{\label{exm:distppv} }Distribuição dos
pesos da amostra da PPV
\EndKnitrBlock{example}

Os dados deste exemplo são relativos à distribuição dos pesos na amostra
da Pesquisa sobre Padrões de Vida (PPV), realizada pelo IBGE nos anos
1996-97. \citep{albieri} descrevem resumidamente a PPV, que foi
realizada nas Regiões Nordeste e Sudeste do País.

O plano amostral empregado na seleção da amostra da PPV foi
estratificado e conglomerado em dois estágios, com alocação igual mas
desproporcional da amostra nos estratos geográficos. A estratificação
considerou inicialmente 10 estratos geográficos conforme listados na
Tabela \ref{tab:numset}.

As Unidades Primárias de Amostragem (UPAs) foram os setores censitários
da Base Operacional Geográfica do IBGE conforme usada para o Censo
Demográfico de 1991. A seleção dos setores dentro de cada estrato foi
feita com probabilidade proporcional ao tamanho. Os domicílios foram as
unidades de segundo estágio, selecionados por amostragem aleatória
simples sem reposição em cada setor selecionado, após a atualização do
cadastro de domicílios do setor.

Em cada um dos 10 estratos geográficos, os setores foram subdivididos em
três estratos de acordo com a renda média mensal do chefe do domicílio
por setor, perfazendo um total de 30 estratos finais para seleção da
amostra.

O tamanho da amostra para cada estrato geográfico foi fixado em 480
domicílios, e o número de setores selecionados foi fixado em 60, com 8
domicílios sendo selecionados em cada setor. A exceção ficou por conta
dos estratos que correspondiam ao restante da área rural de cada Região,
onde foram selecionados 30 setores, com 16 domicílios selecionados por
setor, em função da maior dificuldade de acesso a esses setores, o que
implicaria em aumento de custo da coleta caso fosse mantido o mesmo
tamanho da amostra do segundo estágio em cada setor.

A alocação da amostra entre os estratos de renda dentro de cada estrato
geográfico foi proporcional ao número de domicílios particulares
permanentes ocupados do estrato de renda conforme o Censo de 1991. No
final foram incluídos 554 setores na amostra, distribuídos tal como
mostrado na Tabela \ref{tab:numset}.

\begin{table}

\caption{\label{tab:numset}Número de setores na população e na amostra, por estrato geográfico}
\centering
\begin{tabular}[t]{lrrlrrlrr}
\toprule
Estrato\_Geográfico & População & Amostra\\
\midrule
Região Metropolitana de Fortaleza & 2.263 & 62\\
Região Metropolitana de Recife & 2.309 & 61\\
Região Metropolitana de Salvador & 2.186 & 61\\
Restante Nordeste Urbano & 15.057 & 61\\
Restante Nordeste Rural & 23.711 & 33\\
\addlinespace
Região Metropolitana de Belo Horizonte & 3.283 & 62\\
Região Metropolitana do Rio de Janeiro & 10.420 & 61\\
Região Metropolitana de São Paulo & 14.931 & 61\\
Restante Sudeste Urbano & 25.855 & 61\\
Restante Sudeste Rural & 12.001 & 31\\
Total & 112.016 & 554\\
\bottomrule
\end{tabular}
\end{table}

A Tabela \ref{tab:dispesos} apresenta um resumo das distribuições dos
pesos amostrais das pessoas pesquisadas na PPV para as Regiões Nordeste
(5 estratos geográficos) e Sudeste (5 estratos geográficos)
separadamente, e também para o conjunto da amostra da PPV.

\begin{table}

\caption{\label{tab:dispesos}Resumos da distribuição dos pesos da amostra da PPV}
\centering
\begin{tabular}[t]{lrrrrrlrrrrrlrrrrrlrrrrrlrrrrrlrrrrr}
\toprule
Região & Mínimo & Quartil 1 & Mediana & Quartil 3 & Máximo\\
\midrule
Nordeste & 724 & 1.194 & 1.556 & 6.937 & 15.348\\
Sudeste & 991 & 2.789 & 5.429 & 9.509 & 29.234\\
Nordeste+Sudeste & 724 & 1.403 & 3.785 & 8.306 & 29.234\\
\bottomrule
\end{tabular}
\end{table}

No cálculo dos pesos amostrais foram consideradas as probabilidades de
inclusão dos elementos na amostra, bem como correções para compensar a
não-resposta. Contudo, a grande variabilidade dos pesos amostrais da PPV
é devida, principalmente, à variabilidade das probabilidades de inclusão
na amostra, ilustrando desta forma o ponto i) citado anteriormente nesta
seção. Tal variabilidade foi provocada pela decisão de alocar a amostra
de forma igual entre os estratos geográficos, cujos totais populacionais
são bastante distintos.

Na análise de dados desta pesquisa, deve-se considerar que há elementos
da amostra com pesos muito distintos. Por exemplo, a razão entre o maior
e o menor peso é cerca de 40 vezes. Os pesos também variam bastante
entre as regiões, com mediana 3,5 vezes maior na região Sudeste quando
comparada com a região Nordeste, em função da alocação igual mas
desproporcional da amostra nas regiões.

Tais pesos são utilizados para \emph{expandir} os dados,
multiplicando-se cada observação pelo seu respectivo peso. Assim, por
exemplo, para \texttt{estimar} quantos elementos \texttt{da\ população}
pertencem a determinado conjunto (domínio), basta somar os pesos dos
elementos da amostra que pertencem a este conjunto. É possível ainda
incorporar os pesos, de maneira simples e natural, quando estimamos
medidas descritivas simples da população tais como totais, médias,
proporções, razões, etc.

Por outro lado, quando utilizamos a amostra para estudos analíticos, as
opções padrão disponíveis nos pacotes estatísticos usuais para levar em
conta os pesos distintos das observações são apropriadas somente para
observações independentes e identicamente distribuídas (IID). Por
exemplo, os procedimentos padrão disponíveis para estimar a média
populacional permitem utilizar pesos distintos das observações
amostrais, mas tratariam tais pesos como se fossem frequências de
observações repetidas na amostra, e portanto interpretariam a soma dos
pesos como tamanho amostral, situação que na maioria das vezes gera
inferências incorretas sobre a precisão das estimativas. Isto ocorre
pois o tamanho da amostra é muito menor que a soma dos pesos amostrais
usualmente encontrados nos arquivos de microdados de pesquisas
disseminados por agências de estatísticas oficiais. Em tais pesquisas, a
opção mais freqüente é disseminar pesos que, quando somados, estimam o
total de unidades \texttt{da\ população}.

Além disso, a variabilidade dos pesos para distintas observações
amostrais produz impactos tanto na estimação pontual quanto na estimação
das variâncias dessas estimativas, que sofre ainda influência da
conglomeração e da estratificação - pontos ii) e iii) mencionados
anteriormente.

Para exemplificar o impacto de ignorar os pesos e o plano amostral ao
estimar quantidades descritivas populacionais, tais como totais, médias,
proporções e razões, calculamos estimativas de quantidades desses
diferentes tipos usando a amostra da PPV juntamente com estimativas das
respectivas variâncias. Tais estimativas de variância foram calculadas
sob duas estratégias: a) considerando amostragem aleatória simples
(portanto ignorando o plano amostral efetivamente adotado na pesquisa),
e b) considerando o plano amostral da pesquisa e os pesos diferenciados
das unidades.

A razão entre as estimativas de variância obtidas sob o plano amostral
verdadeiro e sob amostragem aleatória simples foi calculada para cada
uma das estimativas consideradas usando o pacote \texttt{survey} do R
\citep{R-survey}. Essa razão fornece uma medida do efeito de ignorar o
plano amostral. Os resultados das estimativas ponderadas e variâncias
considerando o plano amostral são apresentados na Tabela \ref{tab:epas},
juntamente com as medidas dos efeitos de plano amostral (EPA).

Exemplos de utilização do pacote \texttt{survey} para obtenção de
estimativas apresentadas na \ref{tab:epas} estão na Seção \ref{epa}. As
outras estimativas da Tabela \ref{tab:epas} podem ser obtidas de maneira
análoga.

Na Tabela \ref{tab:epas} apresentamos as estimativas dos seguintes
parâmetros populacionais:

\begin{enumerate}
\def\labelenumi{\arabic{enumi}.}
\tightlist
\item
  Número médio de pessoas por domicílio;
\item
  \% de domicílios alugados;
\item
  Número total de pessoas que avaliaram seu estado de de saúde como
  ruim;
\item
  Total de analfabetos de 7 a 14 anos;
\item
  Total de analfabetos de mais de 14 anos;
\item
  \% de analfabetos de 7 a 14 anos;
\item
  \% de analfabetos de mais de 14 anos;
\item
  Total de mulheres de 12 a 49 anos que tiveram filhos;
\item
  Total de mulheres de 12 a 49 anos que tiveram filhos vivos;
\item
  Total de mulheres de 12 a 49 anos que tiveram filhos mortos;
\item
  Número médio de filhos tidos por mulheres de 12 a 49 anos;
\item
  Razão de dependência.
\end{enumerate}

\begin{table}

\caption{\label{tab:epas}Estimativas de Efeitos de Plano Amostral (EPAs)
para variáveis selecionadas da PPV - Região Sudeste}
\centering
\begin{tabular}[t]{crrrcrrrcrrrcrrr}
\toprule
Parâmetro & Estimativa & Erro.Padrão & EPA\\
\midrule
1. & 3,62 & 0,05 & 2,64\\
2. & 10,70 & 1,15 & 2,97\\
3. & 1.208.123,00 & 146.681,00 & 3,37\\
4. & 1.174.220,00 & 127.982,00 & 2,64\\
5. & 4.792.344,00 & 318.877,00 & 4,17\\
\addlinespace
6. & 11,87 & 1,18 & 2,46\\
7. & 10,87 & 0,67 & 3,86\\
8. & 10.817.590,00 & 322.947,00 & 2,02\\
9. & 10.804.511,00 & 323.182,00 & 3,02\\
10. & 709.145,00 & 87.363,00 & 2,03\\
\addlinespace
11. & 1,39 & 0,03 & 1,26\\
12. & 0,53 & 0,01 & 1,99\\
\bottomrule
\end{tabular}
\end{table}

Como se pode observar da quarta coluna da Tabela \ref{tab:epas}, os
valores do efeito do plano amostral variam de um modesto 1,26 para o
número médio de filhos tidos por mulheres em idade fértil (12 a 49 anos
de idade) até um substancial 4,17 para o total de analfabetos entre
pessoas de mais de 14 anos. Nesse último caso, usar a estimativa de
variância como se o plano amostral fosse amostragem aleatória simples
implicaria em subestimar consideravelmente a variância da estimativa
pontual, que é mais que 4 vezes maior se consideramos o plano amostral
efetivamente utilizado.

Note que as variáveis e parâmetros cujas estimativas são apresentadas na
Tabela \ref{tab:epas} não foram escolhidas de forma a acentuar os
efeitos ilustrados, mas tão somente para representar distintos
parâmetros (totais, médias, proporções, razões) e variáveis de
interesse. Os resultados apresentados para as estimativas de EPA
ilustram bem o cenário típico em pesquisas amostrais complexas: o
impacto do plano amostral sobre a inferência varia conforme a variável e
o tipo de parâmetro de interesse. Note ainda que, à exceção dos dois
menores valores (1,26 e 1,99), todas as demais estimativas de EPA
apresentaram valores superiores a 2.

\section{Objetivos do Livro}\label{objetivos-do-livro}

Este livro tem três objetivos principais:

\begin{enumerate}
\def\labelenumi{\arabic{enumi})}
\item
  \textbf{Ilustrar e analisar o impacto das simplificações feitas ao
  utilizar pacotes usuais de análise de dados quando estes são
  provenientes de pesquisas amostrais complexas};
\item
  \textbf{Apresentar uma coleção de métodos e recursos computacionais
  disponíveis para análise de dados amostrais complexos, equipando o
  analista para trabalhar com tais dados, reduzindo assim o risco de
  inferências incorretas};
\item
  \textbf{Ilustrar o potencial analítico de muitas das pesquisas
  produzidas por agências de estatísticas oficiais para responder
  questões de interesse, mediante uso de ferramentas de análise
  estatística agora já bastante difundidas, aumentando assim o valor
  adicionado destas pesquisas}.
\end{enumerate}

Para alcançar tais objetivos, adotamos uma abordagem fortemente ancorada
na apresentação de exemplos de análises de dados obtidos em pesquisas
amostrais complexas, usando os recursos do pacote estatístico R
(\url{http://www.r-project.org/}).

A comparação dos resultados de análises feitas das duas formas
(considerando ou ignorando o plano amostral) permite avaliar o impacto
de não se considerar os pontos i) a iv) anteriormente citados. O ponto
iv) não é tratado de forma completa neste texto. O leitor interessado na
análise de dados sujeitos a não-resposta pode consultar
\citep{kalton83a}, \citep{LR2002}, \citep{Rubin87}, \citep{SSW92}, ou
Schafer (1997), por exemplo.

\section{Estrutura do Livro}\label{estrutura-do-livro}

O livro está organizado em catorze capítulos. Este primeiro capítulo
discute a motivação para estudar o assunto e apresenta uma ideia geral
dos objetivos e da estrutura do livro.

No segundo capítulo, procuramos dar uma visão das diferentes abordagens
utilizadas na análise estatística de dados de pesquisas amostrais
complexas. Apresentamos um referencial para inferência com ênfase no
\texttt{Modelo\ de\ Superpopulação} que incorpora, de forma natural,
tanto uma estrutura estocástica para descrever a geração dos dados
populacionais (modelo) como o plano amostral efetivamente utilizado para
obter os dados amostrais (plano amostral). As referências básicas para
seguir este capítulo são o capítulo 2 em \citep{Silva}, o capítulo 1 em
\citep{SHS89} e os capítulos 1 e 2 em \citep{CHSK2003}.

Esse referencial tem evoluído ao longo dos anos como uma forma de
permitir a incorporação de ideias e procedimentos de análise e
inferência usualmente associados à Estatística Clássica à prática da
análise e interpretação de dados provenientes de pesquisas amostrais.
Apesar dessa evolução, sua adoção não é livre de controvérsia e uma
breve revisão dessa discussão é apresentada no Capítulo \ref{refinf}.

No Capítulo \ref{capplanamo} apresentamos uma revisão sucinta, para
recordação, de alguns resultados básicos da Teoria de Amostragem,
requeridos nas partes subsequentes do livro. São discutidos os
procedimentos básicos para estimação de totais considerando o plano
amostral, e em seguida revistas algumas técnicas para estimação de
variâncias que são necessárias e úteis para o caso de estatísticas
complexas, tais como razões e outras estatísticas requeridas na
inferência analítica com dados amostrais. As referências centrais para
este capítulo são os capítulos 2 e 3 em \citep{SSW92}, \citep{W85} e
\citep{cochran}.

No Capítulo \ref{epa} introduzimos o conceito de
\texttt{Efeito\ do\ Plano\ Amostral\ (EPA)}, que permite avaliar o
impacto de ignorar a estrutura dos dados populacionais ou do plano
amostral sobre a estimativa da variância de um estimador. Para isso,
comparamos o estimador da variância apropriado para dados obtidos por
amostragem aleatória simples (hipótese de AAS) com o valor esperado
deste mesmo estimador sob a distribuição de aleatorização induzida pelo
plano amostral efetivamente utilizado (plano amostral verdadeiro). Aqui
a referência principal foi o livro \citep{SHS89}, complementado com o
texto de \citep{lethonen}.

No Capítulo \ref{ajmodpar} estudamos a questão do uso de pesos ao
analisar dados provenientes de pesquisas amostrais complexas, e
introduzimos um método geral, denominado
\texttt{Método\ de\ Máxima\ Pseudo\ Verossimilhança(MPV)}, para
incorporar os pesos e o plano amostral na obtenção não só de estimativas
de parâmetros dos modelos de interesse mais comuns, como também das
variâncias dessas estimativas. As referências básicas utilizadas nesse
capítulo foram \citep{SHS89}, \citep{Pfeff}, \citep{binder83} e o
capítulo 6 em \citep{Silva}.

O Capítulo \ref{modreg} trata da obtenção de
\texttt{Estimadores\ de\ Máxima\ Pseudo-Verossimilhança\ (EMPV)} e da
respectiva matriz de covariância para os parâmetros em modelos de
regressão linear e de regressão logística, quando os dados vêm de
pesquisas amostrais complexas. Apresentamos um exemplo de aplicação com
dados do Suplemento Trabalho da Pesquisa Nacional por Amostra de
Domicílios (PNAD) de 1990, onde ajustamos um modelo de regressão
logística. Neste exemplo, são feitas comparações entre resultados de
ajustes obtidos através de um programa especializado, o pacote
\texttt{survey} \citep{R-survey}, e através de um programa de uso geral,
a função \texttt{glm} do R. As referências centrais são o capítulo 6 em
\citep{Silva} e Binder(1983), além de \citep{Pessoa}.

Os Capítulos \ref{testqualajust} e \ref{testetab2} tratam da análise de
dados categóricos com ênfase na adaptação dos testes clássicos para
proporções, de independência e de homogeneidade em tabelas de
contingência, para dados provenientes de pesquisas amostrais complexas.
Apresentamos correções das estatísticas clássicas e também a estatística
de Wald baseada no plano amostral. As referências básicas usadas nesses
capítulos foram os o capítulo 4 em \citep{SHS89} e o capítulo 7
\citep{lethonen}. Também são apresentadas as ideias básicas de como
efetuar ajuste de modelos log-lineares a dados de frequências em tabelas
de múltiplas entradas.

O Capítulo \ref{estimacao-de-densidades} trata da estimação de
densidades e funções de distribuição, ferramentas que tem assumido
importância cada dia maior com a maior disponibilidade de microdados de
pesquisas amostrais para analistas fora das agências produtoras.

O Capítulo \ref{modelos-hierarquicos} trata da estimação e ajuste de
modelos hierárquicos considerando o plano amostral. Modelos hierárquicos
(ou modelos multiníveis) têm sido bastante utilizados para explorar
situações em que as relações entre variáveis de interesse em uma certa
população de unidades elementares (por exemplo, crianças em escolas,
pacientes em hospitais, empregados em empresas, moradores em regiões,
etc.) são afetadas por efeitos de grupos determinados ao nível de
unidades conglomeradas (os grupos). Ajustar e interpretar tais modelos é
tarefa mais difícil que o mero ajuste de modelos lineares, mesmo em
casos onde os dados são obtidos de forma exaustiva, mas ainda mais
complicada quando se trata de dados obtidos através de pesquisas
amostrais complexas. Várias alternativas de métodos para ajuste de
modelos hierárquicos estão disponíveis, e este capítulo apresenta uma
revisão de tais abordagens, ilustrando com aplicações a dados de
pesquisas amostrais de escolares.

O Capítulo \ref{nao-resposta} trata da não resposta e suas conseqüências
sobre a análise de dados. As abordagens de tratamento usuais,
reponderação e imputação, são descritas de maneira resumida, com
apresentação de alguns exemplos ilustrativos, e referências à ampla
literatura existente sobre o assunto. Em seguida destacamos a
importância de considerar os efeitos da não-resposta e dos tratamentos
compensatórios aplicados nas análises dos dados resultantes, destacando
em particular as ferramentas disponíveis para a estimação de variâncias
na presença de dados incompletos tratados mediante reponderação e/ou
imputação.

O Capítulo \ref{diagnostico-de-ajuste-de-modelo} trata de assunto ainda
emergente: diagnósticos do ajuste de modelos quando os dados foram
obtidos de amostras complexas. A literatura sobre o assunto ainda é
incipiente, mas o assunto é importante e procura-se estimular sua
investigação com a revisão do estado da arte no assunto.

O Capítulo \ref{agregdesag} discute algumas formas alternativas de
analisar dados de pesquisas complexas, contrapondo algumas abordagens
distintas à que demos preferência nos capítulos anteriores, para dar aos
leitores condições de apreciar de forma crítica o material apresentado
no restante deste livro. Entre as abordagens discutidas, há duas
principais: a denominada \texttt{análise\ desagregada}, e a abordagem
denominada \texttt{obtenção\ \ do\ modelo\ amostral} proposta por
\citep{PKR}. A chamada \texttt{análise\ desagregada} incorpora
explicitamente na análise vários aspectos do plano amostral utilizado,
através do emprego de modelos hierárquicos \citep{bryk}. Em contraste, a
abordagem adotada nos oito primeiros capítulos é denominada
\texttt{análise\ agregada}, e procura \texttt{eliminar} da análise
efeitos tais como conglomeração induzida pelo plano amostral,
considerando tais efeitos como \texttt{ruídos} ou fatores de perturbação
que \texttt{atrapalham} o emprego dos procedimentos clássicos de
estimação, ajuste de modelos e teste de hipóteses.

A abordagem de \texttt{obtenção\ do\ modelo\ amostral} parte de um
modelo de superpopulação e procura derivar o modelo amostral (ou que
valeria para as observações da amostra obtida), considerando modelos
para as probabilidades de inclusão dadas as variáveis auxiliares e as
variáveis resposta de interesse. Uma vez obtidos tais modelos, seu
ajuste prossegue por métodos convencionais tais como máxima
verossimilhança ou mesmo MCMC (Markov Chain Monte Carlo).

Por último, no Capítulo \ref{pacotes}, listamos alguns pacotes
computacionais especializados disponíveis para a análise de dados de
pesquisas amostrais complexas. Sem pretender ser exaustiva ou detalhada,
essa revisão dos pacotes procura também apresentar suas características
mais importantes. Alguns destes programas podem ser adquiridos
gratuitamente via \texttt{internet}, nos endereços fornecidos de seus
produtores. Com isto, pretendemos indicar aos leitores o caminho mais
curto para permitir a implementação prática das técnicas e métodos aqui
discutidos.

Uma das características que procuramos dar ao livro foi o emprego de
exemplos com dados reais, retirados principalmente da experiência do
IBGE com pesquisas amostrais complexas. Embora a experiência de fazer
inferência analítica com dados desse tipo seja ainda incipiente no
Brasil, acreditamos ser fundamental difundir essas ideias para alimentar
um processo de melhoria do aproveitamento dos dados das inúmeras
pesquisas realizadas pelo IBGE e instituições congêneres, que permita ir
além da tradicional estimação de totais, médias, proporções e razões.
Esperamos com esse livro fazer uma contribuição a esse processo.

Uma dificuldade em escrever um livro como este vem do fato de que não é
possível começar do zero: é preciso assumir algum conhecimento prévio de
ideias e conceitos necessários à compreensão do material tratado.
Procuramos tornar o livro acessível para um estudante de fim de curso de
graduação em Estatística. Por essa razão, optamos por não apresentar
provas de resultados e, sempre que possível, apresentar os conceitos e
ideias de maneira intuitiva, juntamente com uma discussão mais formal
para dar solidez aos resultados apresentados. As provas de vários dos
resultados aqui discutidos se restringem a material disponível apenas em
artigos em periódicos especializados estrangeiros e portanto, são de
acesso mais difícil. Ao leitor em busca de maior detalhamento e rigor,
sugerimos consultar diretamente as inúmeras referências incluídas ao
longo do texto. Para um tratamento mais profundo do assunto, os livros
de \citep{SHS89} e \citep{CHSK2003} são as referências centrais a
consultar. Para aqueles querendo um tratamento ainda mais prático que o
nosso, os livro de \citep{lethonen} e \citep{heeringa} podem ser opções
interessantes.

\section{Laboratório de R do Capítulo 1.}\label{epa}

\BeginKnitrBlock{example}
\protect\hypertarget{exm:exe12}{}{\label{exm:exe12} }Utilização do pacote
survey do R para estimar alguns totais e razões na Tabela \ref{tab:epas}
\EndKnitrBlock{example} Os exemplos a seguir utilizam dados da Pesquisa
de Padrões de Vida (\texttt{PPV}) do IBGE, cujo plano amostral
encontra-se descrito no Exemplo \ref{exm:distppv}. Os dados da
\texttt{PPV} que usamos aqui estão disponíveis no arquivo (data frame)
\texttt{ppv} do pacote \texttt{anamco}.

\begin{Shaded}
\begin{Highlighting}[]
\CommentTok{# Leitura dos dados}
\KeywordTok{library}\NormalTok{(anamco)}
\NormalTok{ppv_dat <-}\StringTok{ }\NormalTok{ppv}
\CommentTok{# Características dos dados da PPV}
\KeywordTok{dim}\NormalTok{(ppv_dat)}
\end{Highlighting}
\end{Shaded}

\begin{verbatim}
## [1] 19409    13
\end{verbatim}

\begin{Shaded}
\begin{Highlighting}[]
\KeywordTok{names}\NormalTok{(ppv_dat)}
\end{Highlighting}
\end{Shaded}

\begin{verbatim}
##  [1] "serie"    "ident"    "codmor"   "v04a01"   "v04a02"   "v04a03"  
##  [7] "estratof" "peso1"    "peso2"    "pesof"    "nsetor"   "regiao"  
## [13] "v02a08"
\end{verbatim}

Inicialmente, adicionamos quatro variáveis de interesse por meio de
transformação das variáveis existentes no data frame \texttt{ppv\_dat},
a saber:

\begin{itemize}
\tightlist
\item
  analf1 - indicador de analfabeto na faixa etária de 7 a 14 anos;
\item
  analf2 - indicador de analfabeto na faixa etária acima de 14 anos;
\item
  faixa1 - indicador de idade entre 7 e 14 anos;
\item
  faixa2 - indicador de idade acima de 14 anos;
\end{itemize}

\begin{Shaded}
\begin{Highlighting}[]
\CommentTok{# Adiciona variáveis ao arquivo da ppv_dat}
\NormalTok{ppv_dat <-}\StringTok{ }\KeywordTok{transform}\NormalTok{(ppv_dat, }
\DataTypeTok{analf1 =}\NormalTok{ ((v04a01 }\OperatorTok{==}\StringTok{ }\DecValTok{2} \OperatorTok{|}\StringTok{ }\NormalTok{v04a02 }\OperatorTok{==}\StringTok{ }\DecValTok{2}\NormalTok{) }\OperatorTok{&}\StringTok{ }\NormalTok{(v02a08 }\OperatorTok{>=}\StringTok{ }\DecValTok{7} \OperatorTok{&}\StringTok{ }\NormalTok{v02a08 }\OperatorTok{<=}\StringTok{ }\DecValTok{14}\NormalTok{)) }\OperatorTok{*}\StringTok{ }\DecValTok{1}\NormalTok{, }
\DataTypeTok{analf2 =}\NormalTok{ ((v04a01 }\OperatorTok{==}\StringTok{ }\DecValTok{2} \OperatorTok{|}\StringTok{ }\NormalTok{v04a02 }\OperatorTok{==}\StringTok{ }\DecValTok{2}\NormalTok{) }\OperatorTok{&}\StringTok{ }\NormalTok{(v02a08 }\OperatorTok{>}\DecValTok{14}\NormalTok{)) }\OperatorTok{*}\StringTok{ }\DecValTok{1}\NormalTok{, }
\DataTypeTok{faixa1 =}\NormalTok{ (v02a08 }\OperatorTok{>=}\StringTok{ }\DecValTok{7} \OperatorTok{&}\StringTok{ }\NormalTok{v02a08 }\OperatorTok{<=}\StringTok{ }\DecValTok{14}\NormalTok{) }\OperatorTok{*}\DecValTok{1}\NormalTok{, }
\DataTypeTok{faixa2 =}\NormalTok{ (v02a08 }\OperatorTok{>}\StringTok{ }\DecValTok{14}\NormalTok{) }\OperatorTok{*}\StringTok{ }\DecValTok{1}\NormalTok{)}
\CommentTok{#str(ppv_dat)}
\end{Highlighting}
\end{Shaded}

A seguir, mostramos como utilizar o pacote \texttt{survey}
\citep{R-survey} do R para obter algumas estimativas da Tabela
\ref{tab:epas}. Os dados da pesquisa estão contidos no data frame
\texttt{ppv\_dat}, que contém as variáveis que caracterizam o plano
amostral:

\begin{itemize}
\tightlist
\item
  estratof - identifica os estratos de seleção;
\item
  nsetor - identifica as unidades primárias de amostragem ou
  conglomerados;
\item
  pesof - identifica os pesos do plano amostral.
\end{itemize}

O passo fundamental para utilização do pacote \texttt{survey}
\citep{R-survey} é criar um objeto que guarde as informações relevantes
sobre a estrutura do plano amostral junto dos dados. Isso é feito por
meio da função \texttt{svydesign()}. As variáveis que definem estratos,
conglomerados e pesos na \texttt{PPV} são \texttt{estratof},
\texttt{nsetor} e \texttt{pesof} respectivamente. O objeto de desenho
amostral, \texttt{ppv\_plan} incorpora as informações da estrutura do
plano amostral adotado na \texttt{PPV}.

\begin{Shaded}
\begin{Highlighting}[]
\CommentTok{# Carrega o pacote survey}
\KeywordTok{library}\NormalTok{(survey)}
\CommentTok{# Cria objeto de desenho}
\NormalTok{ppv_plan <-}\StringTok{ }\KeywordTok{svydesign}\NormalTok{(}\DataTypeTok{ids =} \OperatorTok{~}\NormalTok{nsetor, }\DataTypeTok{strata =} \OperatorTok{~}\NormalTok{estratof, }\DataTypeTok{data =}\NormalTok{ ppv_dat, }
                      \DataTypeTok{nest =} \OtherTok{TRUE}\NormalTok{, }\DataTypeTok{weights =} \OperatorTok{~}\NormalTok{pesof)}
\end{Highlighting}
\end{Shaded}

Como todos os exemplos a seguir serão relativos a estimativas para a
Região Sudeste, vamos criar um objeto de desenho restrito a essa
região.-

\begin{Shaded}
\begin{Highlighting}[]
\NormalTok{ppv_se_plan <-}\StringTok{ }\KeywordTok{subset}\NormalTok{(ppv_plan, regiao }\OperatorTok{==}\StringTok{ "Sudeste"}\NormalTok{)}
\end{Highlighting}
\end{Shaded}

Para exemplificar as análises descritivas de interesse, vamos estimar
algumas características da população, descritas na Tabela
\ref{tab:epas}. Os totais das variáveis \texttt{analf1} e
\texttt{analf2} para a região Sudeste fornecem os resultados mostrados
nas linhas 4 e 5 da Tabela \ref{tab:epas}:

\begin{itemize}
\tightlist
\item
  total de analfabetos nas faixas etárias de 7 a 14 anos
  (\texttt{analf1}) e
\item
  total de analfabetos acima de 14 anos (\texttt{analf2}).
\end{itemize}

\begin{Shaded}
\begin{Highlighting}[]
\KeywordTok{svytotal}\NormalTok{(}\OperatorTok{~}\NormalTok{analf1, ppv_se_plan, }\DataTypeTok{deff =} \OtherTok{TRUE}\NormalTok{)}
\end{Highlighting}
\end{Shaded}

\begin{verbatim}
##          total      SE DEff
## analf1 1174220  127982 2,05
\end{verbatim}

\begin{Shaded}
\begin{Highlighting}[]
\KeywordTok{svytotal}\NormalTok{(}\OperatorTok{~}\NormalTok{analf2, ppv_se_plan, }\DataTypeTok{deff =} \OtherTok{TRUE}\NormalTok{)}
\end{Highlighting}
\end{Shaded}

\begin{verbatim}
##          total      SE DEff
## analf2 4792344  318877 3,32
\end{verbatim}

\begin{itemize}
\tightlist
\item
  percentual de analfabetos nas faixas etárias consideradas, que fornece
  os resultados nas linhas 6 e 7 da Tabela \ref{tab:epas}:
\end{itemize}

\begin{Shaded}
\begin{Highlighting}[]
\KeywordTok{svyratio}\NormalTok{(}\OperatorTok{~}\NormalTok{analf1, }\OperatorTok{~}\NormalTok{faixa1, ppv_se_plan)}
\end{Highlighting}
\end{Shaded}

\begin{verbatim}
## Ratio estimator: svyratio.survey.design2(~analf1, ~faixa1, ppv_se_plan)
## Ratios=
##        faixa1
## analf1  0,119
## SEs=
##        faixa1
## analf1 0,0118
\end{verbatim}

\begin{Shaded}
\begin{Highlighting}[]
\KeywordTok{svyratio}\NormalTok{(}\OperatorTok{~}\NormalTok{analf2, }\OperatorTok{~}\NormalTok{faixa2, ppv_se_plan)}
\end{Highlighting}
\end{Shaded}

\begin{verbatim}
## Ratio estimator: svyratio.survey.design2(~analf2, ~faixa2, ppv_se_plan)
## Ratios=
##        faixa2
## analf2  0,109
## SEs=
##         faixa2
## analf2 0,00673
\end{verbatim}

Uma alternativa para obter estimativa por domínios é utilizar a função
\texttt{svyby()} do pacote \texttt{survey} \citep{R-survey}. Assim,
poderíamos estimar os totais da variável \texttt{analf1} para as regiões
\texttt{Nordeste\ (1)} e \texttt{Sudeste\ (2)} da seguinte forma:

\begin{Shaded}
\begin{Highlighting}[]
\KeywordTok{svyby}\NormalTok{(}\OperatorTok{~}\NormalTok{analf1, }\OperatorTok{~}\NormalTok{regiao, ppv_plan, svytotal, }\DataTypeTok{deff =} \OtherTok{TRUE}\NormalTok{)}
\end{Highlighting}
\end{Shaded}

\begin{verbatim}
##            regiao  analf1     se DEff.analf1
## Nordeste Nordeste 3512866 352620        9,66
## Sudeste   Sudeste 1174220 127982        2,05
\end{verbatim}

Observe que as estimativas de totais e desvios padrão obtidas coincidem
com as Tabela \ref{tab:epas}, porém as estimativas de Efeitos de Plano
Amostral(EPA) são distintas. Uma explicação detalhada para essa
diferença será apresentada no capítulo 4, após a discussão do conceito
de efeito do plano amostral e de métodos para sua estimação.

\section{Laboratório de R do Capítulo 1 -
Extra.}\label{laboratorio-de-r-do-capitulo-1---extra.}

\BeginKnitrBlock{example}
\protect\hypertarget{exm:exe13}{}{\label{exm:exe13} }Exemplo anterior usando
o pacote srvyr
\EndKnitrBlock{example} - Carrega o pacote \texttt{srvyr}:

\begin{Shaded}
\begin{Highlighting}[]
\KeywordTok{library}\NormalTok{(srvyr)}
\end{Highlighting}
\end{Shaded}

\begin{itemize}
\tightlist
\item
  Cria objeto de desenho:
\end{itemize}

\begin{Shaded}
\begin{Highlighting}[]
\NormalTok{ppv_plan <-}\StringTok{ }\NormalTok{ppv_dat }\OperatorTok\StringTok{ }
\StringTok{            }\KeywordTok{as_survey_design}\NormalTok{(}\DataTypeTok{strata =}\NormalTok{ estratof, }\DataTypeTok{ids =}\NormalTok{ nsetor, }\DataTypeTok{nest =} \OtherTok{TRUE}\NormalTok{, }
                             \DataTypeTok{weights =}\NormalTok{ pesof)}
\end{Highlighting}
\end{Shaded}

Vamos criar novamente as variáveis derivadas necessárias, mas observe
que, desta vez, estas variáveis estão sendo adicionadas ao objeto que já
contem os dados e as informações da estrutura do plano amostral.

\begin{Shaded}
\begin{Highlighting}[]
\NormalTok{ppv_plan <-}\StringTok{ }\NormalTok{ppv_plan }\OperatorTok\StringTok{ }
\StringTok{            }\KeywordTok{mutate}\NormalTok{(}
\DataTypeTok{analf1 =} \KeywordTok{as.numeric}\NormalTok{((v04a01 }\OperatorTok{==}\StringTok{ }\DecValTok{2} \OperatorTok{|}\StringTok{ }\NormalTok{v04a02 }\OperatorTok{==}\StringTok{ }\DecValTok{2}\NormalTok{) }\OperatorTok{&}\StringTok{ }\NormalTok{(v02a08 }\OperatorTok{>=}\StringTok{ }\DecValTok{7} \OperatorTok{&}\StringTok{ }\NormalTok{v02a08 }\OperatorTok{<=}\StringTok{ }\DecValTok{14}\NormalTok{)), }
\DataTypeTok{analf2 =} \KeywordTok{as.numeric}\NormalTok{((v04a01 }\OperatorTok{==}\StringTok{ }\DecValTok{2} \OperatorTok{|}\StringTok{ }\NormalTok{v04a02 }\OperatorTok{==}\StringTok{ }\DecValTok{2}\NormalTok{) }\OperatorTok{&}\StringTok{ }\NormalTok{(v02a08 }\OperatorTok{>}\DecValTok{14}\NormalTok{)), }
\DataTypeTok{faixa1 =} \KeywordTok{as.numeric}\NormalTok{(v02a08 }\OperatorTok{>=}\StringTok{ }\DecValTok{7} \OperatorTok{&}\StringTok{ }\NormalTok{v02a08 }\OperatorTok{<=}\StringTok{ }\DecValTok{14}\NormalTok{), }
\DataTypeTok{faixa2 =} \KeywordTok{as.numeric}\NormalTok{(v02a08 }\OperatorTok{>}\StringTok{ }\DecValTok{14}\NormalTok{)   }
\NormalTok{)}
\end{Highlighting}
\end{Shaded}

\begin{itemize}
\tightlist
\item
  Estimar a taxa de analfabetos por região para as faixas etárias de
  7-14 anos e mais de 14 anos.
\end{itemize}

\begin{Shaded}
\begin{Highlighting}[]
\NormalTok{result1 <-}\StringTok{ }\NormalTok{ppv_plan }\OperatorTok\StringTok{  }
\StringTok{           }\KeywordTok{group_by}\NormalTok{(regiao) }\OperatorTok\StringTok{ }
\StringTok{           }\KeywordTok{summarise}\NormalTok{(}
           \DataTypeTok{taxa_analf1 =} \DecValTok{100}\OperatorTok{*}\KeywordTok{survey_ratio}\NormalTok{(analf1, faixa1),}
           \DataTypeTok{taxa_analf2 =} \DecValTok{100}\OperatorTok{*}\KeywordTok{survey_ratio}\NormalTok{(analf2, faixa2)  }
\NormalTok{           )}
\NormalTok{result1}\OperatorTok{$}\NormalTok{regiao <-}\KeywordTok{c}\NormalTok{(}\StringTok{"Nordeste"}\NormalTok{,}\StringTok{"Sudeste"}\NormalTok{)}
\NormalTok{knitr}\OperatorTok{::}\KeywordTok{kable}\NormalTok{(}\KeywordTok{as.data.frame}\NormalTok{(result1), }\DataTypeTok{booktabs =} \OtherTok{TRUE}\NormalTok{, }\DataTypeTok{row.names =} \OtherTok{FALSE}\NormalTok{, }\DataTypeTok{digits =} \DecValTok{1}\NormalTok{,}
             \DataTypeTok{align =} \StringTok{"crrrr"}\NormalTok{, }\DataTypeTok{format.args=} \KeywordTok{list}\NormalTok{(}\DataTypeTok{decimal.mark=}\StringTok{","}\NormalTok{),}
\DataTypeTok{caption =} \StringTok{"Proporção de analfabetos para faixas etárias 7-14 anos e mais de 14 anos"}\NormalTok{)}
\end{Highlighting}
\end{Shaded}

\begin{table}

\caption{\label{tab:unnamed-chunk-12}Proporção de analfabetos para faixas etárias 7-14 anos e mais de 14 anos}
\centering
\begin{tabular}[t]{crrrrcrrrrcrrrrcrrrrcrrrr}
\toprule
regiao & taxa\_analf1 & taxa\_analf1\_se & taxa\_analf2 & taxa\_analf2\_se\\
\midrule
Nordeste & 42,3 & 3,1 & 33,6 & 1,6\\
Sudeste & 11,9 & 1,2 & 10,9 & 0,7\\
\bottomrule
\end{tabular}
\end{table}

\chapter{Referencial para Inferência}\label{refinf}

\section{Modelagem - Primeiras Ideias}\label{classic}

\subsection{Abordagem 1 - Modelagem
Clássica}\label{abordagem-1---modelagem-classica}

\subsection{Abordagem 2 - Amostragem
Probabilística}\label{abordagem-2---amostragem-probabilistica}

\subsection{Discussão das Abordagens 1 e
2}\label{discussao-das-abordagens-1-e-2}

\subsection{Abordagem 3 - Modelagem de
Superpopulação}\label{modelsuperpop}

\section{Fontes de Variação}\label{fontes-de-variacao}

\section{Modelos de Superpopulação}\label{modelos-de-superpopulacao}

\section{Planejamento Amostral}\label{planamo}

\section{Planos Amostrais Informativos e Ignoráveis}\label{inform}

\chapter{Estimação Baseada no Plano Amostral}\label{capplanamo}

\section{Estimação de Totais}\label{estimatotais}

\section{Por que Estimar Variâncias}\label{por-que-estimar-variancias}

\section{Linearização de Taylor para Estimar variâncias}\label{taylor}

\section{Método do Conglomerado
Primário}\label{metodo-do-conglomerado-primario}

\section{Métodos de Replicação}\label{metodos-de-replicacao}

\section{Laboratório de R}\label{laboratorio-de-r}

\section{faixa}\label{faixa}

\section{0,0118}\label{section}

\section{Ratio estimator:
svyratio.survey.design2(\textasciitilde{}analf.faixa,
\textasciitilde{}faixa,
ppv\_se\_plan)}\label{ratio-estimator-svyratio.survey.design2analf.faixa-faixa-ppv_se_plan}

\section{Ratios=}\label{ratios}

\section{faixa}\label{faixa-1}

\section{analf.faixa 0,119}\label{analf.faixa-0119}

\section{SEs=}\label{ses}

\section{faixa}\label{faixa-2}

\section{analf.faixa 0,0118}\label{analf.faixa-00118}

\section{Ratio estimator:
svyratio.svyrep.design(\textasciitilde{}analf.faixa,
\textasciitilde{}faixa,
ppv\_se\_plan\_jkn)}\label{ratio-estimator-svyratio.svyrep.designanalf.faixa-faixa-ppv_se_plan_jkn}

\section{Ratios=}\label{ratios-1}

\section{faixa}\label{faixa-3}

\section{analf.faixa 0,119}\label{analf.faixa-0119-1}

\section{SEs=}\label{ses-1}

\section{{[},1{]}}\label{section-1}

\section{{[}1,{]} 0,0118}\label{section-2}

\section{Ratio estimator:
svyratio.svyrep.design(\textasciitilde{}analf.faixa,
\textasciitilde{}faixa,
ppv\_se\_plan\_boot)}\label{ratio-estimator-svyratio.svyrep.designanalf.faixa-faixa-ppv_se_plan_boot}

\section{Ratios=}\label{ratios-2}

\section{faixa}\label{faixa-4}

\section{analf.faixa 0,119}\label{analf.faixa-0119-2}

\section{SEs=}\label{ses-2}

\section{{[},1{]}}\label{section-3}

\section{{[}1,{]} 0,0129}\label{section-4}

\section{\texorpdfstring{{[}1{]}
``svyrep.design''}{{[}1{]} svyrep.design}}\label{svyrep.design}

\section{\texorpdfstring{{[}1{]} ``repweights'' ``pweights''
``type''}{{[}1{]} repweights pweights type}}\label{repweights-pweights-type}

\section{\texorpdfstring{{[}4{]} ``rho'' ``scale''
``rscales''}{{[}4{]} rho scale rscales}}\label{rho-scale-rscales}

\section{\texorpdfstring{{[}7{]} ``call'' ``combined.weights''
``selfrep''}{{[}7{]} call combined.weights selfrep}}\label{call-combined.weights-selfrep}

\section{\texorpdfstring{{[}10{]} ``mse'' ``variables''
``degf''}{{[}10{]} mse variables degf}}\label{mse-variables-degf}

\section{{[}1{]} 8903}\label{section-5}

\section{{[}1{]} 276}\label{section-6}

\section{num {[}1:8903, 1:276{]} 0 0 1,06 1,06 1,06
\ldots{}}\label{num-18903-1276-0-0-106-106-106}

\section{{[}1{]} 0,0118}\label{section-7}

\section{theta SE}\label{theta-se}

\section{{[}1,{]} 0,119 0,01}\label{section-8}

\section{theta SE}\label{theta-se-1}

\section{{[}1,{]} 0,504 0,05}\label{section-9}

\section{nlcon SE}\label{nlcon-se}

\section{contrast 0,504 0,05}\label{contrast-0504-005}

\chapter{Efeitos do Plano Amostral}\label{epa}

\section{Introdução}\label{introducao}

\section{Efeito do Plano Amostral (EPA) de
Kish}\label{efeito-do-plano-amostral-epa-de-kish}

\section{Efeito do Plano Amostral
Ampliado}\label{efeito-do-plano-amostral-ampliado}

\section{245,188/1719,979=0,143}\label{section-10}

\section{0,401/1,207=0,332}\label{section-11}

\section{244,176/1613,3=0,151}\label{section-12}

\section{0,435/1,188=0,366}\label{section-13}

\section{Intervalos de Confiança e Testes de Hipóteses}\label{icth}

\section{Efeitos Multivariados de Plano
Amostral}\label{efeitos-multivariados-de-plano-amostral}

\section{Laboratório de R}\label{laboratorio-de-r-1}

\section{Média das estimativas pontuais para as 500 amostras
aas}\label{media-das-estimativas-pontuais-para-as-500-amostras-aas}

\section{{[}1{]} 163,50 4,17}\label{section-14}

\section{Média das estimativas pontuais para as 500 amostras
aes}\label{media-das-estimativas-pontuais-para-as-500-amostras-aes}

\section{sal rec}\label{sal-rec}

\section{78,07 2,06}\label{section-15}

\section{{[},1{]} {[},2{]}}\label{section-16}

\section{{[}1,{]} 1720,0 26,78}\label{section-17}

\section{{[}2,{]} 26,8 1,21}\label{section-18}

\section{sal rec}\label{sal-rec-1}

\section{sal 245,19 3,172}\label{sal-24519-3172}

\section{rec 3,17 0,401}\label{rec-317-0401}

\section{Matriz de covariância
populacional}\label{matriz-de-covariancia-populacional}

\section{Matriz de covariância considerando o plano
amostral}\label{matriz-de-covariancia-considerando-o-plano-amostral}

\section{verdadeiro}\label{verdadeiro}

\section{estimativa de efeitos generalizados do plano
amostral}\label{estimativa-de-efeitos-generalizados-do-plano-amostral}

\section{{[}1{]} 706 14}\label{section-19}

\section{\texorpdfstring{{[}1{]} ``setor'' ``np'' ``domic'' ``sexo''
``renda'' ``lrenda''
``raca''}{{[}1{]} setor np domic sexo renda lrenda raca}}\label{setor-np-domic-sexo-renda-lrenda-raca}

\section{\texorpdfstring{{[}8{]} ``estudo'' ``idade'' ``na'' ``peso''
``domtot'' ``peso1''
``pesof''}{{[}8{]} estudo idade na peso domtot peso1 pesof}}\label{estudo-idade-na-peso-domtot-peso1-pesof}

\section{raca renda se}\label{raca-renda-se}

\section{1 1 110406 11262}\label{section-20}

\section{2 2 73560 8207}\label{section-21}

\section{sexo renda se}\label{sexo-renda-se}

\section{1 1 108746 11696}\label{section-22}

\section{2 2 40039 4042}\label{section-23}

\section{}\label{section-24}

\section{Design-based t-test}\label{design-based-t-test}

\section{}\label{section-25}

\section{data: renda \textasciitilde{} sexo}\label{data-renda-sexo}

\section{t = -6, df = 20, p-value =
0,000005}\label{t--6-df-20-p-value-0000005}

\section{alternative hypothesis: true difference in mean is not equal to
0}\label{alternative-hypothesis-true-difference-in-mean-is-not-equal-to-0}

\section{sample estimates:}\label{sample-estimates}

\section{difference in mean}\label{difference-in-mean}

\section{-68707}\label{section-26}

\section{}\label{section-27}

\section{Design-based t-test}\label{design-based-t-test-1}

\section{}\label{section-28}

\section{data: renda \textasciitilde{} raca}\label{data-renda-raca}

\section{t = -4, df = 20, p-value =
0,0006}\label{t--4-df-20-p-value-00006}

\section{alternative hypothesis: true difference in mean is not equal to
0}\label{alternative-hypothesis-true-difference-in-mean-is-not-equal-to-0-1}

\section{sample estimates:}\label{sample-estimates-1}

\section{difference in mean}\label{difference-in-mean-1}

\section{-36846}\label{section-29}

\chapter{Ajuste de Modelos Paramétricos}\label{ajmodpar}

\section{Introdução}\label{modpar1}

\section{Método de Máxima Verossimilhança
(MV)}\label{metodo-de-maxima-verossimilhanca-mv}

\section{Ponderação de Dados
Amostrais}\label{ponderacao-de-dados-amostrais}

\section{Método de Máxima Pseudo-Verossimilhança}\label{modpar3}

\section{Robustez do Procedimento
MPV}\label{robustez-do-procedimento-mpv}

\section{Desvantagens da Inferência de
Aleatorização}\label{desvantagens-da-inferencia-de-aleatorizacao}

\section{Laboratório de R}\label{laboratorio-de-r-2}

\chapter{Modelos de Regressão}\label{modreg}

\section{Modelo de Regressão Linear Normal}\label{modlinear}

\subsection{Especificação do Modelo}\label{especificacao-do-modelo}

\subsection{Pseudo-parâmetros do
Modelo}\label{pseudo-parametros-do-modelo}

\subsection{Estimadores de MPV dos Parâmetros do
Modelo}\label{estimadores-de-mpv-dos-parametros-do-modelo}

\subsection{Estimação da Variância de Estimadores de
MPV}\label{estimacao-da-variancia-de-estimadores-de-mpv}

\section{Modelo de Regressão Logística}\label{modlogist}

\section{Teste de Hipóteses}\label{teste-de-hipoteses}

\section{Laboratório de R}\label{laboratorio-de-r-3}

\section{\texorpdfstring{{[}1{]} ``stra'' ``psu'' ``pesopes''
``informal'' ``sx''
``id''}{{[}1{]} stra psu pesopes informal sx id}}\label{stra-psu-pesopes-informal-sx-id}

\section{\texorpdfstring{{[}7{]} ``ae'' ``ht'' ``re''
``um''}{{[}7{]} ae ht re um}}\label{ae-ht-re-um}

\section{stra psu pesopes informal sx id
ae}\label{stra-psu-pesopes-informal-sx-id-ae}

\section{\texorpdfstring{``numeric'' ``numeric'' ``numeric'' ``numeric''
``numeric'' ``numeric''
``numeric''}{numeric numeric numeric numeric numeric numeric numeric}}\label{numeric-numeric-numeric-numeric-numeric-numeric-numeric}

\section{ht re um}\label{ht-re-um}

\section{\texorpdfstring{``numeric'' ``numeric''
``numeric''}{numeric numeric numeric}}\label{numeric-numeric-numeric}

\section{Wald test for ht:re}\label{wald-test-for-htre}

\section{in svyglm(formula = informal \textasciitilde{} sx + ae + ht +
id + re + sx * id
+}\label{in-svyglmformula-informal-sx-ae-ht-id-re-sx-id}

\section{sx * ht + ae * ht + ht * id + ht * re, design =
pnad.des,}\label{sx-ht-ae-ht-ht-id-ht-re-design-pnad.des}

\section{family = quasibinomial())}\label{family-quasibinomial}

\section{F = 6,74 on 4 and 616 df: p=
0,00003}\label{f-674-on-4-and-616-df-p-000003}

\chapter{Testes de Qualidade de Ajuste}\label{testqualajust}

\section{Introdução}\label{introducao-1}

\section{Teste para uma Proporção}\label{teste-para-uma-proporcao}

\subsection{Correção de Estatísticas
Clássicas}\label{correcao-de-estatisticas-classicas}

\subsection{Estatística de Wald}\label{estatistica-de-wald}

\section{Teste para Várias
Proporções}\label{teste-para-varias-proporcoes}

\subsection{Estatística de Wald Baseada no Plano
Amostral}\label{estatistica-de-wald-baseada-no-plano-amostral}

\subsection{Situações Instáveis}\label{situacoes-instaveis}

\subsection{Estatística de Pearson com Ajuste de
Rao-Scott}\label{raoscott}

\section{Laboratório de R}\label{laboratorio-de-r-4}

\section{{[},1{]}}\label{section-30}

\section{{[}1,{]} 5,74}\label{section-31}

\section{{[},1{]}}\label{section-32}

\section{{[}1,{]} 0,219}\label{section-33}

\section{{[},1{]}}\label{section-34}

\section{{[}1,{]} 11,5}\label{section-35}

\section{{[},1{]}}\label{section-36}

\section{{[}1,{]} 0,021}\label{section-37}

\chapter{Testes em Tabelas de Duas Entradas}\label{testetab2}

\section{Introdução}\label{introducao-2}

\section{Tabelas 2x2}\label{tabelas22}

\subsection{Teste de Independência}\label{teste-de-independencia}

\subsection{Teste de Homogeneidade}\label{teste-de-homogeneidade}

\subsection{Efeitos de Plano Amostral nas
Celas}\label{efeitos-de-plano-amostral-nas-celas}

\section{Tabelas de Duas Entradas (Caso
Geral)}\label{tabelas-de-duas-entradas-caso-geral}

\subsection{Teste de Homogeneidade}\label{teste-de-homogeneidade-1}

\subsection{Teste de Independência}\label{teste-de-independencia-1}

\subsection{Estatística de Wald Baseada no Plano
Amostral}\label{estatistica-de-wald-baseada-no-plano-amostral-1}

\subsection{Estatística de Pearson com Ajuste de
Rao-Scott}\label{estatistica-de-pearson-com-ajuste-de-rao-scott}

\section{Laboratório de R}\label{laboratorio-de-r-5}

\section{\texorpdfstring{{[}1{]} ``stra'' ``psu'' ``pesopes''
``informal'' ``sx''
``id''}{{[}1{]} stra psu pesopes informal sx id}}\label{stra-psu-pesopes-informal-sx-id-1}

\section{\texorpdfstring{{[}7{]} ``ae'' ``ht'' ``re''
``um''}{{[}7{]} ae ht re um}}\label{ae-ht-re-um-1}

\section{stra psu pesopes informal sx id
ae}\label{stra-psu-pesopes-informal-sx-id-ae-1}

\section{\texorpdfstring{``numeric'' ``numeric'' ``numeric'' ``numeric''
``numeric'' ``numeric''
``numeric''}{numeric numeric numeric numeric numeric numeric numeric}}\label{numeric-numeric-numeric-numeric-numeric-numeric-numeric-1}

\section{ht re um}\label{ht-re-um-1}

\section{\texorpdfstring{``numeric'' ``numeric''
``numeric''}{numeric numeric numeric}}\label{numeric-numeric-numeric-1}

\section{mean SE}\label{mean-se}

\section{sx1 0,657 0,01}\label{sx1-0657-001}

\section{sx2 0,343 0,01}\label{sx2-0343-001}

\section{mean SE}\label{mean-se-1}

\section{re1 0,155 0,01}\label{re1-0155-001}

\section{re2 0,584 0,01}\label{re2-0584-001}

\section{re3 0,261 0,01}\label{re3-0261-001}

\section{mean SE}\label{mean-se-2}

\section{ae1 0,313 0,01}\label{ae1-0313-001}

\section{ae2 0,320 0,01}\label{ae2-0320-001}

\section{ae3 0,367 0,01}\label{ae3-0367-001}

\section{ht1 ht2 ht3}\label{ht1-ht2-ht3}

\section{0,210 0,615 0,175}\label{section-38}

\section{\$names}\label{names}

\section{\texorpdfstring{{[}1{]} ``ht1'' ``ht2''
``ht3''}{{[}1{]} ht1 ht2 ht3}}\label{ht1-ht2-ht3-1}

\section{}\label{section-39}

\section{\$var}\label{var}

\section{ht1 ht2 ht3}\label{ht1-ht2-ht3-2}

\section{ht1 0,00003666 -0,0000332
-0,00000344}\label{ht1-000003666--00000332--000000344}

\section{ht2 -0,00003323 0,0000676
-0,00003436}\label{ht2--000003323-00000676--000003436}

\section{ht3 -0,00000344 -0,0000344
0,00003780}\label{ht3--000000344--00000344-000003780}

\section{}\label{section-40}

\section{\$statistic}\label{statistic}

\section{\texorpdfstring{{[}1{]} ``mean''}{{[}1{]} mean}}\label{mean}

\section{}\label{section-41}

\section{\$class}\label{class}

\section{\texorpdfstring{{[}1{]}
``svystat''}{{[}1{]} svystat}}\label{svystat}

\section{ht1 ht2 ht3}\label{ht1-ht2-ht3-3}

\section{ht1 0,00003666 -0,0000332
-0,00000344}\label{ht1-000003666--00000332--000000344-1}

\section{ht2 -0,00003323 0,0000676
-0,00003436}\label{ht2--000003323-00000676--000003436-1}

\section{ht3 -0,00000344 -0,0000344
0,00003780}\label{ht3--000000344--00000344-000003780-1}

\section{ht1 ht2 ht3}\label{ht1-ht2-ht3-4}

\section{ht1 0,00003666 -0,0000332
-0,00000344}\label{ht1-000003666--00000332--000000344-2}

\section{ht2 -0,00003323 0,0000676
-0,00003436}\label{ht2--000003323-00000676--000003436-2}

\section{ht3 -0,00000344 -0,0000344
0,00003780}\label{ht3--000000344--00000344-000003780-2}

\section{mean SE DEff}\label{mean-se-deff}

\section{re1 0,15546 0,00690 2,36}\label{re1-015546-000690-236}

\section{re2 0,58356 0,00820 1,80}\label{re2-058356-000820-180}

\section{re3 0,26098 0,00961 3,12}\label{re3-026098-000961-312}

\section{re}\label{re}

\section{sx 1 2 3}\label{sx-1-2-3}

\section{1 0,073 0,388 0,196}\label{section-42}

\section{2 0,082 0,195 0,065}\label{section-43}

\section{sx re1 re2 re3 se.re1 se.re2
se.re3}\label{sx-re1-re2-re3-se.re1-se.re2-se.re3}

\section{1 1 0,111 0,591 0,298 0,00573 0,0103 0,0111}\label{section-44}

\section{2 2 0,240 0,570 0,190 0,01250 0,0119 0,0111}\label{section-45}

\section{mean SE DEff}\label{mean-se-deff-1}

\section{I((sx == 1 \& re == 1) * 1) 0,07299 0,00383
1,42}\label{isx-1-re-1-1-007299-000383-142}

\section{re}\label{re-1}

\section{sx 1 2 3}\label{sx-1-2-3-1}

\section{1 0,0730 0,3882 0,1959}\label{section-46}

\section{2 0,0825 0,1953 0,0651}\label{section-47}

\section{re}\label{re-2}

\section{sx 1 2 3}\label{sx-1-2-3-2}

\section{1 7,30 38,82 19,59}\label{section-48}

\section{2 8,25 19,53 6,51}\label{section-49}

\section{{[}1{]} 1,64}\label{section-50}

\section{}\label{section-51}

\section{Pearson's Chi-squared test}\label{pearsons-chi-squared-test}

\section{}\label{section-52}

\section{data: tab.amo}\label{data-tab.amo}

\section{X-squared = 200, df = 2, p-value
\textless{}0,0000000000000002}\label{x-squared-200-df-2-p-value-00000000000000002}

\chapter{Estimação de densidades}\label{estimacao-de-densidades}

\section{Introdução}\label{introducao-3}

\chapter{Modelos Hierárquicos}\label{modelos-hierarquicos}

\section{Introdução}\label{introducao-4}

\chapter{Não-Resposta}\label{nao-resposta}

\section{Introdução}\label{introducao-5}

\chapter{Diagnóstico de ajuste de
modelo}\label{diagnostico-de-ajuste-de-modelo}

\section{Introdução}\label{introducao-6}

\chapter{Agregação vs.~Desagregação}\label{agregdesag}

\section{Introdução}\label{introducao-7}

\section{Modelagem da Estrutura
Populacional}\label{modelagem-da-estrutura-populacional}

\section{Modelos Hierárquicos}\label{modelos-hierarquicos-1}

\section{Análise Desagregada: Prós e
Contras}\label{analise-desagregada-pros-e-contras}

\chapter{Pacotes para Analisar Dados Amostrais}\label{pacotes}

\section{Introdução}\label{introducao-8}

\section{Pacotes Computacionais}\label{pacotes-computacionais}

\chapter{Placeholder}\label{placeholder}

\bibliography{packages,book}


\end{document}
