\documentclass[]{book}
\usepackage{lmodern}
\usepackage{amssymb,amsmath}
\usepackage{ifxetex,ifluatex}
\usepackage{fixltx2e} % provides \textsubscript
\ifnum 0\ifxetex 1\fi\ifluatex 1\fi=0 % if pdftex
  \usepackage[T1]{fontenc}
  \usepackage[utf8]{inputenc}
\else % if luatex or xelatex
  \ifxetex
    \usepackage{mathspec}
  \else
    \usepackage{fontspec}
  \fi
  \defaultfontfeatures{Ligatures=TeX,Scale=MatchLowercase}
\fi
% use upquote if available, for straight quotes in verbatim environments
\IfFileExists{upquote.sty}{\usepackage{upquote}}{}
% use microtype if available
\IfFileExists{microtype.sty}{%
\usepackage{microtype}
\UseMicrotypeSet[protrusion]{basicmath} % disable protrusion for tt fonts
}{}
\usepackage[margin=1in]{geometry}
\usepackage{hyperref}
\hypersetup{unicode=true,
            pdftitle={Análise de Dados Amostrais Complexos},
            pdfauthor={Djalma Pessoa e Pedro Nascimento Silva},
            pdfborder={0 0 0},
            breaklinks=true}
\urlstyle{same}  % don't use monospace font for urls
\usepackage{natbib}
\bibliographystyle{apalike}
\usepackage{longtable,booktabs}
\usepackage{graphicx,grffile}
\makeatletter
\def\maxwidth{\ifdim\Gin@nat@width>\linewidth\linewidth\else\Gin@nat@width\fi}
\def\maxheight{\ifdim\Gin@nat@height>\textheight\textheight\else\Gin@nat@height\fi}
\makeatother
% Scale images if necessary, so that they will not overflow the page
% margins by default, and it is still possible to overwrite the defaults
% using explicit options in \includegraphics[width, height, ...]{}
\setkeys{Gin}{width=\maxwidth,height=\maxheight,keepaspectratio}
\IfFileExists{parskip.sty}{%
\usepackage{parskip}
}{% else
\setlength{\parindent}{0pt}
\setlength{\parskip}{6pt plus 2pt minus 1pt}
}
\setlength{\emergencystretch}{3em}  % prevent overfull lines
\providecommand{\tightlist}{%
  \setlength{\itemsep}{0pt}\setlength{\parskip}{0pt}}
\setcounter{secnumdepth}{5}
% Redefines (sub)paragraphs to behave more like sections
\ifx\paragraph\undefined\else
\let\oldparagraph\paragraph
\renewcommand{\paragraph}[1]{\oldparagraph{#1}\mbox{}}
\fi
\ifx\subparagraph\undefined\else
\let\oldsubparagraph\subparagraph
\renewcommand{\subparagraph}[1]{\oldsubparagraph{#1}\mbox{}}
\fi

%%% Use protect on footnotes to avoid problems with footnotes in titles
\let\rmarkdownfootnote\footnote%
\def\footnote{\protect\rmarkdownfootnote}

%%% Change title format to be more compact
\usepackage{titling}

% Create subtitle command for use in maketitle
\newcommand{\subtitle}[1]{
  \posttitle{
    \begin{center}\large#1\end{center}
    }
}

\setlength{\droptitle}{-2em}
  \title{Análise de Dados Amostrais Complexos}
  \pretitle{\vspace{\droptitle}\centering\huge}
  \posttitle{\par}
  \author{Djalma Pessoa e Pedro Nascimento Silva}
  \preauthor{\centering\large\emph}
  \postauthor{\par}
  \predate{\centering\large\emph}
  \postdate{\par}
  \date{2017-09-05}

% \documentclass[]{book}
% \usepackage{booktabs}
% \usepackage[brazil]{babel}
% \newtheorem{example}{Exemplo}
% \numberwithin{example}{chapter}
% \newtheorem{remark}{Observação}
% \numberwithin{remark}{chapter}
% \newtheorem{definition}{Definição}
% \numberwithin{definition}{chapter}
% \begin{document}

\usepackage{amsthm}
\newtheorem{theorem}{Teorema}[chapter]
\newtheorem{lemma}{Lema}[chapter]
\theoremstyle{definition}
\newtheorem{definition}{Definição}[chapter]
\newtheorem{corollary}{Corolário}[chapter]
\newtheorem{proposition}{Proposição}[chapter]
\theoremstyle{definition}
\newtheorem{example}{Example}[chapter]
\theoremstyle{definition}
\newtheorem{exercise}{Exercise}[chapter]
\theoremstyle{remark}
\newtheorem*{remark}{Observação }
\newtheorem*{solution}{Solution}
\begin{document}
\maketitle

{
\setcounter{tocdepth}{1}
\tableofcontents
}
\chapter*{Prefácio}\label{prefacio}
\addcontentsline{toc}{chapter}{Prefácio}

Uma preocupação básica de toda instituição produtora de informações
estatísticas é com a utilização `'correta'' de seus dados. Isso pode ser
intrepretado de várias formas, algumas delas com reflexos até na
confiança do público e na própria sobrevivência do órgão. Do nosso ponto
de vista, como técnicos da área de metodologia do IBGE, enfatizamos um
aspecto técnico particular, mas nem por isso menos importante para os
usuários dos dados.

A revolução da informática com a resultante facilidade de acesso ao
computador, criou condições extremamente favoráveis à utilização de
dados estatísticos, produzidos por órgãos como o IBGE. Algumas vezes
esses dados são utilizados para fins puramente descritivos. Outras
vezes, porém, sua utilização é feita para fins analíticos, envolvendo a
construção de modelos, quando o objetivo é extrair conclusões aplicáveis
também a populações distintas daquela da qual se extraiu a amostra.
Neste caso, é comum empregar, sem grandes preocupações, pacotes
computacionais padrões disponíveis para a seleção e ajuste de modelos. é
neste ponto que entra a nossa preocupação com o uso adequado dos dados
produzidos pelo IBGE.

O que torna tais dados especiais para quem pretende usá-los para fins
analíticos? Esta é a questão básica que será amplamente discutida ao
longo deste texto. A mensagem principal que pretendemos transmitir é que
certos cuidados precisam ser tomados para utilização correta dos dados
de pesquisas amostrais como as que o IBGE realiza.

O que torna especiais dados como os produzidos pelo IBGE é que estes são
obtidos através de pesquisas amostrais complexas de populações finitas
que envolvem: \textbf{probabilidades distintas de seleção,
estratificação e conglomeração das unidades, ajustes para compensar
não-resposta e outros ajustes}. Os pacotes tradicionais de análise
ignoram estes aspectos, podendo produzir estimativas incorretas tanto
dos parâmetros como para as variâncias destas estimativas. Quando
utilizamos a amostra para estudos analíticos, as opções disponíveis nos
pacotes estatísticos usuais para levar em conta os pesos distintos das
observações são apropriadas somente para observações independentes e
identicamente distribuídas (IID). Além disso, a variabilidade dos pesos
produz impactos tanto na estimação pontual quanto na estimação das
variâncias dessas estimativas, que sofre ainda influência da
estratificação e conglomeração.

O objetivo deste livro é analisar o impacto das simplificações feitas ao
utilizar procedimentos e pacotes usuais de análise de dados, e
apresentar os ajustes necessários desses procedimentos de modo a
incorporar na análise, de forma apropriada, os aspectos aqui
ressaltados. Para isto serão apresentados exemplos de análises de dados
obtidos em pesquisas amostrais complexas, usando pacotes clássicos e
também pacotes estatísticos especializados. A comparação dos resultados
das análises feitas das duas formas permitirá avaliar o impacto de
ignorar o plano amostral na análise dos dados resultantes de pesquisas
amostrais complexas.

\section*{Agradecimentos}\label{agradecimentos}
\addcontentsline{toc}{section}{Agradecimentos}

A elaboração de um texto como esse não se faz sem a colaboração de
muitas pessoas. Em primeiro lugar, agradecemos à Comissão Organizadora
do SINAPE por ter propiciado a oportunidade ao selecionar nossa proposta
de minicurso. Agradecemos também ao IBGE por ter proporcionado as
condições e os meios usados para a produção da monografia, bem como o
acesso aos dados detalhados e identificados que utilizamos em vários
exemplos.

No plano pessoal, agradecemos a Zélia Bianchini pela revisão do
manuscrito e sugestões que o aprimoraram. Agradecemos a Marcos Paulo de
Freitas e Renata Duarte pela ajuda com a computação de vários exemplos.
Agradecemos a Waldecir Bianchini, Luiz Pessoa e Marinho Persiano pela
colaboração na utilização do processador de textos. Aos demais colegas
do Departamento de Metodologia do IBGE, agradecemos o companheirismo e
solidariedade nesses meses de trabalho na preparação do manuscrito.

Finalmente, agradecemos a nossas famílias pela aceitação resignada de
nossas ausências e pelo incentivo à conclusão da empreitada.

\chapter{Introdução}\label{introduc}

\section{Motivação}\label{motivacao}

\section{Objetivos do Livro}\label{objetivos-do-livro}

\section{Laboratório de R do Capítulo 1.}\label{epa}

\section{total SE DEff}\label{total-se-deff}

\section{analf1 1174220 127982
2.0543}\label{analf1-1174220-127982-2.0543}

\section{total SE DEff}\label{total-se-deff-1}

\section{analf2 4792344 318877
3.3237}\label{analf2-4792344-318877-3.3237}

\section{Ratio estimator:
svyratio.survey.design2(\textasciitilde{}analf1,
\textasciitilde{}faixa1,
ppv.se.des)}\label{ratio-estimator-svyratio.survey.design2analf1-faixa1-ppv.se.des}

\section{Ratios=}\label{ratios}

\section{faixa1}\label{faixa1}

\section{analf1 0.118689}\label{analf1-0.118689}

\section{SEs=}\label{ses}

\section{faixa1}\label{faixa1-1}

\section{analf1 0.01178896}\label{analf1-0.01178896}

\section{Ratio estimator:
svyratio.survey.design2(\textasciitilde{}analf2,
\textasciitilde{}faixa2,
ppv.se.des)}\label{ratio-estimator-svyratio.survey.design2analf2-faixa2-ppv.se.des}

\section{Ratios=}\label{ratios-1}

\section{faixa2}\label{faixa2}

\section{analf2 0.1086871}\label{analf2-0.1086871}

\section{SEs=}\label{ses-1}

\section{faixa2}\label{faixa2-1}

\section{analf2 0.006732254}\label{analf2-0.006732254}

\section{regiao analf1 se
DEff.analf1}\label{regiao-analf1-se-deff.analf1}

\section{1 1 3512866 352619.5 9.660561}\label{section}

\section{2 2 1174220 127982.2 2.054345}\label{section-1}

\subsection{Estimativa do efeito de plano amostral
(EPA)}\label{estimativa-do-efeito-de-plano-amostral-epa}

\section{analf1}\label{analf1}

\section{analf1 2.054049}\label{analf1-2.054049}

\section{analf2}\label{analf2}

\section{analf2 3.32324}\label{analf2-3.32324}

\section{total SE DEff}\label{total-se-deff-2}

\section{{[}1,{]} 1174220 127982 2.6426}\label{section-2}

\section{total SE DEff}\label{total-se-deff-3}

\section{{[}1,{]} 4792344 318877 4.1667}\label{section-3}

\section{total SE DEff}\label{total-se-deff-4}

\section{I(ifelse(regiao == 2, analf1, 0)) 1174220 127982
2.6426}\label{iifelseregiao-2-analf1-0-1174220-127982-2.6426}

\section{total SE DEff}\label{total-se-deff-5}

\section{I(ifelse(regiao == 2, analf2, 0)) 4792344 318877
4.1667}\label{iifelseregiao-2-analf2-0-4792344-318877-4.1667}

\section{Taxa Erro\_Padrao CV}\label{taxa-erro_padrao-cv}

\section{1 11.80303 0.1174791 0.9953299}\label{section-4}

\section{{[}1{]} 60202 1019}\label{section-5}

\section{diag\_dep}\label{diag_dep}

\section{7.6}\label{section-6}

\section{diag\_dep}\label{diag_dep-1}

\section{diag\_dep 0.2}\label{diag_dep-0.2}

\section{{[}1{]} 7.2 8.1}\label{section-7}

\section{diag\_dep}\label{diag_dep-2}

\section{7.6 7.2 8.1}\label{section-8}

\chapter{Referencial para Inferência}\label{refinf}

\section{Modelagem - Primeiras Idéias}\label{classic}

\subsection{Abordagem 1 - Modelagem
Clássica}\label{abordagem-1---modelagem-classica}

\subsection{Abordagem 2 - Amostragem
Probabilística}\label{abordagem-2---amostragem-probabilistica}

\subsection{Discussão das Abordagens 1 e
2}\label{discussao-das-abordagens-1-e-2}

\subsection{Abordagem 3 - Modelagem de
Superpopulação}\label{modelsuperpop}

\section{Fontes de Variação}\label{fontes-de-variacao}

\section{Modelos de Superpopulação}\label{modelos-de-superpopulacao}

\section{Planejamento Amostral}\label{planamo}

\section{Planos Amostrais Informativos e Ignoráveis}\label{inform}

\chapter{Estimação Baseada no Plano Amostral}\label{capplanamo}

\section{Estimação de Totais}\label{estimatotais}

\section{Por que Estimar Variâncias}\label{por-que-estimar-variancias}

\section{Linearização de Taylor para Estimar variâncias}\label{taylor}

\section{Método do Conglomerado
Primário}\label{metodo-do-conglomerado-primario}

\section{Métodos de Replicação}\label{metodos-de-replicacao}

\section{Laboratório de R}\label{laboratorio-de-r}

\section{faixa}\label{faixa}

\section{0.01178896}\label{section-9}

\section{Ratio estimator:
svyratio.survey.design2(\textasciitilde{}analf.faixa,
\textasciitilde{}faixa,
ppv.se.des)}\label{ratio-estimator-svyratio.survey.design2analf.faixa-faixa-ppv.se.des}

\section{Ratios=}\label{ratios-2}

\section{faixa}\label{faixa-1}

\section{analf.faixa 0.118689}\label{analf.faixa-0.118689}

\section{SEs=}\label{ses-2}

\section{faixa}\label{faixa-2}

\section{analf.faixa 0.01178896}\label{analf.faixa-0.01178896}

\section{Ratio estimator:
svyratio.svyrep.design(\textasciitilde{}analf.faixa,
\textasciitilde{}faixa,
ppv.se.des.jkn)}\label{ratio-estimator-svyratio.svyrep.designanalf.faixa-faixa-ppv.se.des.jkn}

\section{Ratios=}\label{ratios-3}

\section{faixa}\label{faixa-3}

\section{analf.faixa 0.118689}\label{analf.faixa-0.118689-1}

\section{SEs=}\label{ses-3}

\section{{[},1{]}}\label{section-10}

\section{{[}1,{]} 0.01181434}\label{section-11}

\section{Ratio estimator:
svyratio.svyrep.design(\textasciitilde{}analf.faixa,
\textasciitilde{}faixa,
ppv.se.des.boot)}\label{ratio-estimator-svyratio.svyrep.designanalf.faixa-faixa-ppv.se.des.boot}

\section{Ratios=}\label{ratios-4}

\section{faixa}\label{faixa-4}

\section{analf.faixa 0.118689}\label{analf.faixa-0.118689-2}

\section{SEs=}\label{ses-4}

\section{{[},1{]}}\label{section-12}

\section{{[}1,{]} 0.01313725}\label{section-13}

\section{\texorpdfstring{{[}1{]}
``svyrep.design''}{{[}1{]} svyrep.design}}\label{svyrep.design}

\section{\texorpdfstring{{[}1{]} ``repweights'' ``pweights''
``type''}{{[}1{]} repweights pweights type}}\label{repweights-pweights-type}

\section{\texorpdfstring{{[}4{]} ``rho'' ``scale''
``rscales''}{{[}4{]} rho scale rscales}}\label{rho-scale-rscales}

\section{\texorpdfstring{{[}7{]} ``call'' ``combined.weights''
``selfrep''}{{[}7{]} call combined.weights selfrep}}\label{call-combined.weights-selfrep}

\section{\texorpdfstring{{[}10{]} ``mse'' ``variables''
``degf''}{{[}10{]} mse variables degf}}\label{mse-variables-degf}

\section{{[}1{]} 8903}\label{section-14}

\section{{[}1{]} 276}\label{section-15}

\section{num {[}1:8903, 1:276{]} 0 0 1.06 1.06 1.06
\ldots{}}\label{num-18903-1276-0-0-1.06-1.06-1.06}

\section{{[}1{]} 0.01181}\label{section-16}

\section{theta SE}\label{theta-se}

\section{{[}1,{]} 0.11869 0.0118}\label{section-17}

\section{theta SE}\label{theta-se-1}

\section{{[}1,{]} 0.50623 0.0494}\label{section-18}

\chapter{Efeitos do Plano Amostral}\label{epa}

\section{Introdução}\label{introducao}

\section{Efeito do Plano Amostral (EPA) de
Kish}\label{efeito-do-plano-amostral-epa-de-kish}

\section{Efeito do Plano Amostral
Ampliado}\label{efeito-do-plano-amostral-ampliado}

\section{Intervalos de Confiança e Testes de
Hipóteses}\label{intervalos-de-confianca-e-testes-de-hipoteses}

\section{Efeitos Multivariados de Plano
Amostral}\label{efeitos-multivariados-de-plano-amostral}

\section{Laboratório de R}\label{laboratorio-de-r-1}

\section{\texorpdfstring{Warning: package `survey' was built under R
version
3.4.1}{Warning: package survey was built under R version 3.4.1}}\label{warning-package-survey-was-built-under-r-version-3.4.1}

\section{Carregando pacotes exigidos:
grid}\label{carregando-pacotes-exigidos-grid}

\section{Carregando pacotes exigidos:
methods}\label{carregando-pacotes-exigidos-methods}

\section{Carregando pacotes exigidos:
Matrix}\label{carregando-pacotes-exigidos-matrix}

\section{Carregando pacotes exigidos:
survival}\label{carregando-pacotes-exigidos-survival}

\section{}\label{section-19}

\section{\texorpdfstring{Attaching package:
`survey'}{Attaching package: survey}}\label{attaching-package-survey}

\section{\texorpdfstring{The following object is masked from
`package:graphics':}{The following object is masked from package:graphics:}}\label{the-following-object-is-masked-from-packagegraphics}

\section{}\label{section-20}

\section{dotchart}\label{dotchart}

\section{{[}1{]} 0.1402485}\label{section-21}

\section{{[}1{]} 0.2827575}\label{section-22}

\chapter{Ajuste de Modelos Paramétricos}\label{ajmodpar}

\section{Introdução}\label{modpar1}

\section{Método de Máxima Verossimilhança
(MV)}\label{metodo-de-maxima-verossimilhanca-mv}

\section{Ponderação de Dados
Amostrais}\label{ponderacao-de-dados-amostrais}

\section{Método de Máxima Pseudo-Verossimilhança}\label{modpar3}

\section{Robustez do Procedimento
MPV}\label{robustez-do-procedimento-mpv}

\section{Desvantagens da Inferência de
Aleatorização}\label{desvantagens-da-inferencia-de-aleatorizacao}

\section{Laboratório de R}\label{laboratorio-de-r-2}

\chapter{Modelos de Regressão}\label{modreg}

\section{Modelo de Regressão Linear Normal}\label{modlinear}

\subsection{Especificação do Modelo}\label{especificacao-do-modelo}

\subsection{Pseudo-parâmetros do
Modelo}\label{pseudo-parametros-do-modelo}

\subsection{Estimadores de MPV dos Parâmetros do
Modelo}\label{estimadores-de-mpv-dos-parametros-do-modelo}

\subsection{Estimação da Variância de Estimadores de
MPV}\label{estimacao-da-variancia-de-estimadores-de-mpv}

\section{Modelo de Regressão Logística}\label{modlogist}

\section{Warning in 1.96 * sqrt(var\_raz112) * c(-1, 1): Recycling array
of length 1 in array-vector arithmetic is
deprecated.}\label{warning-in-1.96-sqrtvar_raz112-c-1-1-recycling-array-of-length-1-in-array-vector-arithmetic-is-deprecated.}

\section{Use c() or as.vector()
instead.}\label{use-c-or-as.vector-instead.}

\section{Warning in 1.96 * sqrt(var\_raz123) * c(-1, 1): Recycling array
of length 1 in array-vector arithmetic is
deprecated.}\label{warning-in-1.96-sqrtvar_raz123-c-1-1-recycling-array-of-length-1-in-array-vector-arithmetic-is-deprecated.}

\section{Use c() or as.vector()
instead.}\label{use-c-or-as.vector-instead.-1}

\section{Warning in 1.96 * sqrt(var\_raz212) * c(-1, 1): Recycling array
of length 1 in array-vector arithmetic is
deprecated.}\label{warning-in-1.96-sqrtvar_raz212-c-1-1-recycling-array-of-length-1-in-array-vector-arithmetic-is-deprecated.}

\section{Use c() or as.vector()
instead.}\label{use-c-or-as.vector-instead.-2}

\section{Warning in 1.96 * sqrt(var\_raz223) * c(-1, 1): Recycling array
of length 1 in array-vector arithmetic is
deprecated.}\label{warning-in-1.96-sqrtvar_raz223-c-1-1-recycling-array-of-length-1-in-array-vector-arithmetic-is-deprecated.}

\section{Use c() or as.vector()
instead.}\label{use-c-or-as.vector-instead.-3}

\section{Warning in 1.96 * sqrt(var\_raz312) * c(-1, 1): Recycling array
of length 1 in array-vector arithmetic is
deprecated.}\label{warning-in-1.96-sqrtvar_raz312-c-1-1-recycling-array-of-length-1-in-array-vector-arithmetic-is-deprecated.}

\section{Use c() or as.vector()
instead.}\label{use-c-or-as.vector-instead.-4}

\section{Warning in 1.96 * sqrt(var\_raz323) * c(-1, 1): Recycling array
of length 1 in array-vector arithmetic is
deprecated.}\label{warning-in-1.96-sqrtvar_raz323-c-1-1-recycling-array-of-length-1-in-array-vector-arithmetic-is-deprecated.}

\section{Use c() or as.vector()
instead.}\label{use-c-or-as.vector-instead.-5}

\section{Teste de Hipóteses}\label{teste-de-hipoteses}

\section{Laboratório de R}\label{laboratorio-de-r-3}

\section{\texorpdfstring{{[}1{]} ``stra'' ``psu'' ``pesopes''
``informal'' ``sx''
``id''}{{[}1{]} stra psu pesopes informal sx id}}\label{stra-psu-pesopes-informal-sx-id}

\section{\texorpdfstring{{[}7{]} ``ae'' ``ht'' ``re''
``um''}{{[}7{]} ae ht re um}}\label{ae-ht-re-um}

\section{stra psu pesopes informal sx id
ae}\label{stra-psu-pesopes-informal-sx-id-ae}

\section{\texorpdfstring{``numeric'' ``numeric'' ``numeric'' ``numeric''
``numeric'' ``numeric''
``numeric''}{numeric numeric numeric numeric numeric numeric numeric}}\label{numeric-numeric-numeric-numeric-numeric-numeric-numeric}

\section{ht re um}\label{ht-re-um}

\section{\texorpdfstring{``numeric'' ``numeric''
``numeric''}{numeric numeric numeric}}\label{numeric-numeric-numeric}

\section{Wald test for ht:re}\label{wald-test-for-htre}

\section{in svyglm(formula = informal \textasciitilde{} sx + ae + ht +
id + re + sx * id
+}\label{in-svyglmformula-informal-sx-ae-ht-id-re-sx-id}

\section{sx * ht + ae * ht + ht * id + ht * re, design =
pnad.des,}\label{sx-ht-ae-ht-ht-id-ht-re-design-pnad.des}

\section{family = binomial)}\label{family-binomial}

\section{F = 6.742662 on 4 and 616 df: p=
2.58e-05}\label{f-6.742662-on-4-and-616-df-p-2.58e-05}

\chapter{Testes de Qualidade de Ajuste}\label{testqualajust}

\section{Introdução}\label{introducao-1}

\section{Teste para uma Proporção}\label{teste-para-uma-proporcao}

\subsection{Correção de Estatísticas
Clássicas}\label{correcao-de-estatisticas-classicas}

\section{{[}1{]} 10}\label{section-23}

\section{{[}1{]} 20}\label{section-24}

\section{{[}1{]} 0.5}\label{section-25}

\section{{[}1{]} 0.4795001}\label{section-26}

\section{{[}1{]} 10.56154}\label{section-27}

\section{{[}1{]} 0.528077}\label{section-28}

\section{{[}1{]} 0.4674165}\label{section-29}

\subsection{Estatística de Wald}\label{estatistica-de-wald}

\section{{[}1{]} 0.5952381}\label{section-30}

\section{{[}1{]} 0.4404007}\label{section-31}

\section{Teste para Várias
Proporções}\label{teste-para-varias-proporcoes}

\subsection{Estatística de Wald Baseada no Plano
Amostral}\label{estatistica-de-wald-baseada-no-plano-amostral}

\subsection{Situações Instáveis}\label{situacoes-instaveis}

\subsection{Estatística de Pearson com Ajuste de
Rao-Scott}\label{raoscott}

\section{{[}1{]} 2.376168}\label{section-32}

\section{{[}1{]} 2.459607}\label{section-33}

\section{{[}1{]} 1.249524}\label{section-34}

\section{{[}1{]} 11.50719}\label{section-35}

\section{{[}1{]} 4.842751}\label{section-36}

\section{{[}1{]} 4.678467}\label{section-37}

\section{{[}1{]} 1.169617}\label{section-38}

\section{{[}1{]} 3.744199}\label{section-39}

\section{{[},1{]}}\label{section-40}

\section{{[}1,{]} 5.742022}\label{section-41}

\section{{[},1{]}}\label{section-42}

\section{{[}1,{]} 1.419005}\label{section-43}

\section{{[},1{]}}\label{section-44}

\section{{[}1,{]} 1.435505}\label{section-45}

\section{Laboratório de R}\label{laboratorio-de-r-4}

\section{{[},1{]}}\label{section-46}

\section{{[}1,{]} 5.742022}\label{section-47}

\section{{[},1{]}}\label{section-48}

\section{{[}1,{]} 0.219}\label{section-49}

\section{{[},1{]}}\label{section-50}

\section{{[}1,{]} 11.50719}\label{section-51}

\section{{[},1{]}}\label{section-52}

\section{{[}1,{]} 0.021}\label{section-53}

\chapter{Testes em Tabelas de Duas Entradas}\label{testetab2}

\section{Introdução}\label{introducao-2}

\section{Tabelas 2x2}\label{tabelas22}

\subsection{Teste de Independência}\label{teste-de-independencia}

\subsection{Teste de Homogeneidade}\label{teste-de-homogeneidade}

\subsection{Efeitos de Plano Amostral nas
Celas}\label{efeitos-de-plano-amostral-nas-celas}

\section{Tabelas de Duas Entradas (Caso
Geral)}\label{tabelas-de-duas-entradas-caso-geral}

\subsection{Teste de Homogeneidade}\label{teste-de-homogeneidade-1}

\subsection{Teste de Independência}\label{teste-de-independencia-1}

\subsection{Estatística de Wald Baseada no Plano
Amostral}\label{estatistica-de-wald-baseada-no-plano-amostral-1}

\subsection{Estatística de Pearson com Ajuste de
Rao-Scott}\label{estatistica-de-pearson-com-ajuste-de-rao-scott}

\section{Laboratório de R}\label{laboratorio-de-r-5}

\section{\texorpdfstring{{[}1{]} ``stra'' ``psu'' ``pesopes''
``informal'' ``sx''
``id''}{{[}1{]} stra psu pesopes informal sx id}}\label{stra-psu-pesopes-informal-sx-id-1}

\section{\texorpdfstring{{[}7{]} ``ae'' ``ht'' ``re''
``um''}{{[}7{]} ae ht re um}}\label{ae-ht-re-um-1}

\section{stra psu pesopes informal sx id
ae}\label{stra-psu-pesopes-informal-sx-id-ae-1}

\section{\texorpdfstring{``numeric'' ``numeric'' ``numeric'' ``numeric''
``numeric'' ``numeric''
``numeric''}{numeric numeric numeric numeric numeric numeric numeric}}\label{numeric-numeric-numeric-numeric-numeric-numeric-numeric-1}

\section{ht re um}\label{ht-re-um-1}

\section{\texorpdfstring{``numeric'' ``numeric''
``numeric''}{numeric numeric numeric}}\label{numeric-numeric-numeric-1}

\section{mean SE}\label{mean-se}

\section{sx1 0.65708 0.0056}\label{sx1-0.65708-0.0056}

\section{sx2 0.34292 0.0056}\label{sx2-0.34292-0.0056}

\section{mean SE}\label{mean-se-1}

\section{re1 0.15546 0.0069}\label{re1-0.15546-0.0069}

\section{re2 0.58356 0.0082}\label{re2-0.58356-0.0082}

\section{re3 0.26098 0.0096}\label{re3-0.26098-0.0096}

\section{mean SE}\label{mean-se-2}

\section{ae1 0.31304 0.0095}\label{ae1-0.31304-0.0095}

\section{ae2 0.31972 0.0071}\label{ae2-0.31972-0.0071}

\section{ae3 0.36725 0.0105}\label{ae3-0.36725-0.0105}

\section{ht1 ht2 ht3}\label{ht1-ht2-ht3}

\section{0.2103714 0.6148881 0.1747405}\label{section-54}

\section{\$names}\label{names}

\section{\texorpdfstring{{[}1{]} ``ht1'' ``ht2''
``ht3''}{{[}1{]} ht1 ht2 ht3}}\label{ht1-ht2-ht3-1}

\section{}\label{section-55}

\section{\$var}\label{var}

\section{ht1 ht2 ht3}\label{ht1-ht2-ht3-2}

\section{ht1 3.666206e-05 -3.322546e-05
-3.436592e-06}\label{ht1-3.666206e-05--3.322546e-05--3.436592e-06}

\section{ht2 -3.322546e-05 6.758652e-05
-3.436106e-05}\label{ht2--3.322546e-05-6.758652e-05--3.436106e-05}

\section{ht3 -3.436592e-06 -3.436106e-05
3.779765e-05}\label{ht3--3.436592e-06--3.436106e-05-3.779765e-05}

\section{}\label{section-56}

\section{\$statistic}\label{statistic}

\section{\texorpdfstring{{[}1{]} ``mean''}{{[}1{]} mean}}\label{mean}

\section{}\label{section-57}

\section{\$class}\label{class}

\section{\texorpdfstring{{[}1{]}
``svystat''}{{[}1{]} svystat}}\label{svystat}

\section{ht1 ht2 ht3}\label{ht1-ht2-ht3-3}

\section{ht1 3.666206e-05 -3.322546e-05
-3.436592e-06}\label{ht1-3.666206e-05--3.322546e-05--3.436592e-06-1}

\section{ht2 -3.322546e-05 6.758652e-05
-3.436106e-05}\label{ht2--3.322546e-05-6.758652e-05--3.436106e-05-1}

\section{ht3 -3.436592e-06 -3.436106e-05
3.779765e-05}\label{ht3--3.436592e-06--3.436106e-05-3.779765e-05-1}

\section{ht1 ht2 ht3}\label{ht1-ht2-ht3-4}

\section{ht1 3.666206e-05 -3.322546e-05
-3.436592e-06}\label{ht1-3.666206e-05--3.322546e-05--3.436592e-06-2}

\section{ht2 -3.322546e-05 6.758652e-05
-3.436106e-05}\label{ht2--3.322546e-05-6.758652e-05--3.436106e-05-2}

\section{ht3 -3.436592e-06 -3.436106e-05
3.779765e-05}\label{ht3--3.436592e-06--3.436106e-05-3.779765e-05-2}

\section{mean SE DEff}\label{mean-se-deff}

\section{re1 0.1554555 0.0068977
2.3611}\label{re1-0.1554555-0.0068977-2.3611}

\section{re2 0.5835630 0.0082001
1.8027}\label{re2-0.5835630-0.0082001-1.8027}

\section{re3 0.2609815 0.0096130
3.1216}\label{re3-0.2609815-0.0096130-3.1216}

\section{re}\label{re}

\section{sx 1 2 3}\label{sx-1-2-3}

\section{1 0.073 0.388 0.196}\label{section-58}

\section{2 0.082 0.195 0.065}\label{section-59}

\section{sx re1 re2 re3 se.re1 se.re2
se.re3}\label{sx-re1-re2-re3-se.re1-se.re2-se.re3}

\section{1 1 0.1110831 0.5908215 0.2980955 0.005726888 0.01025759
0.01112131}\label{section-60}

\section{2 2 0.2404788 0.5696548 0.1898663 0.012502636 0.01193753
0.01114102}\label{section-61}

\section{mean SE DEff}\label{mean-se-deff-1}

\section{I((sx == 1 \& re == 1) * 1) 0.0729904 0.0038343
1.4156}\label{isx-1-re-1-1-0.0729904-0.0038343-1.4156}

\section{re}\label{re-1}

\section{sx 1 2 3}\label{sx-1-2-3-1}

\section{1 0.07299044 0.38821684 0.19587250}\label{section-62}

\section{2 0.08246505 0.19534616 0.06510900}\label{section-63}

\section{re}\label{re-2}

\section{sx 1 2 3}\label{sx-1-2-3-2}

\section{1 7.299044 38.821684 19.587250}\label{section-64}

\section{2 8.246505 19.534616 6.510900}\label{section-65}

\section{{[}1{]} 1.642161}\label{section-66}

\section{}\label{section-67}

\section{Pearson's Chi-squared test}\label{pearsons-chi-squared-test}

\section{}\label{section-68}

\section{data: tab.amo}\label{data-tab.amo}

\section{X-squared = 227.03, df = 2, p-value \textless{}
2.2e-16}\label{x-squared-227.03-df-2-p-value-2.2e-16}

\chapter{Estimação de densidades}\label{estimacao-de-densidades}

\section{Introdução}\label{introducao-3}

\chapter{Modelos Hierárquicos}\label{modelos-hierarquicos}

\section{Introdução}\label{introducao-4}

\chapter{Não-Resposta}\label{nao-resposta}

\section{Introdução}\label{introducao-5}

\chapter{Diagnóstico de ajuste de
modelo}\label{diagnostico-de-ajuste-de-modelo}

\section{Introdução}\label{introducao-6}

\chapter{Agregação vs.~Desagregação}\label{agregdesag}

\section{Introdução}\label{introducao-7}

\section{Modelagem da Estrutura
Populacional}\label{modelagem-da-estrutura-populacional}

\section{Modelos Hierárquicos}\label{modelos-hierarquicos-1}

\section{Análise Desagregada: Prós e
Contras}\label{analise-desagregada-pros-e-contras}

\chapter{Pacotes para Analisar Dados Amostrais}\label{pacotes}

\section{Introdução}\label{introducao-8}

\section{Pacotes Computacionais}\label{pacotes-computacionais}

\chapter{Placeholder}\label{placeholder}

\bibliography{packages,book}


\end{document}
