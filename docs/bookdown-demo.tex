\documentclass[]{book}
\usepackage{lmodern}
\usepackage{amssymb,amsmath}
\usepackage{ifxetex,ifluatex}
\usepackage{fixltx2e} % provides \textsubscript
\ifnum 0\ifxetex 1\fi\ifluatex 1\fi=0 % if pdftex
  \usepackage[T1]{fontenc}
  \usepackage[utf8]{inputenc}
\else % if luatex or xelatex
  \ifxetex
    \usepackage{mathspec}
  \else
    \usepackage{fontspec}
  \fi
  \defaultfontfeatures{Ligatures=TeX,Scale=MatchLowercase}
\fi
% use upquote if available, for straight quotes in verbatim environments
\IfFileExists{upquote.sty}{\usepackage{upquote}}{}
% use microtype if available
\IfFileExists{microtype.sty}{%
\usepackage{microtype}
\UseMicrotypeSet[protrusion]{basicmath} % disable protrusion for tt fonts
}{}
\usepackage[margin=1in]{geometry}
\usepackage{hyperref}
\hypersetup{unicode=true,
            pdftitle={Análise de Dados Amostrais Complexos},
            pdfauthor={Djalma Pessoa e Pedro Nascimento Silva},
            pdfborder={0 0 0},
            breaklinks=true}
\urlstyle{same}  % don't use monospace font for urls
\usepackage{natbib}
\bibliographystyle{apalike}
\usepackage{color}
\usepackage{fancyvrb}
\newcommand{\VerbBar}{|}
\newcommand{\VERB}{\Verb[commandchars=\\\{\}]}
\DefineVerbatimEnvironment{Highlighting}{Verbatim}{commandchars=\\\{\}}
% Add ',fontsize=\small' for more characters per line
\usepackage{framed}
\definecolor{shadecolor}{RGB}{248,248,248}
\newenvironment{Shaded}{\begin{snugshade}}{\end{snugshade}}
\newcommand{\KeywordTok}[1]{\textcolor[rgb]{0.13,0.29,0.53}{\textbf{{#1}}}}
\newcommand{\DataTypeTok}[1]{\textcolor[rgb]{0.13,0.29,0.53}{{#1}}}
\newcommand{\DecValTok}[1]{\textcolor[rgb]{0.00,0.00,0.81}{{#1}}}
\newcommand{\BaseNTok}[1]{\textcolor[rgb]{0.00,0.00,0.81}{{#1}}}
\newcommand{\FloatTok}[1]{\textcolor[rgb]{0.00,0.00,0.81}{{#1}}}
\newcommand{\ConstantTok}[1]{\textcolor[rgb]{0.00,0.00,0.00}{{#1}}}
\newcommand{\CharTok}[1]{\textcolor[rgb]{0.31,0.60,0.02}{{#1}}}
\newcommand{\SpecialCharTok}[1]{\textcolor[rgb]{0.00,0.00,0.00}{{#1}}}
\newcommand{\StringTok}[1]{\textcolor[rgb]{0.31,0.60,0.02}{{#1}}}
\newcommand{\VerbatimStringTok}[1]{\textcolor[rgb]{0.31,0.60,0.02}{{#1}}}
\newcommand{\SpecialStringTok}[1]{\textcolor[rgb]{0.31,0.60,0.02}{{#1}}}
\newcommand{\ImportTok}[1]{{#1}}
\newcommand{\CommentTok}[1]{\textcolor[rgb]{0.56,0.35,0.01}{\textit{{#1}}}}
\newcommand{\DocumentationTok}[1]{\textcolor[rgb]{0.56,0.35,0.01}{\textbf{\textit{{#1}}}}}
\newcommand{\AnnotationTok}[1]{\textcolor[rgb]{0.56,0.35,0.01}{\textbf{\textit{{#1}}}}}
\newcommand{\CommentVarTok}[1]{\textcolor[rgb]{0.56,0.35,0.01}{\textbf{\textit{{#1}}}}}
\newcommand{\OtherTok}[1]{\textcolor[rgb]{0.56,0.35,0.01}{{#1}}}
\newcommand{\FunctionTok}[1]{\textcolor[rgb]{0.00,0.00,0.00}{{#1}}}
\newcommand{\VariableTok}[1]{\textcolor[rgb]{0.00,0.00,0.00}{{#1}}}
\newcommand{\ControlFlowTok}[1]{\textcolor[rgb]{0.13,0.29,0.53}{\textbf{{#1}}}}
\newcommand{\OperatorTok}[1]{\textcolor[rgb]{0.81,0.36,0.00}{\textbf{{#1}}}}
\newcommand{\BuiltInTok}[1]{{#1}}
\newcommand{\ExtensionTok}[1]{{#1}}
\newcommand{\PreprocessorTok}[1]{\textcolor[rgb]{0.56,0.35,0.01}{\textit{{#1}}}}
\newcommand{\AttributeTok}[1]{\textcolor[rgb]{0.77,0.63,0.00}{{#1}}}
\newcommand{\RegionMarkerTok}[1]{{#1}}
\newcommand{\InformationTok}[1]{\textcolor[rgb]{0.56,0.35,0.01}{\textbf{\textit{{#1}}}}}
\newcommand{\WarningTok}[1]{\textcolor[rgb]{0.56,0.35,0.01}{\textbf{\textit{{#1}}}}}
\newcommand{\AlertTok}[1]{\textcolor[rgb]{0.94,0.16,0.16}{{#1}}}
\newcommand{\ErrorTok}[1]{\textcolor[rgb]{0.64,0.00,0.00}{\textbf{{#1}}}}
\newcommand{\NormalTok}[1]{{#1}}
\usepackage{longtable,booktabs}
\usepackage{graphicx,grffile}
\makeatletter
\def\maxwidth{\ifdim\Gin@nat@width>\linewidth\linewidth\else\Gin@nat@width\fi}
\def\maxheight{\ifdim\Gin@nat@height>\textheight\textheight\else\Gin@nat@height\fi}
\makeatother
% Scale images if necessary, so that they will not overflow the page
% margins by default, and it is still possible to overwrite the defaults
% using explicit options in \includegraphics[width, height, ...]{}
\setkeys{Gin}{width=\maxwidth,height=\maxheight,keepaspectratio}
\IfFileExists{parskip.sty}{%
\usepackage{parskip}
}{% else
\setlength{\parindent}{0pt}
\setlength{\parskip}{6pt plus 2pt minus 1pt}
}
\setlength{\emergencystretch}{3em}  % prevent overfull lines
\providecommand{\tightlist}{%
  \setlength{\itemsep}{0pt}\setlength{\parskip}{0pt}}
\setcounter{secnumdepth}{5}
% Redefines (sub)paragraphs to behave more like sections
\ifx\paragraph\undefined\else
\let\oldparagraph\paragraph
\renewcommand{\paragraph}[1]{\oldparagraph{#1}\mbox{}}
\fi
\ifx\subparagraph\undefined\else
\let\oldsubparagraph\subparagraph
\renewcommand{\subparagraph}[1]{\oldsubparagraph{#1}\mbox{}}
\fi

%%% Use protect on footnotes to avoid problems with footnotes in titles
\let\rmarkdownfootnote\footnote%
\def\footnote{\protect\rmarkdownfootnote}

%%% Change title format to be more compact
\usepackage{titling}

% Create subtitle command for use in maketitle
\newcommand{\subtitle}[1]{
  \posttitle{
    \begin{center}\large#1\end{center}
    }
}

\setlength{\droptitle}{-2em}
  \title{Análise de Dados Amostrais Complexos}
  \pretitle{\vspace{\droptitle}\centering\huge}
  \posttitle{\par}
  \author{Djalma Pessoa e Pedro Nascimento Silva}
  \preauthor{\centering\large\emph}
  \postauthor{\par}
  \predate{\centering\large\emph}
  \postdate{\par}
  \date{2017-04-06}

\usepackage{booktabs}
\usepackage[brazil]{babel}
\newtheorem{example}{Exemplo}
\numberwithin{example}{chapter}
\newtheorem{remark}{Observação}
\numberwithin{remark}{chapter}
\newtheorem{definition}{Definição}
\numberwithin{definition}{chapter}

\let\BeginKnitrBlock\begin \let\EndKnitrBlock\end
\begin{document}
\maketitle

{
\setcounter{tocdepth}{1}
\tableofcontents
}
\chapter*{Prefácio}\label{prefacio}
\addcontentsline{toc}{chapter}{Prefácio}

\chapter{Introdução}\label{introduc}

\chapter{Referencial para Inferência}\label{refinf}

\chapter{Estimação Baseada no Plano Amostral}\label{capplanamo}

\chapter{Efeitos do Plano Amostral}\label{epa}

\chapter{Ajuste de Modelos Paramétricos}\label{ajmodpar}

\chapter{Modelos de Regressão}\label{modreg}

\chapter{Testes de Qualidade de Ajuste}\label{testqualajust}

\chapter{Testes em Tabelas de Duas Entradas}\label{testetab2}

\section{Introdução}\label{introducao}

Os principais testes em tabelas de duas entradas são os de homogeneidade
e de independência. O \texttt{teste\ de\ homogeneidade} é apropriado
para estudar a igualdade das distribuições condicionais de uma variável
resposta categórica correspondentes a diferentes níveis de uma variável
preditora também categórica. O \texttt{teste\ de\ independência} é
adequado para estudar a associação entre duas variáveis categóricas.
Enquanto o primeiro teste se refere às distribuições condicionais da
variável resposta para níveis fixados da variável preditora, o segundo
se refere à distribuição conjunta das duas variáveis categóricas que
definem as celas da tabela. Apesar de conceitualmente distintas, as duas
hipóteses podem ser testadas, no caso de amostragem aleatória simples,
utilizando a mesma estatística de teste multinomial de Pearson.

Nos testes de homogeneidade e de independência para tabelas de
frequências \(L\times C\) obtidas por amostragem aleatória simples, a
estatística de teste de Pearson tem distribuição assintótica
qui-quadrado com \((L-1)(C-1)\) graus de liberdade, isto é
\(\chi ^{2}\left( (L-1)(C-1)\right)\). Para pesquisas com planos
amostrais complexos, esta propriedade assintótica padrão não é válida.
Por exemplo, testes definidos em tabelas de frequências obtidas mediante
amostragem por conglomerados são mais liberais (rejeitam mais)
relativamente aos níveis nominais de significância, devido à correlação
intraclasse positiva das variáveis usadas para definir a tabela. Além
disso, para planos amostrais complexos, as estatísticas de teste das
duas hipóteses devem ser corrigidas de formas diferentes.

Neste capítulo, apresentamos versões modificadas de procedimentos
clássicos de testes para dados categóricos, de maneira a incorporar os
efeitos de plano amostral na análise. Procedimentos mais recentes,
baseados em ajustes de modelos regressivos, estão disponíveis em pacotes
especializados como o SUDAAN (procedimento CATAN, para dados tabelados,
e procedimento LOGISTIC, para regressão com respostas individuais
binárias, por exemplo), porém não serão aqui considerados.

\section{Tabelas 2x2}\label{tabelas22}

Para fixar idéias, vamos considerar inicialmente uma tabela de
contingência \(2\times 2\), isto é, com \(L=2\) e \(C=2\), representada
pela Tabela \ref{tab81}. A entrada \(p_{lc}\) na Tabela \ref{tab81}
representa a proporção populacional de unidades no nível \(l\) da
variável 1 e \(c\) da variável 2, ou seja \(p_{lc}=\frac{N_{lc}}{N}\) ,
onde \(N_{lc}\) é o número de observações na cela \(\left( l,c\right)\)
na população, \(N\) é o tamanho da população e
\(\sum\nolimits_{l}\sum\nolimits_{c}p_{lc}=1\). Vamos denotar, ainda, as
proporções marginais na tabela por \(p_{l+}=\sum\nolimits_{c}p_{lc}\) e
\(p_{+c}=\sum_{l}p_{lc}\).

\begin{center}
\begin{table}[tbp] \centering
\caption{Tabela 2x2 de proporções }\bigskip \label{tab81}
\begin{tabular}{|c|c|c|c|}
\hline
& \multicolumn{3}{|c|}{Var 2} \\ \cline{2-4}
Var 1 & 1 & 2 & Total \\ \hline
1 & $p_{11}$ & $p_{12}$ & $p_{1+}$ \\ 
2 & $p_{21}$ & $p_{22}$ & $p_{2+}$ \\ \hline
Total & $p_{+1}$ & $p_{+2}$ & $1$ \\ \hline
\end{tabular}
\end{table}
\end{center}

\subsection{Teste de Independência}\label{teste-de-independencia}

A hipótese de independência corresponde a \[
H_{0}:p_{lc}=p_{l+}p_{+c}\;\;\forall l,c=1,2\;. 
\]

A estatística de teste de Pearson para testar esta hipótese, no caso de
amostragem aleatória simples, é dada por \[
X_{P}^{2}\left( I\right) =n\sum\limits_{l=1}^{2}\sum\limits_{c=1}^{2}\frac{
\left( \hat{p}_{lc}-\hat{p}_{l+}\hat{p}_{+c}\right) ^{2}}{\hat{p}_{l+}\hat{p}
_{+c}} 
\] onde \(\hat{p}_{lc}=n_{lc}/n\) , \(n_{lc}\) é o número de observações
da amostra na cela \(\left( l,c\right)\) da tabela, \(n\) é o tamanho
total da amostra, \(\hat{p}_{l+}=\sum\nolimits_{c}\widehat{p}_{lc}\) e
\(\hat{p}_{+c}=\sum_{l}\hat{p}_{lc}\) .

Sob a hipótese nula, a estatística \(X_{P}^{2}\left( I\right)\) tem
distribuição de referência qui-quadrado com um grau de liberdade.
Observe que esta estatística mede uma distância (em certa escala) entre
os valores observados na amostra e os valores esperados (estimados) sob
a hipótese nula de independência.

\subsection{Teste de Homogeneidade}\label{teste-de-homogeneidade}

No caso do teste de independência, as duas variáveis envolvidas são
consideradas como respostas. No teste de homogeneidade, uma das
variáveis, a variável 2, por exemplo, é considerada a resposta enquanto
a variável 1 é considerada explicativa. Vamos agora analisar a
distribuição da variável 2 (coluna) para cada nível da variável 1
(linha). Considerando ainda uma tabela \(2\times 2,\) queremos testar a
hipótese \[
H_{0}:p_{1c}=p_{2c}\quad c=1,2\;\;. 
\] onde agora \(p_{lc}\) representa a proporção na linha \(l\) de
unidades na coluna \(c\). Com as restrições usuais de que as proporções
nas linhas somam \(1\), isto é, \(p_{11}+p_{12}=p_{21}+p_{22}=1\), a
hipótese nula considerada se reduz a \(p_{11}=p_{21}\) e novamente temos
apenas um grau de liberdade.

Para o teste de homogeneidade, usamos a seguinte estatística de teste de
Pearson: \[
X_{P}^{2}\left( H\right) =\sum\limits_{l=1}^{2}\sum\limits_{c=1}^{2}\frac{
n_{l+}\left( \widehat{p}_{lc}-\hat{p}_{+c}\right) ^{2}}{\hat{p}_{+c}},
\] onde \(n_{l+}=\sum\nolimits_{c}n_{lc}\) para \(l=1,2\) e
\(\widehat{p} _{lc}=n_{lc}/n_{l+}\) para \(l=1,2\) e \(c=1,2\).

Esta estatística mede a distância entre valores observados e esperados
sob a hipótese nula de homogeneidade e tem, também, distribuição de
referência qui-quadrado com um grau de liberdade. Embora as expressões
de \(X_{P}^{2}\left( I\right)\) e \(X_{P}^{2}\left( H\right)\) sejam
distintas, seus valores numéricos são iguais.

\subsection{Efeitos de Plano Amostral nas
Celas}\label{efeitos-de-plano-amostral-nas-celas}

Para relacionar os testes tratados neste capítulo com o teste de
qualidade de ajuste apresentado no capítulo anterior, observe que os
testes de independência e de homogeneidade são definidos sobre o vetor
de proporções de distribuições multinomiais. No caso de independência,
temos uma distribuição multinomial com vetor de probabilidades
\(\left( p_{11},p_{12},p_{21},p_{22}\right) ,\) e no caso do teste de
homogeneidade, temos duas multinomiais (no caso binomiais) com vetores
de probabilidades \(\left( p_{11},p_{12}\right)\) e
\(\left( p_{21},p_{22}\right)\). O processo de contagem que gera estas
multinomiais pressupõe que as observações individuais (indicadores de
classe) são independentes e com mesma distribuição. Estas hipóteses só
são válidas no caso de amostragem aleatória simples com reposição.

Quando os dados são gerados através de um plano amostral complexo,
surgem efeitos de conglomeração e estratificação que devem ser
considerados no cálculo das estatísticas de teste. Neste caso, as
frequências nas celas da tabela são estimadas, levando em conta os pesos
dos elementos da amostra bem como o plano amostral efetivamente
utilizado.

Denotemos por \(\hat{N}_{lc}\) o estimador do número de observações na
cela \(\left( l,c\right)\) na população, e designemos por
\(\hat{n}_{lc}=\) \(\left( \hat{N}_{lc}/\hat{N}\right) \times n\) o
valor padronizado de \(\hat{N}_{lc}\), de modo que
\(\sum\limits_{l=1}^{L}\sum\limits_{c=1}^{C}\hat{n}_{lc}=n\). Sejam,
agora, os estimadores das proporções nas celas dados por
\(\hat{p}_{lc}=\hat{n}_{lc}/n\) no caso do teste de independência e por
\(\hat{p}_{lc}=\hat{n}_{lc}/n_{l+}\) no caso do teste de homogeneidade.
As estatísticas \(X_{P}^{2}\left( I\right)\) e
\(X_{P}^{2}\left(H\right)\) calculadas com as estimativas
\(\hat{n}_{lc}\) no lugar dos valores \(n_{lc}\) não têm, como antes,
distribuição assintótica qui-quadrado com um grau de liberdade.

Por outro lado, é importante observar que as agências produtoras de
dados estatísticos geralmente apresentam os resultados de suas pesquisas
em tabelas contendo as estimativas \(\hat{N}_{lc}\), como ilustrado no
Exemplo \ref{ex:Analisec} do Capítulo \ref{ajmodpar}. Se calcularmos as
estatísticas \(X_{P}^{2}\left( I\right)\) e \(X_{P}^{2}\left( H\right)\)
a partir dos valores dos \(\hat{N}_{lc}\) fornecidos, com a estimativa
do tamanho da população \(\hat{N}\) no lugar de \(n\), os resultados
assintóticos obtidos para amostragem aleatória simples com reposição
(IID) deixarão de ser válidos. Devemos calcular as estatísticas de teste
\(X_{P}^{2}\left( I\right)\) e \(X_{P}^{2}\left( H\right)\) a partir dos
\(\hat{n}_{lc}\) anteriormente definidos, que correspondem aos
\(\hat{N}_{lc}\) padronizados para totalizar \(n\).

As estatísticas baseadas nos valores estimados \(\hat{n}_{lc}\) podem
ser corrigidas para ter distribuição de referência qui-quadrado com um
grau de liberdade, no caso de tabela \(2\times 2\). Mas, é importante
observar que os efeitos de plano amostral e as correções a serem
considerados são distintos para as duas estatísticas
\(X_{P}^{2}\left( I\right)\) e \(X_{P}^{2}\left(H\right)\).

Para ilustrar esse ponto vamos considerar o \emph{ajuste de EPA médio},
que será apresentado na próxima seção para o caso de tabelas
\(L\times C\) . Este ajuste, no caso da estatística
\(X_{P}^{2}\left( I\right)\), se baseia no EPA médio das estimativas das
proporções nas celas \(\hat{p}_{lc}=\hat{n}_{lc}/n\), enquanto que para
a estatística \(X_{P}^{2}\left( H\right)\) ele se baseia no EPA médio
das estimativas das proporções nas linhas
\(\hat{p}_{lc}=\hat{n} _{lc}/n_{l+}\).

Os valores das estatísticas \(X_{P}^{2}\left( I\right)\) e
\(X_{P}^{2}\left( H\right)\) são iguais no caso IID, mas para planos
amostrais complexos, as estatísticas corrigidas pelo EPA médio são
distintas, apesar de terem, para tabelas \(2\times 2\), a mesma
distribuição de referência qui- quadrado com um grau de liberdade.
Adiante apresentaremos um exemplo numérico para ilustrar este ponto.

\section{Tabelas de Duas Entradas (Caso
Geral)}\label{tabelas-de-duas-entradas-caso-geral}

\subsection{Teste de Homogeneidade}\label{teste-de-homogeneidade-1}

O teste de homogeneidade pode ser usado para comparar distribuições de
uma variável categórica (\(C\) categorias) para um conjunto de \(L\)
regiões não superpostas, a partir de amostras independentes obtidas
através de um plano amostral com vários estágios. Vamos considerar uma
tabela \(L\times C\) e supor que as colunas da tabela correspondem às
classes da variável resposta e as linhas correspondem às regiões, de
modo que as somas da proporções nas linhas na tabela de proporções são
iguais a \(1\). A tabela para a população é da forma da Tabela
\ref{tab82}.

\begin{center}
\begin{table}[tbp] \centering
\caption{Proporções de linhas em tabela $L\times C$}
\bigskip \label{tab82}
\begin{tabular}{|c|cccccc|c|}
\hline\hline
Região & $1$ & $2$ & $\ldots $ & $c$ & $\ldots $ & $C$ & Total \\
\hline\hline
$1$ & $p_{11}$ & $p_{12}$ & $\ldots $ & $p_{1c}$ & $\ldots $ & $p_{1C}$ & $1$
\\
$2$ & $p_{21}$ & $p_{22}$ & $\ldots $ & $p_{2c}$ & $\ldots $ & $p_{2C}$ & $1$
\\
$\vdots $ & $\vdots $ & $\vdots $ & . & $\vdots $ & . & $\vdots $ & $\vdots $
\\
$l$ & $p_{l1}$ & $p_{l1}$ & $\ldots $ & $p_{lc}$ & $\ldots $ & $p_{lC}$ & $1$
\\
$\vdots $ & $\vdots $ & $\vdots $ & . & $\vdots $ & . & $\vdots $ & $\vdots $
\\
$L$ & $p_{L1}$ & $p_{L2}$ & $\ldots $ & $p_{Lc}$ & $\ldots $ & $p_{LC}$ & $1$
\\ \hline\hline
\end{tabular}
\end{table}
\end{center}

Note que aqui as proporções que aparecem nas linhas da tabela são
proporções calculadas em relação à freqüência total da linha, e não
proporções calculadas em relação ao total da tabela como na seção
anterior. Portanto, \(p_{lc}=N_{lc}/N_{l+}\) para todo \(l=1,\ldots ,L\)
e \(c=1,\ldots ,C\).

Vamos considerar o caso em que \(L=2\) regiões devem ser comparadas.
Seja
\(\mathbf{p}_{l}=\left( p_{l1},\ldots ,p_{l\;C-1}\right) ^{\prime }\) o
vetor de proporções da \(l\)-ésima região, sem incluir a proporção
referente à última categoria (\(p_{lC}\)), \(l=1,2.\) A hipótese de
igualdade das distribuições da resposta nas duas regiões pode ser
expressa como \(H_{0}:\mathbf{p}_{1}=\mathbf{p}_{2}\) , com \(C-1\)
componentes em cada vetor, pois em cada região a soma das proporções é
1.

Seja
\(\mathbf{p}_{0}=\left( p_{+1},\ldots p_{+\;C-1}\right) ^{\prime }\) o
vetor comum de proporções sob \(H_{0}\), desconhecido. Denotemos por
\(\mathbf{\hat{p}}_{l}=\left( \hat{p}_{l1},\ldots ,\hat{p} _{l\;C-1}\right) ^{\prime }\)
os vetores de proporções estimadas (\(l=1,2\)), baseados em amostras
independentes para as diferentes regiões, onde
\(\hat{p}_{lc}=\widehat{N}_{lc}/\widehat{N}_{l+}\) é um estimador
consistente da proporção \(p_{lc}\) na população correspondente, e
\(\widehat{N}_{lc}\) e \(\widehat{N}_{l+}\) são estimadores ponderados
das frequências nas celas e nas marginais de linha da tabela,
respectivamente, de modo que
\(\sum\nolimits_{c=1}^{C} \widehat{N}_{lc}=\widehat{N}_{l+}\) . Estes
estimadores levam em consideração as probabilidades desiguais de
inclusão na amostra e os ajustes por não-resposta. Observe que, se os
tamanhos das amostras dos subgrupos regionais não forem fixados, os
\(\hat{p}_{lc}\) são estimadores de razão.

Sejam \(\mathbf{\hat{V}}_{p}\left( \widehat{\mathbf{p}}_{1}\right)\) e
\(\mathbf{\hat{V}}_{p}\left( \widehat{\mathbf{p}}_{2}\right)\)
estimadores consistentes das matrizes de variância de aleatorização dos
vetores \(\widehat{\mathbf{p}}_{1}\) e \(\widehat{\mathbf{p}}_{2}\) ,
respectivamente. A estatística de Wald baseada no plano amostral
\(X_{W}^{2}\left( H\right)\) para efetuar o teste de homogeneidade no
caso de duas regiões \(\left( L=2\right)\) é dada por

\begin{equation}
X_{W}^{2}\left( H\right) =\left( \mathbf{\hat{p}}_{1}-\mathbf{\hat{p}}
_{2}\right) ^{^{\prime }}\left[ \mathbf{\hat{V}}_{p}\left( \widehat{\mathbf{p
}}_{1}\right) +\mathbf{\hat{V}}_{p}\left( \widehat{\mathbf{p}}_{2}\right)
\right] ^{-1}\left( \mathbf{\hat{p}}_{1}-\mathbf{\hat{p}}_{2}\right) ,
\label{eq:Tab1}
\end{equation}

pois as amostras são disjuntas e supostas independentes.

No caso, a estatística de Wald \(X_{W}^{2}\left( H\right)\) tem
distribuição assintótica qui-quadrado com
\(\left( 2-1\right) \times \left( C-1\right)\) graus de liberdade.
Quando o número de unidades primárias de amostragem na amostra de cada
região é grande, a estatística de Wald funciona adequadamente. Caso
contrário, ocorre problema de instabilidade e usamos, alternativamente,
uma estatística F-corrigida de Wald. Freitas et al.(1997) descrevem uma
aplicação da estatística \(X_{W}^{2}\left( H\right)\) para testar a
hipótese de igualdade das pirâmides etárias estimadas pela Pesquisa
sobre Padrões de Vida 96/97 (PPV) e da Pesquisa Nacional por Amostra de
Domicílios 95 para as regiões Sudeste e Nordeste. Tal compara\c{c
}ão fez parte do processo de avaliação da qualidade dos resultados da
PPV.

Designemos por \(f=m-H\) o número total de graus de liberdade disponível
para estimar
\(\left[ \mathbf{\hat{V}}_{p}\left( \widehat{ \mathbf{p}}_{1}\right) +\mathbf{\hat{V}}_{p}\left( \widehat{\mathbf{p}} _{2}\right) \right] ,\)
onde \(m\) e \(H\) são os números totais de conglomerados e de estratos
nas amostras das duas regiões, respectivamente. As correções F da
estatística \(X_{W}^{2}\left( H\right)\) são dadas por

\begin{equation}
F_{1.p}=\frac{f-\left( C-1\right) +1}{f\left( C-1\right) }X_{W}^{2}\left(
H\right) , \label{eq:Tab2} 
\end{equation}

que tem distribuição de referência \(F\) com \(\left(C-1\right)\) e
\(\left( f-\left( C-1\right)+1\right)\) graus de liberdade e, ainda,

\begin{equation}
F_{2.p}=X_{W}^{2}\left( H\right) /\left(C-1\right) \label{eq:Tab3}
\end{equation}

que tem distribuição de referência \emph{F} com \(\left(C-1\right)\) e
\(f\) graus de liberdade.

As estatísticas \(F_{1.p}\) e \(F_{2.p}\) podem amenizar o efeito de
instabilidade, quando \(f\) não é grande relativamente ao número de
classes (\(C\)) da variável resposta.

No caso de \(L=2\) regiões, a estatística de teste de homogeneidade de
Pearson é dada por

\begin{equation}
X_{P}^{2}\left( H\right) =\left( \mathbf{\hat{p}}_{1}-\mathbf{\hat{p}}
_{2}\right) ^{^{\prime }}\left( \mathbf{\hat{P}}/\widehat{n}_{1+}+\mathbf{
\hat{P}}/\widehat{n}_{2+}\right) ^{-1}\left( \mathbf{\hat{p}}_{1}-\mathbf{
\hat{p}}_{2}\right) \;\;,  \label{eq:Tab4}
\end{equation}

onde
\(\mathbf{\hat{P}=diag}\left( \mathbf{\hat{p}}_{0}\right) -\mathbf{\hat{p }}_{0}\mathbf{\hat{p}}_{0}^{^{\prime }}\)
e \(\mathbf{\hat{p}}_{0}\) é o estimador do vetor comum de proporções
sob a hipótese de homogeneidade.

Neste caso, \(\mathbf{\hat{P}}/\widehat{n}_{1+}\) é o estimador da
matriz de covariância de \(\mathbf{\hat{p}}_{0}\) na primeira região e
\(\mathbf{\hat{P}}/\widehat{n}_{2+}\) na segunda. Observe que
\eqref{eq:Tab4} e \eqref{eq:Tab1} têm a mesma forma, diferindo só no
estimador da matriz de covariância usado para definir a métrica de
distância. No caso da estatística \(X_{P}^{2}\left( H\right)\), o
estimador da matriz de covariância baseia-se nas hipóteses relativas à
distribuição multinomial, apropriadas para a amostragem aleatória
simples. A distribuição de referência da estatística
\(X_{P}^{2}\left( H\right)\) é qui-quadrado com \(\left( C-1\right)\)
graus de liberdade.

Para introduzir em \(X_{P}^{2}\left( H\right)\) o ajuste de EPA médio e
o ajuste de Rao-Scott de primeira ordem, é preciso calcular estimativas
de efeitos de plano amostral das estimativas das proporções nas linhas
em ambas as regiões. O ajuste de segunda ordem de Rao-Scott, por sua
vez, depende da matriz de efeito multivariado do plano amostral. As
estimativas de efeitos de plano amostral na região \(l\) são da forma

\begin{equation}
\hat{d}_{lc}=\widehat{n}_{l+}\hat{V}_{lc}/\left( \hat{p}_{+c}\left( 1-\hat{p}
_{+c}\right) \right) ,\ \;l=1,2\mbox{ e }c=1,\ldots ,C,  \label{eq:Tab5}
\end{equation}

onde \(\hat{V}_{lc}\) é o \(c\)-ésimo elemento da diagonal de
\(\mathbf{ \hat{V}}_{p}\left( \widehat{\mathbf{p}}_{l}\right)\).

A matriz estimada de efeito multivariado de plano amostral é

\begin{equation}
\mathbf{\hat{\Delta}=}\frac{\widehat{n}_{1+}\times \widehat{n}_{2+}}{
\widehat{n}_{1+}+\widehat{n}_{2+}}\mathbf{\hat{P}}^{-1}\left( \mathbf{\hat{V}
}_{p}\left( \widehat{\mathbf{p}}_{1}\right) +\mathbf{\hat{V}}_{p}\left( 
\widehat{\mathbf{p}}_{2}\right) \right) \;\;.  \label{eq:Tab6}
\end{equation}

A estatística de Pearson com ajuste de EPA médio é dada por

\begin{equation}
X_{P}^{2}\left( H;\hat{d}_{\cdot }\right) =X_{P}^{2}\left( H\right) /\hat{d}
_{\cdot },  \label{eq:Tab7}
\end{equation}

onde
\(\hat{d}_{\cdot }=\sum\limits_{l=1}^{2}\sum\limits_{c=1}^{C}\hat{d} _{lc}/2C\)
é a média das estimativas dos efeitos univariados de plano amostral.

Usando os autovalores \(\hat{\delta}_{c}\) de \(\mathbf{\hat{\Delta}}\),
o ajuste de primeira ordem de Rao-Scott é dado por

\begin{equation}
X_{P}^{2}\left( H;\hat{\delta}_{.}\right) =X_{P}^{2}\left( H\right) /\hat{
\delta}_{.},  \label{eq:Tab8}
\end{equation}

onde \[
\hat{\delta}_{.}=\frac{tr\left( \mathbf{\hat{\Delta}}\right) }{\left(
C-1\right) }=\frac{1}{C-1}\sum\limits_{l=1}^{2}\left( 1-\frac{\widehat{n}
_{l+}}{\widehat{n}_{1+}+\widehat{n}_{2+}}\right) \sum\limits_{c=1}^{C}\frac{
\hat{p}_{lc}}{\hat{p}_{+c}}\left( 1-\hat{p}_{lc}\right) \hat{d}_{lc} 
\] é um estimador da média \(\bar{\delta}\) dos autovalores
\(\delta _{c}\) da matriz \(\mathbf{\Delta }\), desconhecida, de efeito
multivariado do plano amostral. Como a soma dos autovalores de
\(\mathbf{\hat{\Delta}}\) é igual ao traço de \(\mathbf{\hat{\Delta}}\),
esta correção pode ser obtida sem ser necessário calcular os
autovalores.

As distribuições de referência, tanto de
\(X_{P}^{2}\left( H;\hat{ d}_{\cdot }\right)\) como de
\(X_{P}^{2}\left( H;\hat{\delta}_{.}\right)\), são qui-quadrado com
\(\left( C-1\right)\) graus de liberdade. Estes ajustes corrigem a
estatística \(X_{P}^{2}\left( H\right)\) de modo a obter estatísticas
com valor esperado igual ao da distribuição qui-quadrado de referência.
Tal correção é apropriada quando houver pouca variação das estimativas
dos autovalores \(\hat{\delta} _{c}\). Quando isto não ocorrer, pode ser
introduzido o ajuste de segunda ordem de Rao-Scott, que para a
estatística de Pearson é dado por

\begin{equation}
X_{P}^{2}\left( H;\hat{\delta}_{.},\hat{a}^{2}\right) =X_{P}^{2}\left( H;
\hat{\delta}_{.}\right) /\left( 1+\hat{a}^{2}\right)  \label{eq:Tab9}
\end{equation}

onde \(\hat{a}^{2}\) é o quadrado do coeficiente de variação dos
quadrados das estimativas dos autovalores \(\hat{\delta}_{c}\), dado por
\[
\hat{a}^{2}=\sum\limits_{c=1}^{C}\hat{\delta}_{c}^{2}/\left( \left(
C-1\right) \hat{\delta}_{.}^{2}\right) -1,
\] onde a soma dos quadrados dos autovalores pode ser obtida a partir do
traço de \(\mathbf{\hat{\Delta}}^{2}\) \[
\sum\limits_{c=1}^{C}\hat{\delta}_{c}^{2}=tr\left( \mathbf{\hat{\Delta}}
^{2}\right) \;\;. 
\]

A estatística de Pearson com a correção de segunda ordem de Rao-Scott
\(X_{P}^{2}\left( H;\hat{\delta}_{.},\hat{a}^{2}\right)\) tem
distribuição de referência qui-quadrado com graus de liberdade com
ajuste de Satterhwaite
\(gl_{S}=\left( C-1\right) /\left( 1+\hat{a} ^{2}\right)\).

Quando as estimativas
\(\mathbf{\hat{V}}_{p}\left( \widehat{\mathbf{p}} _{1}\right)\) e
\(\mathbf{\hat{V}}_{p}\left( \widehat{\mathbf{p}}_{2}\right)\) das
matrizes de covariâncias regionais são baseadas em números relativamente
pequenos de unidades primárias de amostragem selecionadas, pode-se usar
a estatística F-corrigida de Pearson. Ela é dada, no caso de duas
regiões, por \[
FX_{P}^{2}\left( H;\hat{\delta}_{.}\right) =X_{P}^{2}\left( H;\hat{\delta}
_{.}\right) /\left( C-1\right),
\] e tem distribuição de referência \emph{F} com \(\left(C-1\right)\) e
\(f\) graus de liberdade.

\subsection{Teste de Independência}\label{teste-de-independencia-1}

Vamos considerar o teste de independência no caso geral de tabela
\(L\times C\), onde os dados são extraídos de uma única população, sem
fixar marginais. Consideremos a Tabela \ref{tab83} com as proporções nas
celas a nível da população, onde agora novamente se tem
\(p_{lc}=N_{lc}/N\).

\begin{center}
\begin{table}[tbp] \centering
\caption{Proporções por cela na população}\bigskip \label{tab83}
\begin{tabular}{|c|cccccc|c|}
\hline\hline
& \multicolumn{7}{|c|}{Variável 2} \\ \cline{2-8}
Variável 1 & $1$ & $2$ & $\ldots $ & $c$ & $\ldots $ & $C$ & Total \\
\hline\hline
$1$ & $p_{11}$ & $p_{12}$ & $\ldots $ & $p_{1c}$ & $\ldots $ & $p_{1C}$ & $
p_{1+}$ \\
$2$ & $p_{21}$ & $p_{22}$ & $\ldots $ & $p_{2c}$ & $\ldots $ & $p_{2C}$ & $
p_{2+}$ \\
$\vdots $ & $\vdots $ & $\vdots $ & . & $\vdots $ & . & $\vdots $ & $\vdots $
\\
$l$ & $p_{l1}$ & $p_{l1}$ & $\ldots $ & $p_{lc}$ & $\ldots $ & $p_{lC}$ & $
p_{l+}$ \\
$\vdots $ & $\vdots $ & $\vdots $ & . & $\vdots $ & . & $\vdots $ & $\vdots $
\\
$L$ & $p_{L1}$ & $p_{L2}$ & $\ldots $ & $p_{Lc}$ & $\ldots $ & $p_{LC}$ & $
p_{L+}$ \\ \hline\hline
Total & $p_{+1}$ & $p_{+2}$ & $\ldots $ & $p_{+c}$ & $\ldots $ & $p_{+C}$ & $
1$ \\ \hline\hline
\end{tabular}
\end{table}
\end{center}

Estamos interessados em testar a hipótese de independência \[
H_{0}:p_{lc}=p_{l+}p_{+c},\quad l=1,\ldots ,L-1,\ c=1,\ldots,C-1, 
\] onde \(p_{l+}=\sum_{c=1}^{C}p_{lc}\) ,
\(p_{+c}=\sum_{l=1}^{L}p_{lc}\) e
\(\sum_{c=1}^{C}\sum_{l=1}^{L}p_{lc}=1\).

Vamos escrever a hipótese de independência numa forma alternativa mas
equivalente, usando contrastes de proporções: \[
H_{0}:f_{lc}=p_{lc}-p_{l+}p_{+c}=0,\quad l=1,\ldots ,L-1,\ c=1,\ldots
,C-1.
\]

Consideremos o vetor \(\mathbf{f}\) com
\(\left( L-1\right) \left( C-1\right)\) componentes formado pelos
contrastes \(f_{lc}\) arranjados em ordem de linhas: \[
\mathbf{f}=\left( f_{11},\ldots ,f_{1\;C-1},\ldots ,f_{L-1\;1},\ldots
,f_{L-1\;C-1}\right) ^{\prime }. 
\]

Um teste da hipótese de independência pode ser definido em termos da
distância entre uma estimativa consistente do vetor de contrastes
\(\mathbf{f}\) e o vetor nulo com mesmo número de componentes. O vetor
de estimativa consistente de \(\mathbf{f}\) é denotado por
\(\mathbf{\hat{f}}=\left( \hat{f}_{11},\ldots ,\hat{f}_{1\;C-1},\ldots ,\hat{f} _{L-1\;1},\ldots ,\hat{f}_{L-1\;C-1}\right) ^{^{\prime }}\),
onde \(\hat{f} _{lc}=\hat{p}_{lc}-\hat{p}_{l+}\hat{p}_{+c}\), onde
\(\hat{p}_{lc}=\hat{n} _{lc}/n\). Os \(\hat{n}_{lc}\) são as frequências
ponderadas nas celas, considerando as diferentes probabilidades de
inclusão e ajustes por não-resposta, onde os pesos amostrais são
normalizados de modo que \(\sum_{c=1}^{C}\sum_{l=1}^{L}\hat{n}_{lc}=n\).
Se \(n\) não for fixado de antemão, os \(\hat{p}_{lc}\) serão
estimadores de razões. Apenas \(\left( L-1\right) \left( C-1\right)\)
componentes são incluídos no vetores \(\mathbf{f}\) e
\(\mathbf{\hat{f}}\), pois a soma das proporções nas celas da tabela é
igual a 1.

\subsection{Estatística de Wald Baseada no Plano
Amostral}\label{estatistica-de-wald-baseada-no-plano-amostral}

A estatística de Wald baseada no plano amostral
\(X_{W}^{2}\left(I\right)\), para o teste de independência, tem a forma
da expressão \eqref{eq:Tab8}, com \(\mathbf{\hat{f}}\) no lugar de
\(\mathbf{\hat{p}}\), o vetor
\(\mathbf{0}_{\left( L-1\right)\left( C-1\right)}\) no lugar de
\(\mathbf{p}_{0}\) e a estimativa baseada no plano amostral
\(\mathbf{\hat{V}}_{\mathbf{f}}\) da matriz de covariância de
\(\mathbf{\hat{f}}\) no lugar de \(\mathbf{ \hat{V}}_{p}\). Assim, a
estatística de teste de independência de Wald é dada por

\begin{equation}
X_{W}^{2}\left( I\right) =\mathbf{\hat{f}}^{\prime }\mathbf{\hat{V}}_{
\mathbf{f}}^{-1}\mathbf{\hat{f}\;\;},  \label{eq:Tab10}
\end{equation}

que é assintoticamente
\(\chi ^{2}\left( \left( L-1\right) \left( C-1\right) \right)\).

A estimativa \(\mathbf{\hat{V}}_{\mathbf{f}}\) da matriz de covariância
de \(\mathbf{\hat{f}}\) pode ser obtida pelo método de linearização de
Taylor apresentado na Seção \ref{taylor}, considerando o vetor de
contrastes \(\mathbf{f}\) como uma função (não-linear) do vetor
\(\mathbf{p}\), isto é,
\(\mathbf{f=g}\left( \mathbf{p}\right) \mathbf{=g}\left( p_{11},\ldots ,p_{1\;C-1},\ldots ,p_{L-1\;1},\ldots,p_{L-1\;C-1}\right)\).
Assim, a matriz de covariância de \(\mathbf{\hat{f}}\) pode ser estimada
por

\begin{equation}
\mathbf{\hat{V}}_{\mathbf{f}}=\mathbf{\Delta g}\left( \mathbf{\hat{p}}
\right) \mathbf{\hat{V}}_{p}^{-1}\mathbf{\Delta g}\left( \mathbf{\hat{p}}
\right) ^{^{\prime }},  \label{eq:Tab11}
\end{equation}

onde \(\mathbf{\Delta g}\left( \mathbf{p}\right)\) é a matriz jacobiana
de dimensão
\(\left(L-1\right)\left( C-1\right)\times\left( L-1\right)\left( C-1\right)\)
dada por

\[
\mathbf{\Delta g}\left( \mathbf{p}\right) =\left[ \partial \mathbf{g}
/\partial p_{11},\ldots ,\partial \mathbf{g}/\partial p_{1\;C-1},\ldots
,\partial \mathbf{g}/\partial p_{L-1\;1},\ldots ,\partial \mathbf{g}
/\partial p_{L-1\;C-1}\right] 
\] e \(\mathbf{\hat{V}}_{p}\) é uma estimativa consistente da matriz de
covariância de \(\mathbf{\hat{p}}\).

é possível ainda introduzir, no caso de se ter o número \(m\) de
unidades primárias pequeno, correção na estatística de Wald, utilizando
as propostas alternativas de estatísticas F-corrigidas, como em
\eqref{eq:qual9} e \eqref{eq:qual10}, com
\(\left( L-1\right) \left( C-1\right)\) no lugar de \(J-1\), obtendo-se
\[
F_{1.p}=\frac{f-\left( L-1\right) \left( C-1\right) -1}{f\left( L-1\right)
\left( C-1\right) }X_{W}^{2}\left( I\right),
\] que tem distribuição assintótica \(\mathbf{F}\) com
\(\left(L-1\right) \left( C-1\right)\) e
\(f-\left( L-1\right) \left(C-1\right) -1\) graus de liberdade e \[
F_{2.p}=\frac{X_{W}^{2}\left( I\right) }{\left( L-1\right) \left( C-1\right) 
}, 
\] que tem distribuição assintótica \(\mathbf{F}\) com
\(\left(L-1\right) \left( C-1\right)\) e \(f\) graus de liberdade.

\subsection{Estatística de Pearson com Ajuste de
Rao-Scott}\label{estatistica-de-pearson-com-ajuste-de-rao-scott}

Na presença de efeitos de plano amostral importantes, as estatísticas
clássicas de teste precisam ser ajustadas para terem a mesma
distribuição assintótica de referência que a obtida para o caso de
amostragem aleatória simples.

A estatística de teste de independência \(X_{P}^{2}\left( I\right)\) de
Pearson para a tabela \(L\times C\) é dada por \[
X_{P}^{2}\left( I\right) =n\sum\limits_{l=1}^{L}\sum\limits_{c=1}^{C}\frac{
\left( \hat{p}_{lc}-\hat{p}_{l+}\hat{p}_{+c}\right) ^{2}}{\hat{p}_{l+}\hat{p}
_{+c}}.
\]

Esta estatística pode ser escrita em forma matricial como

\begin{equation}
X_{P}^{2}\left( I\right) =n\;\mathbf{\hat{f}}^{\prime }\;\widehat{\mathbf{P}}
_{0\mathbf{f}}\;\mathbf{\hat{f}},  \label{eq:Tab12}
\end{equation}

onde

\begin{equation}
\widehat{\mathbf{P}}_{0\mathbf{f}}=\mathbf{\Delta g}\left( \mathbf{\hat{p}}
\right) \mathbf{\hat{P}}_{0}\mathbf{\Delta g}\left( \mathbf{\hat{p}}\right)
^{\prime },  \label{eq:Tab13}
\end{equation}

\[
\mathbf{\hat{P}}_{0}=diag\left( \mathbf{\hat{p}}_{0}\right) -\mathbf{\hat{p}}
_{0}\mathbf{\hat{p}}_{0}^{^{\prime }}, 
\] \(\widehat{\mathbf{P}}_{0}/n\) estima a matriz
\(\left( L-1\right) \left( C-1\right) \times \left( L-1\right) \left( C-1\right)\)
de covariância multinomial de \(\mathbf{\hat{p}}\) sob a hipótese nula,
\(\mathbf{\hat{p}}_{0}\) é o vetor com componentes \(\hat{p}_{l+}\)
\(\hat{p}_{+c}\), e \(diag\left( \mathbf{\hat{p}}_{0}\right)\)
representa a matriz diagonal com elementos \(\hat{p}_{l+}\)
\(\hat{p}_{+c}\) na diagonal.

Observemos que a forma de \(X_{P}^{2}\left( I\right)\) como expressa em
\eqref{eq:Tab12} é semelhante à da estatística de Wald dada em
\eqref{eq:Tab10}, a diferença sendo a estimativa da matriz de covariância
de \(\mathbf{\hat{f}}\) usada em cada uma dessas estatísticas.

Como nos testes de qualidade de ajuste e de homogeneidade no caso de
plano amostral complexo, podemos introduzir correções simples na
estatística de Pearson em \eqref{eq:Tab12} para obter estatísticas de
teste com distribuições assintóticas conhecidas.

Inicialmente, vamos considerar ajustes baseados nos efeitos univariados
de plano amostral estimados, \(\hat{d}_{lc}\), das estimativas das
proporções nas celas \(\hat{p}_{lc}\). O ajuste mais simples é feito
dividindo-se o valor da estatística \(X_{P}^{2}\) de Pearson pela média
\(\hat{d}_{.}\) dos efeitos univariados de plano amostral: \[
X_{P}^{2}\left( I;\hat{d}_{.}\right) =X_{P}^{2}\left( I\right) /\hat{d}
_{.}, 
\] onde
\(\hat{d}_{.}=\sum_{c=1}^{C}\sum_{l=1}^{L}\hat{d}_{lc}/\left( LC\right)\)
é um estimador da média dos efeitos univariados de plano amostral
desconhecidos.

Estimamos os efeitos do plano amostral por
\(\hat{d}_{lc}=\hat{V}_{p}\left(\hat{p}_{lc}\right) /\left( \hat{p}_{lc}\left( 1-\hat{p}_{lc}\right) /n\right)\),
onde \(\hat{V}_{p}\left( \hat{p}_{lc}\right)\) é a estimativa da
variância de aleatorização do estimador de proporção \(\hat{p}_{lc}\).
Este ajustamento requer que estejam disponíveis as estimativas dos
efeitos de plano amostral dos estimadores das proporções nas
\(L\times C\) celas da tabela.

A seguir vamos apresentar as correções de primeira e de segunda ordem de
Rao-Scott para a estatística \(X_{P}^{2}\left( I\right)\) de Pearson
para o teste de independência. Estas correções baseiam-se nos
autovalores da matriz estimada de efeito multivariado de plano amostral,
dada por

\begin{equation}
\mathbf{\hat{\Delta}}=n\;\mathbf{\hat{P}}_{0\mathbf{f}}^{-1}\;\mathbf{\hat{V}
}_{\mathbf{f}},  \label{eq:Tab14}
\end{equation}

onde \(\mathbf{\hat{V}}_{\mathbf{f}}\) foi definido em \eqref{eq:Tab11} e
\(\mathbf{ \hat{P}}_{0\mathbf{f}}\) definido em \eqref{eq:Tab13}.

O ajuste de Rao-Scott de primeira ordem para
\(X_{P}^{2}\left( I\right)\) é dado por

\begin{equation}
X_{P}^{2}\left( I;\hat{\delta}_{.}\right) =X_{P}^{2}\left( I\right) /\hat{
\delta}_{.},  \label{eq:Tab15}
\end{equation}

onde \(\hat{\delta}_{.}\) é um estimador da média \(\bar{\delta}\) dos
autovalores desconhecidos da matriz \(\mathbf{\Delta }\) de efeitos
multivariados de plano amostral.

Podemos estimar a média dos efeitos generalizados, usando os efeitos
univariados nas celas e nas marginais da tabela, por \[
\begin{array}{lll}
\hat{\delta}_{.} & = & \frac{1}{\left( L-1\right) \left( C-1\right) }
\sum\limits_{l=1}^{L}\sum\limits_{c=1}^{C}\frac{\hat{p}_{lc}\left( 1-\hat{p}
_{lc}\right) }{\hat{p}_{l+}\hat{p}_{+c}}\hat{d}_{lc} \\ 
&  & -\sum\limits_{l=1}^{L}\left( 1-\hat{p}_{l+}\right) \hat{d}
_{l+}-\sum\limits_{c=1}^{C}\left( 1-\hat{p}_{+c}\right) \hat{d}_{+c},
\end{array}
\;
\] sem precisar calcular a matriz de efeitos multivariados de plano
amostral. A distribuição assintótica de
\(X_{P}^{2}\left( I;\hat{\delta} _{.}\right)\), sob \(H_{0}\), é
qui-quadrado com \(\left( L-1\right) \times \left( C-1\right)\) graus de
liberdade.

O ajuste de Rao-Scott de segunda ordem é definido por \[
X_{P}^{2}\left( I;\hat{\delta}_{.};\hat{a}^{2}\right) =X_{P}^{2}\left(
I\right) /\left( \hat{\delta}_{.}\left( 1+\hat{a}^{2}\right) \right), 
\] onde \(\hat{\delta}_{.}\) é um estimador da média dos autovalores de
\(\mathbf{\hat{\Delta}}\), dado por

\[
\hat{\delta}_{.}=\frac{tr\left( \mathbf{\hat{\Delta}}\right) }{\left(
L-1\right) \left( C-1\right) } 
\] e \(\hat{a}^{2}\) é um estimador do quadrado do coeficiente de
variação dos autovalores desconhecidos de \(\mathbf{\Delta}\),
\(\delta _{k}\), \(k=1,\ldots ,\left( L-1\right) \left( C-1\right)\),
dado por

\[
\hat{a}^{2}=\sum\limits_{k=1}^{\left( L-1\right) \left( C-1\right) }\hat{
\delta}_{k}^{2}/\left( \left( L-1\right) \left( C-1\right) \hat{\delta}
_{.}^{2}\right) -1. 
\]

Um estimador da soma dos quadrados dos autovalores é \[
\sum\limits_{k=1}^{\left( L-1\right) \left( C-1\right) }\hat{\delta}
_{k}^{2}=tr\left( \mathbf{\hat{\Delta}}^{2}\right).
\]

A estatística \(X_{P}^{2}\left( I;\hat{\delta}_{.};\hat{a}^{2}\right)\)
é assintoticamente qui-quadrado com graus de liberdade com ajuste de
Satterthwaite
\(gl_{S}=\left( L-1\right) \left( C-1\right) /\left( 1+\hat{a} ^{2}\right) .\)

Em situações instáveis, pode ser necessário fazer uma correção F ao
ajuste de primeira ordem de Rao-Scott \eqref{eq:Tab15}. A estatística
F-corrigida é definida por

\begin{equation}
FX_{P}^{2}\left( \hat{\delta}_{.}\right) =X_{P}^{2}\left( \hat{\delta}
_{.}\right) /\left( L-1\right) \left( C-1\right) \;\;.  \label{eq:Tab16}
\end{equation}

A estatística \eqref{eq:Tab16} tem distribuição de referência \(F\) com
\(\left( L-1\right)\) \(\times\left( C-1\right)\) e \(f\) graus de
liberdade.

\BeginKnitrBlock{example}
\protect\hypertarget{ex:unnamed-chunk-1}{}{\label{ex:unnamed-chunk-1}}Correções
de EPA médio das estatísticas \(X_{P}^{2}\left(I\right)\) e
\(X_{P}^{2}\left(H\right)\).
\EndKnitrBlock{example}

Considerando os dados do Exemplo \ref{ex:pnad} do Capítulo \ref{modreg},
vamos testar a hipótese de independência entre as variáveis Sexo (sx) e
Rendimento médio mensal (re). Vamos fazer também um teste de
homogeneidade, para comparar as distribuições de renda para os dois
sexos.

A variável \texttt{sx} tem dois níveis: sx(1)-Homens, sx(2)- Mulheres e
a variável \texttt{re} tem três níveis: re(1)- Menos de salário mínimo,
re(2) - de 1 a 5 salário mínimos e re(3)- mais de 5 salários mínimos. A
Tabela \ref{tab84} apresenta as frequências nas celas para a amostra
pesquisada.

\begin{center}
\begin{table}[tbp] \centering
\caption{Frequências amostrais por celas na PNAD 90}\bigskip \label{tab84}
\begin{tabular}{|c|cccc|}
\hline\hline
& \multicolumn{3}{|c}{\textbf{Renda Mensal}} &  \\
\textbf{Sexo} & 1 & 2 & 3 & \multicolumn{1}{|c|}{\textbf{Total}} \\
\hline\hline
1 & \multicolumn{1}{|r}{$476$} & \multicolumn{1}{r}{$2.527$} &
\multicolumn{1}{r}{$1.273$} & \multicolumn{1}{|r|}{$4.276$} \\
2 & \multicolumn{1}{|r}{$539$} & \multicolumn{1}{r}{$1.270$} &
\multicolumn{1}{r}{$422$} & \multicolumn{1}{|r|}{$2.231$} \\ \hline\hline
\textbf{Total} & \multicolumn{1}{|r}{$1.015$} & \multicolumn{1}{r}{$3.797$}
& \multicolumn{1}{r}{$1.695$} & \multicolumn{1}{|r|}{$6.507$} \\ \hline\hline
\end{tabular}
\end{table}
\end{center}

No teste de homogeneidade das distribuições de renda, consideramos
fixadas as marginais \(4.276\) e \(2.231\) da variável Sexo na tabela de
frequências amostrais. Usando o programa Stata, calculamos as
estimativas das proporções nas linhas da tabela. Nestas estimativas são
considerados os pesos das unidades da amostra e o plano amostral
utilizado na pesquisa (PNAD 90), conforme descrito no Exemplo
\ref(ex:pnad90) do Capítulo \ref{modreg}.

Vamos considerar o teste de homogeneidade entre as variáveis Sexo e
Renda e calcular o efeito de plano amostral médio das estimativas das
proporções nas celas da tabela. A Tabela \ref{tab85} contém, em cada
cela, as estimativas: da proporção na linha, do desvio-padrão da
estimativa da proporção na linha (\(\times10.000\)), e do efeito de
plano amostral da estimativa de proporção na linha.

\begin{center}
\begin{table}[tbp] \centering
\caption{Proporções nas linhas, desvios padrões e EPAs}\bigskip \label{tab85}
\begin{tabular}{|c|ccc|c|}
\hline\hline
& \multicolumn{3}{|c|}{\textbf{Renda Mensal}} &  \\ \cline{2-4}
\textbf{Sexo} & 1 & 2 & 3 & \textbf{Total} \\ \hline\hline
1 & \multicolumn{1}{|r}{$
\begin{tabular}{r}
$0,111$ \\
$57,269$ \\
$1,420$
\end{tabular}
$} & \multicolumn{1}{r}{$
\begin{tabular}{r}
$0,591$ \\
$102,576$ \\
$1,861$
\end{tabular}
$} & \multicolumn{1}{r|}{$
\begin{tabular}{r}
$0,298$ \\
$111,213$ \\
$2,527$
\end{tabular}
$} &
\begin{tabular}{c}
$1,00$
\end{tabular}
\\ \hline
2 & \multicolumn{1}{|r}{$
\begin{tabular}{r}
$0,240$ \\
$125,026$ \\
$1,909$
\end{tabular}
$} & \multicolumn{1}{r}{$
\begin{tabular}{r}
$0,570$ \\
$119,375$ \\
$1,297$
\end{tabular}
$} & \multicolumn{1}{r|}{$
\begin{tabular}{r}
$0,190$ \\
$111,410$ \\
$1,800$
\end{tabular}
$} &
\begin{tabular}{c}
$1,00$
\end{tabular}
\\ \hline\hline
\begin{tabular}{c}
Amostra \\
completa
\end{tabular}
& \multicolumn{1}{|r}{$
\begin{tabular}{r}
$0,155$ \\
$68,977$ \\
$2,358$
\end{tabular}
$} & \multicolumn{1}{r}{$
\begin{tabular}{r}
$0,584$ \\
$82,001$ \\
$1,800$
\end{tabular}
$} & \multicolumn{1}{r|}{$
\begin{tabular}{r}
$0,261$ \\
$96,1300$ \\
$3,117$
\end{tabular}
$} &
\begin{tabular}{c}
$1,00$
\end{tabular}
\\ \hline\hline
\end{tabular}
\end{table}
\end{center}

\begin{Shaded}
\begin{Highlighting}[]
\NormalTok{marg_re_pop <-}\KeywordTok{as.data.frame}\NormalTok{(}\KeywordTok{svymean}\NormalTok{(~re,pnad.des, }\DataTypeTok{deff=}\OtherTok{TRUE} \NormalTok{))}
\NormalTok{knitr::}\KeywordTok{kable}\NormalTok{(marg_re_pop ,}\DataTypeTok{booktabs=}\OtherTok{TRUE}\NormalTok{, }\DataTypeTok{digits=}\DecValTok{3}\NormalTok{,}
  \DataTypeTok{caption=}\StringTok{"Proporções nas linha, desvios padrões e EPAs de `re` na população"}\NormalTok{)}
\end{Highlighting}
\end{Shaded}

Vamos calcular, a título de ilustração, uma das celas de tabela de
efeitos de plano amostral, digamos a cela (1,1). A estimativa da
variância do estimador da proporção de linha nesta cela é
\(\left( 0,0057269\right) ^{2}\). Sob amostragem aleatória simples com
reposição, a estimativa da variância do estimador de proporção de linha
na cela é: \(0,111\left( 1-0,111\right) /4.276\). A estimativa do efeito
de plano amostral do estimador de proporção na cela é portanto igual a
\[
\frac{\left( 0,0057269\right) ^{2}}{0,111\left( 1-0,111\right) /4.276}\cong
1,420\;\;. 
\]

A estimativa do efeito médio de plano amostral para corrigir a
estatística \(X_{P}^{2}\left( H\right)\) é \(\hat{d}_{.}=1,802\) ,
calculada tomando a média dos EPAs das celas correspondentes aos níveis
1 e 2 da variável \texttt{sx}.

Vamos agora considerar o teste de independência entre as variáveis Sexo
e Renda e calcular o efeito de plano amostral médio das estimativas das
proporções nas celas da tabela. A Tabela \ref{tab86} contém, em cada
cela, as estimativas: da proporção na cela, do desvio-padrão da
estimativa da proporção na cela (\(\times 10.000\)), e do efeito de
plano amostral da estimativa de proporção na cela.

\begin{center}
\begin{table}[tbp] \centering
\caption{Proporções nas celas, desvios padrões e EPAs}\bigskip \label{tab86}
\begin{tabular}{|c|ccc|c|}
\hline\hline
& \multicolumn{3}{|c|}{\textbf{Renda Mensal}} &  \\ \cline{2-4}
\textbf{Sexo} & 1 & 2 & 3 & \textbf{Total} \\ \hline\hline
1 & \multicolumn{1}{|r}{$
\begin{tabular}{r}
$0,073$ \\
$38,343$ \\
$1,414$
\end{tabular}
$} & \multicolumn{1}{r}{
\begin{tabular}{r}
$0,388$ \\
$80,435$ \\
$1,772$
\end{tabular}
} & \multicolumn{1}{r|}{
\begin{tabular}{r}
$0,196$ \\
$71,772$ \\
$2,128$
\end{tabular}
} & $
\begin{tabular}{r}
$0,657$ \\
$55,814$ \\
$0,899$
\end{tabular}
$ \\ \hline
2 &
\begin{tabular}{r}
$0,082$ \\
$44,401$ \\
$1,695$
\end{tabular}
&
\begin{tabular}{r}
$0,195$ \\
$51,582$ \\
$1,101$
\end{tabular}
&
\begin{tabular}{r}
$0,065$ \\
$40,219$ \\
$1,729$
\end{tabular}
&
\begin{tabular}{r}
$0,343$ \\
$55,814$ \\
$0,899$
\end{tabular}
\\ \hline\hline
\textbf{Total} &
\begin{tabular}{r}
$0,155$ \\
$68,977$ \\
$2,358$
\end{tabular}
&
\begin{tabular}{r}
$0,584$ \\
$82,001$ \\
$1,800$
\end{tabular}
&
\begin{tabular}{r}
$0,261$ \\
$96,130$ \\
$3,117$
\end{tabular}
& $
\begin{tabular}{r}
$1,000$
\end{tabular}
$ \\ \hline\hline
\end{tabular}
\end{table}
\end{center}

Tabela de proporções de \texttt{sx} para a população inteira:

\begin{Shaded}
\begin{Highlighting}[]
\NormalTok{marg_sx_pop <-}\KeywordTok{data.frame}\NormalTok{(}\KeywordTok{svymean}\NormalTok{(~sx,pnad.des, }\DataTypeTok{deff=}\OtherTok{TRUE} \NormalTok{))}
\NormalTok{marg_sx_pop <-}\StringTok{ }\KeywordTok{transform}\NormalTok{(marg_sx_pop, }\DataTypeTok{SE =} \DecValTok{10000}\NormalTok{*SE)}
\NormalTok{knitr::}\KeywordTok{kable}\NormalTok{(marg_sx_pop, }\DataTypeTok{digits=}\DecValTok{3}\NormalTok{,}
  \DataTypeTok{caption=}\StringTok{"Proporções nas linha, desvios padrões e EPAs de sx na população"}\NormalTok{)}
\end{Highlighting}
\end{Shaded}

Vamos calcular, a título de ilustração, o efeito de plano amostral na
cela (1,1) da Tabela \ref{tab86}. A estimativa da variância do estimador
de proporção nesta cela é \(\left(0,0038343\right) ^{2}\). Sob
amostragem aleatória simples com reposição, a estimativa da variância do
estimador de proporção na cela é:
\(0,073\times \left( 1-0,073\right) /6.507\). A estimativa do efeito de
plano amostral do estimador de proporção na cela é

\[
\frac{\left( 0,0038343\right) ^{2}}{0,073\left( 1-0,073\right) /6.507}\cong
1,414\;\;. 
\]

Portanto, a estimativa do efeito médio de plano amostral requerida para
corrigir a estatística \(X_{P}^{2}\left( I\right)\) é
\(\hat{d}_{.}=1,640,\) calculada tomando a média dos EPAs das celas
correspondentes aos níveis 1 e 2 da variável \texttt{sx}.

Calculando as estatísticas \(X_{P}^{2}\left( I\right)\) e
\(X_{P}^{2}\left( H\right)\) para os testes clássicos de independência e
homogeneidade a partir da Tabela \ref{tab86}, obtemos os valores
\(X_{P}^{2}\left( I\right) =X_{P}^{2}\left( H\right) =227,025\), com
distribuição de referência \(\chi ^{2}\left( 2\right)\), resultado que
indica rejeição da hipótese de independência entre \texttt{sx} e
\texttt{re}, bem como da hipótese de igualdade de distribuição de renda
para os dois sexos a partir do teste de homogeneidade. O valor comum das
estatísticas \(X_{P}^{2}\left( I\right)\) e \(X_{P}^{2}\left( H\right)\)
foi calculado sem considerar os pesos e o plano amostral. Considerando
estes últimos, mediante a correção de EPA médio das estatísticas
clássicas, obtemos os valores
\(X_{P}^{2}\left( I;\hat{d} _{.}\right) =137,117\) e
\(X_{P}^{2}\left( H;\hat{d}_{.}\right) =124,742\), que também indicam a
rejeição das hipóteses de independência e de homogeneidade.

Vale ressaltar que apesar de todos os testes mencionados indicarem forte
rejeição das hipóteses de independência e de homogeneidade, os valores
das estatísticas de teste \(137,117\) e \(124,742\) , calculados
considerando os pesos e plano amostral, são bem menores que o valor
\(227,025\) obtido para o caso de amostra IID. Sob a hipótese nula, a
distribuição de referência de todas essas estatísticas de teste é
\(\chi ^{2}\left( 2\right)\), mostrando novamente que a estatística de
teste calculada sob a hipótese de amostra IID tem maior tendência a
rejeitar a hipótese nula.

A partir da Tabela \ref{tab86}, examinando as estimativas das proporções
nas celas da tabela para cada sexo, observamos uma ordenação estocástica
das distribuições de renda para os dois sexos, com proporções maiores em
valores mais altos para o nível 1 da variável sexo, que é o sexo
masculino.

\section{Laboratório de R}\label{laboratorio-de-r}

Vamos reproduzir alguns resultados usando dados da PNAD descritos na
Seção ???

\BeginKnitrBlock{example}
\protect\hypertarget{ex:unnamed-chunk-4}{}{\label{ex:unnamed-chunk-4}}Estimativas
de medidas descritivas em tabelas
\EndKnitrBlock{example}

\begin{Shaded}
\begin{Highlighting}[]
\KeywordTok{library}\NormalTok{(survey)}
\end{Highlighting}
\end{Shaded}

\begin{verbatim}
## Loading required package: grid
\end{verbatim}

\begin{verbatim}
## Loading required package: Matrix
\end{verbatim}

\begin{verbatim}
## Loading required package: survival
\end{verbatim}

\begin{verbatim}
## 
## Attaching package: 'survey'
\end{verbatim}

\begin{verbatim}
## The following object is masked from 'package:graphics':
## 
##     dotchart
\end{verbatim}

\begin{Shaded}
\begin{Highlighting}[]
\NormalTok{pnadrj90 <-}\StringTok{ }\KeywordTok{readRDS}\NormalTok{(}\StringTok{"~}\CharTok{\textbackslash{}\textbackslash{}}\StringTok{GitHub}\CharTok{\textbackslash{}\textbackslash{}}\StringTok{adac}\CharTok{\textbackslash{}\textbackslash{}}\StringTok{data}\CharTok{\textbackslash{}\textbackslash{}}\StringTok{pnadrj90.rds"}\NormalTok{)}
\KeywordTok{names}\NormalTok{(pnadrj90)}
\end{Highlighting}
\end{Shaded}

\begin{verbatim}
##  [1] "stra"     "psu"      "pesopes"  "informal" "sx"       "id"      
##  [7] "ae"       "ht"       "re"       "um"
\end{verbatim}

Transformação em fatores:

\begin{Shaded}
\begin{Highlighting}[]
\KeywordTok{unlist}\NormalTok{(}\KeywordTok{lapply}\NormalTok{(pnadrj90, mode))}
\end{Highlighting}
\end{Shaded}

\begin{verbatim}
##      stra       psu   pesopes  informal        sx        id        ae 
## "numeric" "numeric" "numeric" "numeric" "numeric" "numeric" "numeric" 
##        ht        re        um 
## "numeric" "numeric" "numeric"
\end{verbatim}

\begin{Shaded}
\begin{Highlighting}[]
\NormalTok{pnadrj90<-}\KeywordTok{transform}\NormalTok{(pnadrj90,}\DataTypeTok{sx=}\KeywordTok{factor}\NormalTok{(sx),}\DataTypeTok{id=}\KeywordTok{factor}\NormalTok{(id),}\DataTypeTok{ae=}\KeywordTok{factor}\NormalTok{(ae),}\DataTypeTok{ht=}\KeywordTok{factor}\NormalTok{(ht),}\DataTypeTok{re=}\KeywordTok{factor}\NormalTok{(re))}
\end{Highlighting}
\end{Shaded}

Definição do objeto de desenho:

\begin{Shaded}
\begin{Highlighting}[]
\NormalTok{pnad.des<-}\KeywordTok{svydesign}\NormalTok{(}\DataTypeTok{id=}\NormalTok{~psu,}\DataTypeTok{strata=}\NormalTok{~stra,}\DataTypeTok{weights=}\NormalTok{~pesopes,}\DataTypeTok{data=}\NormalTok{pnadrj90,}\DataTypeTok{nest=}\OtherTok{TRUE}\NormalTok{)}
\end{Highlighting}
\end{Shaded}

Estimativas de proporções:

\begin{Shaded}
\begin{Highlighting}[]
\KeywordTok{svymean}\NormalTok{(~sx,pnad.des)         }\CommentTok{#estimativa de proporção para sx}
\end{Highlighting}
\end{Shaded}

\begin{verbatim}
##        mean     SE
## sx1 0.65708 0.0056
## sx2 0.34292 0.0056
\end{verbatim}

\begin{Shaded}
\begin{Highlighting}[]
\KeywordTok{svymean}\NormalTok{(~re,pnad.des)         }\CommentTok{#estimativa de proporçâo para re}
\end{Highlighting}
\end{Shaded}

\begin{verbatim}
##        mean     SE
## re1 0.15546 0.0069
## re2 0.58356 0.0082
## re3 0.26098 0.0096
\end{verbatim}

\begin{Shaded}
\begin{Highlighting}[]
\KeywordTok{svymean}\NormalTok{(~ae,pnad.des)         }\CommentTok{#estimativa de proporção para ae}
\end{Highlighting}
\end{Shaded}

\begin{verbatim}
##        mean     SE
## ae1 0.31304 0.0095
## ae2 0.31972 0.0071
## ae3 0.36725 0.0105
\end{verbatim}

\begin{Shaded}
\begin{Highlighting}[]
\NormalTok{ht.mean<-}\KeywordTok{svymean}\NormalTok{(~ht,pnad.des)}
\end{Highlighting}
\end{Shaded}

Exemplos de funções extratoras e atributos

\begin{Shaded}
\begin{Highlighting}[]
\KeywordTok{coef}\NormalTok{(ht.mean)                         }\CommentTok{#estimativas das proporções}
\end{Highlighting}
\end{Shaded}

\begin{verbatim}
##       ht1       ht2       ht3 
## 0.2103714 0.6148881 0.1747405
\end{verbatim}

\begin{Shaded}
\begin{Highlighting}[]
\KeywordTok{attributes}\NormalTok{(ht.mean)                          }\CommentTok{#ver atributos}
\end{Highlighting}
\end{Shaded}

\begin{verbatim}
## $names
## [1] "ht1" "ht2" "ht3"
## 
## $var
##               ht1           ht2           ht3
## ht1  3.666206e-05 -3.322546e-05 -3.436592e-06
## ht2 -3.322546e-05  6.758652e-05 -3.436106e-05
## ht3 -3.436592e-06 -3.436106e-05  3.779765e-05
## 
## $statistic
## [1] "mean"
## 
## $class
## [1] "svystat"
\end{verbatim}

\begin{Shaded}
\begin{Highlighting}[]
\KeywordTok{vcov}\NormalTok{(ht.mean)                         }\CommentTok{#estimativas de variâncias e covariâncias}
\end{Highlighting}
\end{Shaded}

\begin{verbatim}
##               ht1           ht2           ht3
## ht1  3.666206e-05 -3.322546e-05 -3.436592e-06
## ht2 -3.322546e-05  6.758652e-05 -3.436106e-05
## ht3 -3.436592e-06 -3.436106e-05  3.779765e-05
\end{verbatim}

\begin{Shaded}
\begin{Highlighting}[]
\KeywordTok{attr}\NormalTok{(ht.mean, }\StringTok{"var"}\NormalTok{)}
\end{Highlighting}
\end{Shaded}

\begin{verbatim}
##               ht1           ht2           ht3
## ht1  3.666206e-05 -3.322546e-05 -3.436592e-06
## ht2 -3.322546e-05  6.758652e-05 -3.436106e-05
## ht3 -3.436592e-06 -3.436106e-05  3.779765e-05
\end{verbatim}

Podemos obter estimativas de proporções nas classes de renda por
domínios definidos pela variável \texttt{sx}:

\begin{Shaded}
\begin{Highlighting}[]
\KeywordTok{library}\NormalTok{(xtable)}
\KeywordTok{print}\NormalTok{(}\KeywordTok{xtable}\NormalTok{(}\KeywordTok{svyby}\NormalTok{(~re,~sx,pnad.des,svymean,}\DataTypeTok{keep.var=}\OtherTok{TRUE}\NormalTok{),}\DataTypeTok{caption=}\StringTok{"Proporções por sexo"}\NormalTok{),}\DataTypeTok{type=}\StringTok{"html"}\NormalTok{, }\DataTypeTok{size=}\StringTok{"small"}\NormalTok{, }\DataTypeTok{include.rownames=}\OtherTok{FALSE}\NormalTok{)}
\end{Highlighting}
\end{Shaded}

As proporções estimadas nas classes de renda e a tabela cruzada das
variáveis sexo e renda são obtidas a seguir:

\begin{Shaded}
\begin{Highlighting}[]
\KeywordTok{svymean}\NormalTok{(~re,pnad.des,}\DataTypeTok{deff=}\NormalTok{T)}
\end{Highlighting}
\end{Shaded}

\begin{verbatim}
##          mean        SE   DEff
## re1 0.1554555 0.0068977 2.3611
## re2 0.5835630 0.0082001 1.8027
## re3 0.2609815 0.0096130 3.1216
\end{verbatim}

\begin{Shaded}
\begin{Highlighting}[]
\KeywordTok{round}\NormalTok{(}\KeywordTok{svytable}\NormalTok{(~sx+re,pnad.des,}\DataTypeTok{Ntotal=}\DecValTok{1}\NormalTok{),}\DataTypeTok{digits=}\DecValTok{3}\NormalTok{)}
\end{Highlighting}
\end{Shaded}

\begin{verbatim}
##    re
## sx      1     2     3
##   1 0.073 0.388 0.196
##   2 0.082 0.195 0.065
\end{verbatim}

\begin{Shaded}
\begin{Highlighting}[]
\KeywordTok{svyby}\NormalTok{(~re,~sx,pnad.des,svymean,}\DataTypeTok{keep.var=}\NormalTok{T)}
\end{Highlighting}
\end{Shaded}

\begin{verbatim}
##   sx       re1       re2       re3      se.re1     se.re2     se.re3
## 1  1 0.1110831 0.5908215 0.2980955 0.005726888 0.01025759 0.01112131
## 2  2 0.2404788 0.5696548 0.1898663 0.012502636 0.01193753 0.01114102
\end{verbatim}

\begin{Shaded}
\begin{Highlighting}[]
\KeywordTok{svymean}\NormalTok{(~}\KeywordTok{I}\NormalTok{((sx==}\DecValTok{1}\NormalTok{&re==}\DecValTok{1}\NormalTok{)*}\DecValTok{1}\NormalTok{),pnad.des,}\DataTypeTok{deff=}\NormalTok{T)}
\end{Highlighting}
\end{Shaded}

\begin{verbatim}
##                                 mean        SE   DEff
## I((sx == 1 & re == 1) * 1) 0.0729904 0.0038343 1.4156
\end{verbatim}

\begin{Shaded}
\begin{Highlighting}[]
\CommentTok{#proporções nas celas}
\KeywordTok{svytable}\NormalTok{(~sx+re,pnad.des,}\DataTypeTok{Ntotal=}\DecValTok{1}\NormalTok{)}
\end{Highlighting}
\end{Shaded}

\begin{verbatim}
##    re
## sx           1          2          3
##   1 0.07299044 0.38821684 0.19587250
##   2 0.08246505 0.19534616 0.06510900
\end{verbatim}

\begin{Shaded}
\begin{Highlighting}[]
\CommentTok{# porcentagens nas celas}
\KeywordTok{svytable}\NormalTok{(~sx+re,pnad.des,}\DataTypeTok{Ntotal=}\DecValTok{100}\NormalTok{)}
\end{Highlighting}
\end{Shaded}

\begin{verbatim}
##    re
## sx          1         2         3
##   1  7.299044 38.821684 19.587250
##   2  8.246505 19.534616  6.510900
\end{verbatim}

\begin{Shaded}
\begin{Highlighting}[]
\CommentTok{# produz se e deff}
\NormalTok{sx.re_mean<-}\KeywordTok{svymean}\NormalTok{(~}\KeywordTok{interaction}\NormalTok{(sx,re),pnad.des,}\DataTypeTok{deff=}\NormalTok{T)}
\end{Highlighting}
\end{Shaded}

Resultados dos testes obtidos pela library \texttt{survey}
\citep{R-survey} precisam ser identificados com as fórmulas do texto:

\begin{Shaded}
\begin{Highlighting}[]
\CommentTok{# teste Chi-quadrado de Pearson com ajuste de Rao-Scott}
\KeywordTok{svychisq}\NormalTok{( ~}\StringTok{ }\NormalTok{sx+re , pnad.des, }\DataTypeTok{statistic =} \StringTok{"F"}\NormalTok{, }\DataTypeTok{na.rm=}\OtherTok{TRUE}\NormalTok{)}
\end{Highlighting}
\end{Shaded}

\begin{verbatim}
## 
##  Pearson's X^2: Rao & Scott adjustment
## 
## data:  svychisq(~sx + re, pnad.des, statistic = "F", na.rm = TRUE)
## F = 108.34, ndf = 1.9779, ddf = 1275.8000, p-value < 2.2e-16
\end{verbatim}

\begin{Shaded}
\begin{Highlighting}[]
\CommentTok{# teste Chi-quadrado de Pearson com ajuste de Rao-Scott}
\KeywordTok{svychisq}\NormalTok{( ~}\StringTok{ }\NormalTok{sx+re , pnad.des, }\DataTypeTok{statistic =} \StringTok{"Chisq"}\NormalTok{, }\DataTypeTok{na.rm=}\OtherTok{TRUE}\NormalTok{)}
\end{Highlighting}
\end{Shaded}

\begin{verbatim}
## 
##  Pearson's X^2: Rao & Scott adjustment
## 
## data:  svychisq(~sx + re, pnad.des, statistic = "Chisq", na.rm = TRUE)
## X-squared = 224.85, df = 2, p-value < 2.2e-16
\end{verbatim}

\begin{Shaded}
\begin{Highlighting}[]
\CommentTok{# teste de Wald baseado no desenho amostral}
\KeywordTok{svychisq}\NormalTok{( ~}\StringTok{ }\NormalTok{sx+re , pnad.des, }\DataTypeTok{statistic =} \StringTok{"Wald"}\NormalTok{, }\DataTypeTok{na.rm=}\OtherTok{TRUE}\NormalTok{)}
\end{Highlighting}
\end{Shaded}

\begin{verbatim}
## 
##  Design-based Wald test of association
## 
## data:  svychisq(~sx + re, pnad.des, statistic = "Wald", na.rm = TRUE)
## F = 68.41, ndf = 2, ddf = 645, p-value < 2.2e-16
\end{verbatim}

\begin{Shaded}
\begin{Highlighting}[]
\CommentTok{# teste de Wald com ajuste}
\KeywordTok{svychisq}\NormalTok{( ~}\StringTok{ }\NormalTok{sx+re , pnad.des, }\DataTypeTok{statistic =} \StringTok{"adjWald"}\NormalTok{, }\DataTypeTok{na.rm=}\OtherTok{TRUE}\NormalTok{)}
\end{Highlighting}
\end{Shaded}

\begin{verbatim}
## 
##  Design-based Wald test of association
## 
## data:  svychisq(~sx + re, pnad.des, statistic = "adjWald", na.rm = TRUE)
## F = 68.304, ndf = 2, ddf = 644, p-value < 2.2e-16
\end{verbatim}

\begin{Shaded}
\begin{Highlighting}[]
\CommentTok{# teste Chi-quadrado de Pearson: distribuição assintótica exata}
\KeywordTok{svychisq}\NormalTok{( ~}\StringTok{ }\NormalTok{sx+re , pnad.des, }\DataTypeTok{statistic =} \StringTok{"lincom"}\NormalTok{, }\DataTypeTok{na.rm=}\OtherTok{TRUE}\NormalTok{)}
\end{Highlighting}
\end{Shaded}

\begin{verbatim}
## Warning in pFsum(pearson$statistic, rep(1, ncol(Delta)), eigen(Delta,
## only.values = TRUE)$values, : Package 'CompQuadForm' not found, using
## saddlepoint approximation
\end{verbatim}

\begin{verbatim}
## 
##  Pearson's X^2: asymptotic exact distribution
## 
## data:  svychisq(~sx + re, pnad.des, statistic = "lincom", na.rm = TRUE)
## X-squared = 224.85, p-value < 2.2e-16
\end{verbatim}

\begin{Shaded}
\begin{Highlighting}[]
\CommentTok{# teste Chi-quadrado de Pearson: aproximação de ponto de cela}
\KeywordTok{svychisq}\NormalTok{( ~}\StringTok{ }\NormalTok{sx+re , pnad.des, }\DataTypeTok{statistic =} \StringTok{"saddlepoint"}\NormalTok{, }\DataTypeTok{na.rm=}\OtherTok{TRUE}\NormalTok{)}
\end{Highlighting}
\end{Shaded}

\begin{verbatim}
## 
##  Pearson's X^2: saddlepoint approximation
## 
## data:  svychisq(~sx + re, pnad.des, statistic = "saddlepoint", na.rm = TRUE)
## X-squared = 224.85, p-value < 2.2e-16
\end{verbatim}

\chapter{Estimação de densidades}\label{estimacao-de-densidades}

\chapter{Modelos Hierárquicos}\label{modelos-hierarquicos}

\chapter{Não-Resposta}\label{nao-resposta}

\chapter{Diagnóstico de ajuste de
modelo}\label{diagnostico-de-ajuste-de-modelo}

\chapter{Agregação vs.~Desagregação}\label{agregdesag}

\chapter{Pacotes para Analisar Dados Amostrais}\label{pacotes}

\chapter{Placeholder}\label{placeholder}

\bibliography{packages,book}


\end{document}
