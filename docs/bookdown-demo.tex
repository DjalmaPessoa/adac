\documentclass[]{book}
\usepackage{lmodern}
\usepackage{amssymb,amsmath}
\usepackage{ifxetex,ifluatex}
\usepackage{fixltx2e} % provides \textsubscript
\ifnum 0\ifxetex 1\fi\ifluatex 1\fi=0 % if pdftex
  \usepackage[T1]{fontenc}
  \usepackage[utf8]{inputenc}
\else % if luatex or xelatex
  \ifxetex
    \usepackage{mathspec}
  \else
    \usepackage{fontspec}
  \fi
  \defaultfontfeatures{Ligatures=TeX,Scale=MatchLowercase}
\fi
% use upquote if available, for straight quotes in verbatim environments
\IfFileExists{upquote.sty}{\usepackage{upquote}}{}
% use microtype if available
\IfFileExists{microtype.sty}{%
\usepackage{microtype}
\UseMicrotypeSet[protrusion]{basicmath} % disable protrusion for tt fonts
}{}
\usepackage[margin=1in]{geometry}
\usepackage{hyperref}
\hypersetup{unicode=true,
            pdftitle={Análise de Dados Amostrais Complexos},
            pdfauthor={Djalma Pessoa e Pedro Nascimento Silva},
            pdfborder={0 0 0},
            breaklinks=true}
\urlstyle{same}  % don't use monospace font for urls
\usepackage{natbib}
\bibliographystyle{apalike}
\usepackage{longtable,booktabs}
\usepackage{graphicx,grffile}
\makeatletter
\def\maxwidth{\ifdim\Gin@nat@width>\linewidth\linewidth\else\Gin@nat@width\fi}
\def\maxheight{\ifdim\Gin@nat@height>\textheight\textheight\else\Gin@nat@height\fi}
\makeatother
% Scale images if necessary, so that they will not overflow the page
% margins by default, and it is still possible to overwrite the defaults
% using explicit options in \includegraphics[width, height, ...]{}
\setkeys{Gin}{width=\maxwidth,height=\maxheight,keepaspectratio}
\IfFileExists{parskip.sty}{%
\usepackage{parskip}
}{% else
\setlength{\parindent}{0pt}
\setlength{\parskip}{6pt plus 2pt minus 1pt}
}
\setlength{\emergencystretch}{3em}  % prevent overfull lines
\providecommand{\tightlist}{%
  \setlength{\itemsep}{0pt}\setlength{\parskip}{0pt}}
\setcounter{secnumdepth}{5}
% Redefines (sub)paragraphs to behave more like sections
\ifx\paragraph\undefined\else
\let\oldparagraph\paragraph
\renewcommand{\paragraph}[1]{\oldparagraph{#1}\mbox{}}
\fi
\ifx\subparagraph\undefined\else
\let\oldsubparagraph\subparagraph
\renewcommand{\subparagraph}[1]{\oldsubparagraph{#1}\mbox{}}
\fi

%%% Use protect on footnotes to avoid problems with footnotes in titles
\let\rmarkdownfootnote\footnote%
\def\footnote{\protect\rmarkdownfootnote}

%%% Change title format to be more compact
\usepackage{titling}

% Create subtitle command for use in maketitle
\newcommand{\subtitle}[1]{
  \posttitle{
    \begin{center}\large#1\end{center}
    }
}

\setlength{\droptitle}{-2em}
  \title{Análise de Dados Amostrais Complexos}
  \pretitle{\vspace{\droptitle}\centering\huge}
  \posttitle{\par}
  \author{Djalma Pessoa e Pedro Nascimento Silva}
  \preauthor{\centering\large\emph}
  \postauthor{\par}
  \predate{\centering\large\emph}
  \postdate{\par}
  \date{2017-02-03}

\usepackage{booktabs}
\usepackage[brazil]{babel}
\newtheorem{example}{Exemplo}
\numberwithin{example}{chapter}
\newtheorem{remark}{Observação}
\numberwithin{remark}{chapter}
\newtheorem{definition}{Definição}
\numberwithin{definition}{chapter}

\let\BeginKnitrBlock\begin \let\EndKnitrBlock\end
\begin{document}
\maketitle

{
\setcounter{tocdepth}{1}
\tableofcontents
}
\chapter*{Prefácio}\label{prefacio}
\addcontentsline{toc}{chapter}{Prefácio}

\chapter{Introdução}\label{introduc}

\chapter{Referencial para Inferência}\label{refinf}

\chapter{Estimação Baseada no Plano Amostral}\label{capplanamo}

\chapter{Efeitos do Plano Amostral}\label{epa}

\chapter{Ajuste de Modelos Paramétricos}\label{ajmodpar}

\section{Introdução}\label{modpar1}

Nos primórdios do uso \texttt{moderno} de pesquisas por amostragem, os
dados obtidos eram usados principalmente para estimar funções simples
dos valores das variáveis de interesse nas populações finitas, tais como
totais, médias, razões, etc. Isto caracterizava o uso dos dados dessas
pesquisas para \texttt{inferência\ descritiva}. Recentemente, os dados
de pesquisas amostrais têm sido cada vez mais utilizados também para
propósitos analíticos. \texttt{Inferências\ analíticas} baseadas numa
pesquisa amostral são aquelas que envolvem a estimação de parâmetros num
modelo (de superpopulação) \citep{kalton83b}; \citep{binder87}.

Quando os valores \texttt{amostrais} das variáveis da pesquisa podem ser
considerados como realizações de vetores aleatórios independentes e
identicamente distribuídos (IID), modelos podem ser especificados,
ajustados, testados e reformulados usando procedimentos estatísticos
padrões como os apresentados, por exemplo, em \citep{bickel} e
\citep{garthwaite}. Neste caso, métodos e pacotes estatísticos padrões
podem ser usados para executar os cálculos de estimativas de parâmetros
e medidas de precisão correspondentes, bem como diagnóstico e
verificação da adequação das hipóteses dos modelos.

Na prática das pesquisas amostrais, contudo, as hipóteses de modelo IID
para as observações amostrais são raramente adequadas. Com maior
frequência, modelos alternativos com hipóteses mais complexas e/ou
estimadores especiais devem ser considerados a fim de acomodar aspectos
da estrutura da população e/ou do plano amostral. Além disso, usualmente
estão disponíveis informações sobre variáveis auxiliares, utilizadas ou
não na especificação do plano amostral, que podem ser incorporadas com
proveito na estima ção dos parâmetros ou na própria formulação do
modelo.

Os exemplos apresentados no Capítulo \ref{efeitos-do-plano-amostral}
demonstram claramente a inadequa ção de ignorar o plano amostral ao
efetuar análises de dados de pesquisas amostrais. Os valores dos EPAs
calculados, tanto para estimadores de medidas descritivas tais como
médias e totais, como para estatísticas analíticas usadas em testes de
hipóteses e os correspondentes efeitos nos níveis de significância
reais, revelam que ignorar o plano amostral pode levar a decisões
erradas e a avaliações inadequadas da precisão das estimativas
amostrais.

Embora as medidas propostas no Capítulo \ref{epa} para os efeitos de
plano amostral sirvam para avaliar o impacto de ignorar o plano amostral
nas inferências descritivas ou mesmo analíticas baseadas em dados
amostrais, elas não resolvem o problema de como incorporar o plano
amostral nessas análises. No caso das inferências descritivas usuais
para médias, totais e proporções, o assunto é amplamente tratado na
literatura de amostragem e o interessado em maiores detalhes pode
consultar livros clássicos como \citep{cochran}, ou mais recentes como
\citep{SSW92}. Já os métodos requeridos para inferências analíticas só
recentemente foram consolidados em livro (\citep{SHS89}). Este capítulo
apresenta um dos métodos centrais disponíveis para ajuste de modelos
paramétricos regulares considerando dados amostrais complexos, baseado
no trabalho de \citep{binder87}. Antes de descrever esse método,
entretanto, fazemos breve discussão sobre o papel dos pesos na análise
de dados amostrais, considerando o trabalho de \citep{Pfeff}.

Primeiramente, porém, fazemos uma revisão sucinta do método de Máxima
Verossimilhança (MV) para ajustar modelos dentro da abordagem de
modelagem clássica, necessária para compreensão adequada do material
subseqüente. Essa revisão não pretende ser exaustiva ou detalhada, mas
tão somente recordar os principais resultados aqui requeridos. Para uma
discussão mais detalhada do método de Máxima Verossimilhança para
estimação em modelos paramétricos regulares veja, por exemplo,
\citep{garthwaite}.

\section{Método de Máxima Verossimilhança
(MV)}\label{metodo-de-maxima-verossimilhanca-mv}

Seja \(\mathbf{y}_{i}=\left(y_{i1},\ldots,y_{iR}\right)'\) um vetor
\(R\times 1\) dos valores observados das variáveis de interesse
observadas para a unidade \(i\) da amostra, gerado por um vetor
aleatório \(\mathbf{Y}_{i}\), para \(i=1,\ldots ,n\), onde \(n\) é o
tamanho da amostra. Suponha que os vetores aleatórios
\(\mathbf{Y}_{i}\), para \(i=1,\ldots ,n\) , são independentes e
identicamente distribuídos (IID) com distribuição comum
\(f(\mathbf{y};\mathbf{\theta })\), onde
\(\mathbf{\theta}=\left( \theta _{1},\ldots ,\theta _{K}\right) ^{^{\prime }}\)
é um vetor \(K\times 1\) de parâmetros desconhecidos de interesse. Sob
essas hipóteses, a verossimilhança amostral é dada por \[
l\left( \mathbf{\theta }\right) =\prod\limits_{i=1}^{n}f\left( \mathbf{y}
_{i};\mathbf{\theta }\right) 
\] e a correspondente log-verossimilhança por \[
L\left( \mathbf{\theta }\right) =\sum_{i=1}^n\log \left[
f\left( \mathbf{y}_{i};\mathbf{\theta }\right) \right] \;. 
\]

Calculando as derivadas parciais de \(L\left(\mathbf{\theta}\right)\)
com relação a cada componente de \(\mathbf{\theta }\) e igualando a
\(0\), obtemos um sistema de equações \[
\partial L\left( \mathbf{\theta }\right) /\partial \mathbf{\theta }=
\sum_{i=1}^n\mathbf{u}_{i}\left( \mathbf{\theta }\right) =\mathbf{0}, 
\]

onde,
\(\mathbf{u}_{i}\left(\mathbf{\theta }\right) =\partial\log\left[f\left(\mathbf{y}_{i};\mathbf{\theta}\right) \right] /\partial \mathbf{\theta }\)
é o vetor dos escores da unidade \(i\), de dimensão \(K\times1\).

Sob condições de regularidade p.~281 \citep{cox}, a solução
\(\mathbf{\hat{\theta}}\) deste sistema de equações é o
\textbf{Estimador de Máxima Verossimilhança (EMV)} de
\(\mathbf{\theta}\). A variância assintótica do estimador
\(\mathbf{\hat{\theta}}\) sob o modelo adotado, denominado aqui
abreviadamente modelo \$M \$, é dada por \[
V_{M}\left( \mathbf{\hat{\theta}}\right) \simeq \left[ J\left( \mathbf{
\theta }\right) \right] ^{-1} 
\] e um estimador consistente dessa variância é dado por \[
\hat{V}_{M}\left( \mathbf{\hat{\theta}}\right) =\left[ J\left( \mathbf{\hat{
\theta}}\right) \right] ^{-1}\;, 
\] onde \[
J\left( \mathbf{\theta }\right) =\sum\limits_{i=1}^{n}\partial \mathbf{u}
_{i}\left( \mathbf{\theta }\right) /\partial \mathbf{\theta } 
\] e \[
J\left( \mathbf{\hat{\theta}}\right) =\left. J\left( \mathbf{\theta }\right)
\right| _{\mathbf{\theta =\hat{\theta}}}\;. 
\]

\section{Ponderação de Dados
Amostrais}\label{ponderacao-de-dados-amostrais}

O papel da ponderação \texttt{na\ análise\ de\ dados\ amostrais} é alvo
de controvérsia entre os estatísticos. Apesar de incorporada comumente
na inferência descritiva, não há concordância com respeito a seu uso na
inferência analítica, havendo um espectro de opiniões entre dois
extremos. Num extremo estão os \texttt{modelistas}, que consideram o uso
de pesos irrelevante, e no outro os \texttt{amostristas}, que incorporam
pesos em qualquer análise.

\BeginKnitrBlock{example}
\protect\hypertarget{ex:unnamed-chunk-1}{}{\label{ex:unnamed-chunk-1}}Uso
analítico dos dados da Pesquisa Nacional por Amostra de Domicílios
(PNAD)
\EndKnitrBlock{example}

A título de ilustração, consideremos uma pesquisa com uma amostra
complexa como a da PNAD do IBGE, que emprega uma amostra estratificada
de domicílios em três estágios, tendo como unidades primárias de
amostragem (UPAs) os municípios, que são estratificados segundo as
unidades da federação (UFs), e regiões menores dentro das UFs (veja
IBGE, 1981, p.~67).

A seleção de municípios dentro de cada estrato é feita com
probabilidades desiguais, proporcionais ao tamanho, havendo inclusive
municípios incluídos na amostra com certeza (chamados de municípios
auto-representativos). Da mesma forma, a seleção de setores (unidades
secundárias de amostragem ou USAs) dentro de cada município é feita com
probabilidades proporcionais ao número de domicílios em cada setor
segundo o último censo disponível. Dentro de cada setor, a seleção de
domicílios é feita por amostragem sistemática simples (portanto, com
eqüiprobabilidade). Todas as pessoas moradoras em cada domicílio da
amostra são pesquisadas.

A amostra de domicílios e de pessoas dentro de cada estrato é
\texttt{autoponderada}, isto é, tal que todos os domicílios e pessoas
dentro de um mesmo estrato têm igual probabilidade de seleção.
Entretanto, as probabilidades de inclusão (e conseqüentemente os pesos)
variam bastante entre as várias regiões de pesquisa. A Tabela
\ref{tab:proselpnad} revela como variam essas probabilidades de seleção
entre as regiões cobertas pela amostra da PNAD de 93. Como se pode
observar, tais probabilidades de inclusão chegam a ser \(5\) vezes
maiores em Belém do que em São Paulo, e portanto variação semelhante
será observada nos pesos.

\begin{longtable}[]{@{}lc@{}}
\caption{\label{tab:proselpnad} Probabilidades de seleção da amostra da PNAD
de 1993 segundo regiões}\tabularnewline
\toprule
\begin{minipage}[b]{0.75\columnwidth}\raggedright\strut
Região da pesquisa\strut
\end{minipage} & \begin{minipage}[b]{0.19\columnwidth}\centering\strut
Probabilidade de seleção\strut
\end{minipage}\tabularnewline
\midrule
\endfirsthead
\toprule
\begin{minipage}[b]{0.75\columnwidth}\raggedright\strut
Região da pesquisa\strut
\end{minipage} & \begin{minipage}[b]{0.19\columnwidth}\centering\strut
Probabilidade de seleção\strut
\end{minipage}\tabularnewline
\midrule
\endhead
\begin{minipage}[t]{0.75\columnwidth}\raggedright\strut
RM de Belém\strut
\end{minipage} & \begin{minipage}[t]{0.19\columnwidth}\centering\strut
1/150\strut
\end{minipage}\tabularnewline
\begin{minipage}[t]{0.75\columnwidth}\raggedright\strut
RMs de Fortaleza, Recife, Salvador e Porto Alegre Distrito Federal\strut
\end{minipage} & \begin{minipage}[t]{0.19\columnwidth}\centering\strut
1/200\strut
\end{minipage}\tabularnewline
\begin{minipage}[t]{0.75\columnwidth}\raggedright\strut
RMs de Belo Horizonte e Curitiba\strut
\end{minipage} & \begin{minipage}[t]{0.19\columnwidth}\centering\strut
1/250\strut
\end{minipage}\tabularnewline
\begin{minipage}[t]{0.75\columnwidth}\raggedright\strut
Rondô\}nia, Acre, Amazonas, Roraima, Amapá, Tocantins, Sergipe, Mato
Grosso do Sul, Mato Grosso e Goiás\strut
\end{minipage} & \begin{minipage}[t]{0.19\columnwidth}\centering\strut
1/300\strut
\end{minipage}\tabularnewline
\begin{minipage}[t]{0.75\columnwidth}\raggedright\strut
Pará\strut
\end{minipage} & \begin{minipage}[t]{0.19\columnwidth}\centering\strut
1/350\strut
\end{minipage}\tabularnewline
\begin{minipage}[t]{0.75\columnwidth}\raggedright\strut
RM do Rio de Janeiro, Piauí, Ceará, Rio Grande do Norte, Paraíba,
Pernambuco, Alagoas, Bahia, Minas Gerais, Espírito Santo e Rio de
Janeiro\strut
\end{minipage} & \begin{minipage}[t]{0.19\columnwidth}\centering\strut
1/500\strut
\end{minipage}\tabularnewline
\begin{minipage}[t]{0.75\columnwidth}\raggedright\strut
Paraná, Santa Catarina, Rio Grande do Sul\strut
\end{minipage} & \begin{minipage}[t]{0.19\columnwidth}\centering\strut
1/550\strut
\end{minipage}\tabularnewline
\begin{minipage}[t]{0.75\columnwidth}\raggedright\strut
RM de São Paulo, Maranhão, São Paulo\strut
\end{minipage} & \begin{minipage}[t]{0.19\columnwidth}\centering\strut
1/750\strut
\end{minipage}\tabularnewline
\bottomrule
\end{longtable}

Se \({\ \pi }_{i}\)representa a probabilidade de inclusão na amostra do
\(i\)-ésimo domicílio da população, \(i=1,...,N\), então \[
\pi _{i}=\pi _{munic\acute{\imath}pio\left| estrato\right. }\times \pi
_{setor\left| munic\acute{\imath}pio\right. }\times \pi _{domic\acute{\imath}
lio\left| setor\right. } 
\] isto é, a probabilidade global de inclusão de um domicílio (e
conseqüentemente de todas as pessoas nele moradoras) é dada pelo produto
das probabilidades condicionais de inclusão nos vários estágios de
amostragem.

A estimação do total populacional \(Y\) de uma variável de pesquisa
\(y\) num dado estrato usando os dados da PNAD é feita rotineiramente
com estimadores ponderados de tipo razão
\(\widehat{Y}_{R}=\widehat{Y}_{\pi }\,X\,/\,\widehat{X}_{\pi }=\sum_{i\in s}w_{i}^{R}y_{i}\)
(tal como definidos por \eqref{eq:estpa15}, com pesos dados por
\(w_{i}^{R}=\pi_{i}^{-1}X\,/\,\widehat{X}_{\pi }\) (veja
\eqref{eq:estpa17}, onde \(X\) é o total da população no estrato obtido
por métodos demográficos de projeção, utilizado como variável auxiliar,
e \(\widehat{X}_{\pi}\) e \(\widehat{Y}_{\pi}\) são os estimadores
\(\pi\)-ponderados de \(X\) e \(Y\) respectivamente. Para estimar para
conjuntos de estratos basta somar as estimativas para cada estrato
incluído no conjunto. Para estimar médias e proporções, os pesos são
também incorporados da forma apropriada. No caso, a estimação de médias
é feita usando estimadores ponderados da forma \[
\overline{y}^{R}=\frac{\sum_{i\in s}w_{i}^{R}y_{i}}{\sum_{i\in s}w_{i}^{R}} 
\] e a estimação de proporções é caso particular da estimação de médias
quando a variável de pesquisa \(y\) é do tipo indicador (isto é, só toma
valores \(0\) e \(1\)).

Estimadores ponderados (como por exemplo os usados na PNAD) são
preferidos pelos praticantes de amostragem por sua simplicidade e por
serem não viciados (ao menos aproximadamente) com respeito à
distribuição de aleatorização induzida pela seleção da amostra,
independentemente dos valores assumidos pelas variáveis de pesquisa na
população. Já para a modelagem de relações entre variáveis de pesquisa,
o uso dos pesos induzidos pelo planejamento amostral ainda não é
freqüente ou aceito sem controvérsia.

Um exemplo de modelagem desse tipo com dados da PNAD em que os pesos e o
desenho amostral não foram considerados na análise é encontrado em
\citep{Leote}. Essa autora empregou modelos de regressão logística para
traçar um perfil sócio-econômico da mão-de-obra empregada no mercado
informal de trabalho urbano no Rio de Janeiro, usando dados do
suplemento sobre trabalho da PNAD-90. Todos os ajustes efetuados
ignoraram os pesos e o plano amostral da pesquisa. O problema foi
revisitado por \citep{Pessoa}, quando então esses aspectos foram
devidamente incorporados na análise. Um resumo desse trabalho é
discutido no Capítulo \ref{modreg}.

Vamos supor que haja interesse em regredir uma determinada variável de
pesquisa \(y\) contra algumas outras variáveis de pesquisa num vetor de
regressores \(\mathbf{z}\). Seria natural indagar se, como no caso do
total e da média, os pesos amostrais poderiam desempenhar algum papel na
estimação dos parâmetros do modelo (linear) de regressão. Uma
possibilidade de incluir os pesos seria estimar os coeficientes da
regressão por:

\begin{equation}
\widehat{\beta }_{w}=\left(\sum_{i\in s}w_{i}\mathbf{z}_{i}^{\prime }
\mathbf{z}_{i}\right) ^{-1}\sum_{i\in s}w_{i}\mathbf{z}_{i}^{\prime}y_{i}=
\left(\mathbf{Z}_{s}^{\prime }\mathbf{W}_{s}\mathbf{Z}_{s}\right)^{-1}\mathbf{Z}_{s}^{\prime }\mathbf{W}_{s}\mathbf{Y}_{s} 
\label{eq:modpar1}
\end{equation}

em lugar do estimador de mínimos quadrados ordinários (MQO) dado por

\begin{equation}
\widehat{\beta}=\left( \sum_{i\in s}\mathbf{z}_{i}^{\prime }\mathbf{z}_{i}\right)^{-1}\sum_{i\in s}\mathbf{z}_{i}^{\prime }y_{i}=\left(\mathbf{Z}_{s}^{\prime }\mathbf{Z}_{s}\right)^{-1}\mathbf{Z}_{s}^{\prime }\mathbf{Y}_{s}  
\label{eq:modpar2}
\end{equation}

onde \(w_{i}=\pi _{i}^{-1}\), \(y_{i}\) é o valor da variável resposta e
\(\mathbf{z}_{i}\) é o vetor de regressores para a observação \(i\),
\(\mathbf{Z}_{s}\) e \(\mathbf{Y}_{s}\) são respectivamente a matriz e
vetor com os valores amostrais dos \(\mathbf{z}_{i}\) e \(y_{i}\), e
\(\mathbf{W}_{s}=diag\left\{ w_{i};i\in s\right\}\) é a matriz diagonal
com os pesos amostrais.

Não é possível justificar o estimador \(\widehat{\beta }_{w}\) em
\eqref{eq:modpar1} com base em critério de otimalidade, tal como ocorre
com os estimadores usuais de Máxima Verossimilhança ou de Mínimos
Quadrados Ordinários (MQO), se uma modelagem clássica IID fosse adotada
para a amostra.

De um ponto de vista formal (matemático), o estimador
\(\widehat{\beta }_{w}\) em \eqref{eq:modpar1} é equivalente ao estimador
de Mínimos Quadrados Ponderados (MQP) com pesos \(w_{i}\). Entretanto,
esses estimadores diferem de maneira acentuada. Os estimadores de MQP
são usualmente considerados quando o modelo de regressão é
heteroscedástico, isto é, quando os resíduos têm variâncias desiguais.
Nes-te caso, os pesos adequados seriam dados pelos inversos das
variâncias dos resíduos correspondentes a cada uma das observações, e
portanto em geral diferentes dos pesos iguais aos inversos das
correspondentes probabilidades de seleção. Além desta diferença de
interpretação do papel dos pesos no estimador, outro aspecto em que os
dois estimadores diferem de forma acentuada é na estimação da precisão,
com o estimador MQP acoplado a um estimador de variância baseado no
modelo e o estimador \(\widehat{\beta }_{w}\) acoplado a estimadores de
variância que incorporam o planejamento amostral e os pesos, tal como se
verá mais adiante.

O estimador \(\widehat{\beta }_{w}\) foi proposto formalmente por
\citep{fuller75}, que o concebeu como uma função de estimadores de
totais populacionais. A mesma idéia subsidiou vários outros autores que
estudaram a estimação de coeficientes de regressão partindo de dados
amostrais complexos, tais como \citep{NH80}, \citep{PfefeNat}. Uma
revisão abrangente da literatura existente sobre estimação de parâmetros
em modelos de regressão linear com dados amostrais complexos pode ser
encontrada em cap. 6, \citep{Silva}.

Apesar dessas dificuldades, será que é possível justificar o uso de
pesos na inferência baseada em modelos? Se for o caso, sob que
condições? Seria possível desenvolver diretrizes para o uso de pesos em
inferência analítica partindo de dados amostrais complexos? A resposta
para essas perguntas é afirmativa, ao menos quando a questão da robustez
da inferência é relevante. Em inferências analíticas partindo de dados
amostrais complexos, os pesos podem ser usados para proteger:

\begin{enumerate}
\def\labelenumi{\arabic{enumi}.}
\item
  contra \texttt{planos\ amostrais\ não-ignoráveis}, que poderiam
  introduzir ou causar \texttt{vícios};
\item
  contra a \texttt{má\ especificação\ do\ modelo}.
\end{enumerate}

A \texttt{robustez\ dos\ procedimentos} que incorporam pesos é obtida
pela mudança de foco da inferência para
\texttt{quantidades\ da\ população\ finita}, que definem parâmetros-alvo
alternativos aos parâmetros do modelo de superpopulação, conforme já
discutido na Seção \ref{modelsuperpop}.

A questão da construção dos pesos não será tratada neste texto,
usando-se sempre como peso o inverso da probabilidade de inclusão na
amostra. é possível utilizar pesos de outro tipo como, por exemplo,
aqueles de razão empregados na estimação da PNAD, ou mesmo pesos de
regressão. Para esses casos, há que fazer alguns ajustes da teoria aqui
exposta (veja \citep{Silva}, cap. 6).

Há várias formas alternativas de incorporar os pesos amostrais no
processo de inferência. A principal que será adotada ao longo deste
texto será o método de \texttt{Máxima\ Pseudo-Verossimilhança}, que
descrevemos na próxima seção.

\section{Método de Máxima Pseudo-Verossimilhança label}\label{modpar3}

Suponha que os vetores observados \(\mathbf{y}_{i}\) das variáveis de
pesquisa do elemento\(\ i\) são gerados por vetores aleatórios
\(\mathbf{Y}_{i}\) , para \(i\in U\). Suponha também que
\(\mathbf{Y}_{1},\ldots ,\mathbf{Y}_{N}\) são IID com densidade
\(f\left( \mathbf{y},\mathbf{\theta }\right)\). Se todos os elementos da
população finita \(U\) fossem conhecidos, as funções de verossimilhança
e de log-verossimilhança \texttt{populacionais} seriam dadas
respectivamente por

\begin{equation}
l_{U}\left( \mathbf{\theta }\right) =\prod\limits_{i\in U}f\left( \mathbf{y}
_{i};\mathbf{\theta }\right)  
\label{eq:modpar3}
\end{equation}

e

\begin{equation}
L_{U}\left( \mathbf{\theta }\right) =\sum_{i\in U}\log \left[ f\left( 
\mathbf{y}_{i};\mathbf{\theta }\right) \right] \;\;.  
\label{eq:modpar4}
\end{equation}

As equações de verossimilhança \texttt{populacionais} correspondentes
são dadas por

\begin{equation}
\sum_{i\in U}\mathbf{u}_{i}\left( \mathbf{\theta }\right) =\mathbf{0}
\label{eq:modpar5}
\end{equation}

onde

\begin{equation}
\mathbf{u}_{i}\left( \mathbf{\theta }\right) =\partial \log \left[ f\left( 
\mathbf{y}_{i};\mathbf{\theta }\right) \right] /\partial \mathbf{\theta }
\label{eq:modpar6}
\end{equation}

é o vetor \(K\times 1\) dos escores do elemento \(i,i\in U\).

Sob condições de regularidade \citep{cox}, p.~281, a solução
\(\mathbf{\theta }_{U}\) deste sistema é o
\texttt{Estimador\ de\ Máxima\ Verossimilhança} de \(\mathbf{\theta }\)
no caso de um \texttt{censo}. Podemos considerar
\(\mathbf{\theta }_{U}\) como uma
\texttt{Quantidade\ Descritiva\ Populacional\ Correspondente\ (QDPC)} a
\(\mathbf{\theta }\), no sentido definido por \citep{Pfeff}, sobre a
qual se deseja fazer inferências com base em informações da amostra.
Essa definição da QDPC \(\mathbf{\theta }_{U}\) pode ser generalizada
para contemplar outras abordagens de inferência além da abordagem
clássica baseada em maximização da verossimilhança. Basta para isso
especificar outra regra ou critério a otimizar e então definir a QDPC
como a solução ótima segundo essa nova regra. Tal generalização,
discutida em \citep{Pfeff}, não será aqui considerada para manter a
simplicidade.

A QDPC~\(\mathbf{\theta }_{U}\) definida com base em \eqref{eq:modpar5}
não é calculável a menos que um censo seja realizado. Entretanto,
desempenha papel fundamental nessa abordagem inferencial, por
constituir-se num \texttt{pseudo-parâmetro}, eleito como alvo da
inferência num esquema que incorpora o planejamento amostral. Isto se
justifica porque, sob certas condições de regularidade,
\(\mathbf{\theta }_{U}\mathbf{ -\theta }=o_{p}\left( 1\right)\). Como em
pesquisas por amostragem o tamanho da população é geralmente grande, um
estimador adequado para \(\mathbf{\theta}_{U}\) será geralmente adequado
também para \(\mathbf{\theta }\).

Seja
\(\mathbf{T}=\sum_{i\in U}\mathbf{u}_{i}\left( \mathbf{\theta }\right)\)
a soma dos vetores de escores na população, o qual é um vetor de totais
populacionais. Para estimar este vetor de totais, podemos então usar um
estimador linear ponderado da forma
\(\mathbf{\hat{T}}=\sum_{i\in s}w_{i}\mathbf{u}_{i}\left( \mathbf{\theta }\right)\)
(veja Capítulo \ref{estimacao-baseada-no-plano-amostral}) onde \(w_{i}\)
são pesos propriamente definidos. Com essa notação, podemos agora obter
um estimador para \(\mathbf{\theta }_{U}\) resolvendo o sistema de
equações obtido igualando o estimador \(\mathbf{\hat{T}}\) do total
\(\mathbf{T}\) a zero.

\BeginKnitrBlock{definition}
\protect\hypertarget{def:unnamed-chunk-2}{}{\label{def:unnamed-chunk-2}}O
estimador de Máxima Pseudo-Verossimilhança (MPV)
\(\mathbf{\hat{\theta}}_{MPV}\) de \(\mathbf{\theta }_{U}\) (e
consequentemente de \(\mathbf{\theta}\)) será a solução das equações
dePseudo-Verossimilhança dadas por

\begin{equation}
\mathbf{\hat{T}}=\sum_{i\in s}w_{i}\mathbf{u}_{i}\left( \mathbf{\theta }
\right) =\mathbf{0\;\;.}  
\label{eq:modpar7}
\end{equation}
\EndKnitrBlock{definition}

Através da linearização de Taylor (veja Seção \ref{taylor} e
considerando os resultados de \citep{binder83}, podemos obter a
variância de aleatorização assintótica do estimador
\(\mathbf{\hat{\theta}}_{MPV}\) e seu estimador correspondente, dados
respectivamente por:

\begin{equation}
V_{p}\left( \mathbf{\hat{\theta}}_{MPV}\right) \simeq \left[ J\left( \mathbf{
\theta }_{U}\right) \right] ^{-1}V_{p}\left[ \sum_{i\in s}w_{i}\mathbf{u}
_{i}\left( \mathbf{\theta }_{U}\right) \right] \left[ J\left( \mathbf{\theta 
}_{U}\right) \right] ^{-1}  
\label{eq:modpar8}
\end{equation}

e

\begin{equation}
\hat{V}_{p}\left( \mathbf{\hat{\theta}}_{MPV}\right) =\left[ \hat{J}\left( 
\mathbf{\hat{\theta}}_{MPV}\right) \right] ^{-1}\hat{V}_{p}\left[ \sum_{i\in
s}w_{i}\mathbf{u}_{i}\left( \mathbf{\hat{\theta}}_{MPV}\right) \right]
\left[ \hat{J}\left( \mathbf{\hat{\theta}}_{MPV}\right) \right] ^{-1}\;,
\label{eq:modpar9}
\end{equation}

onde

\begin{equation}
J\left( \mathbf{\theta }_{U}\right) =\left. \frac{\partial T\left( \mathbf{
\theta }\right) }{\partial \mathbf{\theta }}\right| _{\mathbf{\theta =\theta 
}_{U}}=\sum_{i\in U}\left. \frac{\partial \mathbf{u}_{i}\left( \mathbf{
\theta }\right) }{\partial \left( \mathbf{\theta }\right) }\right| _{\mathbf{
\theta =\theta }_{U}},  
\label{eq:modpar10}
\end{equation}

\begin{equation}
\hat{J}\left( \mathbf{\hat{\theta}}_{MPV}\right) =\left. \frac{\partial 
\widehat{T}\left( \mathbf{\theta }\right) }{\partial \mathbf{\theta }}
\right| _{\mathbf{\theta =\hat{\theta}}_{MPV}}=\sum_{i\in s}w_{i}\left. 
\frac{\partial \mathbf{u}_{i}\left( \mathbf{\theta }\right) }{\partial 
\mathbf{\theta }}\right| _{\mathbf{\theta =\hat{\theta}}_{MPV}},
\label{eq:modpar11}
\end{equation}

\(V_{p}\left[\sum_{i\in s}w_{i}\mathbf{u}_{i}\left( \mathbf{\theta}_{U}\right) \right]\)
é a matriz de variância (de aleatorização) do estimador do total
populacional dos escores e
\(\hat{V}_{p}\left[\sum_{i\in s}w_{i}\mathbf{u}_{i}\left(\mathbf{\hat{\theta}}_{MPV}\right)\right]\)
é um estimador consistente para esta variância. Binder(1983) mostrou
também que a distribuição assintótica de \(\mathbf{\hat{\theta}}_{MPV}\)
é Normal Multivariada, isto é, que

\begin{equation}
\left[ \hat{V}_{p}\left( \mathbf{\hat{\theta}}_{MPV}\right) \right]^{-1/2}\left(\mathbf{\hat{\theta}}_{MPV}-\mathbf{\theta }_{U}\right) \sim  \mathbf{NM}\left(\mathbf{0};\mathbf{I}\right),  
\label{eq:modpar12}
\end{equation}

o que fornece uma base para a inferência sobre \(\mathbf{\theta }_{U}\)
(ou \(\mathbf{\theta }\)) usando amostras grandes.

Muitos modelos paramétricos, com vários planos amostrais e estimadores
de totais diferentes, podem ser ajustados resolvendo-se as equa ções de
Pseudo-Verossimilhança \eqref{eq:modpar7}, satisfeitas algumas condições
de regularidade enunciadas em \citep{apêndice} e revistas em
\citep{Silva}, p.~126. Entretanto, os estimadores de MPV não serão
únicos, já que existem diversas maneiras de se definir os pesos
\(w_{i}\).

Os pesos \(w_{i}\) devem ser tais que os estimadores de total em
\eqref{eq:modpar7} sejam assintoticamente normais e não-viciados, e
possuam estimadores de variância consistentes, conforme requerido para a
obtenção da distribuição assintótica dos estimadores MPV. Os pesos mais
usados são os do estimador \(\pi\)-ponderado ou de Horvitz-Thompson para
totais, dados pelo inverso das probabilidades de inclusão dos
indivíduos, ou seja \(w_{i}=\pi _{i}^{-1}\). Tais pesos satisfazem essas
condições sempre que \(\pi _{i}>0\) e
\(\pi _{ij}>0\quad \forall i,j\in U\) e algumas condições adicionais de
regularidade são satisfeitas (veja, \citep{fuller84}).

Assim, um procedimento padrão para ajustar um modelo paramétrico regular
\(f\left( \mathbf{y};\mathbf{\theta }\right)\) pelo método da Máxima
Pseudo-Verossimilhança seria dado pelos passos indicados a seguir.

\begin{enumerate}
\def\labelenumi{\arabic{enumi}.}
\item
  Resolver
  \(\sum\limits_{i\in s}\pi _{i}^{-1}\mathbf{u}_{i}\left( \mathbf{\theta }\right) =\mathbf{0}\)
  e calcular o estimador pontual \(\mathbf{ \hat{\theta}}_{\pi }\) do
  parâmetro \(\mathbf{\theta }\)\textbf{\ }no modelo
  \(f\left( \mathbf{y;\theta }\right)\) (ou do pseudo-parâmetro
  \(\mathbf{\theta }_{U}\) correspondente).
\item
  Calcular a matriz de variância estimada

  \begin{equation}
  \hat{V}_{p}\left( \mathbf{\hat{\theta}}_{\pi }\right) =\left[ \hat{J}\left( 
  \mathbf{\hat{\theta}}_{\pi }\right) \right] ^{-1}\hat{V}_{p}\left[
  \sum\limits_{i\in s}\pi _{i}^{-1}\mathbf{u}_{i}\left( \mathbf{\hat{\theta}}
  _{\pi }\right) \right] \left[ \hat{J}\left( \mathbf{\hat{\theta}}_{\pi
  }\right) \right] ^{-1},  
  \label{eq:modpar13}
  \end{equation}

  onde
\end{enumerate}

\begin{equation}
\hat{V}_{p}\left[ \sum\limits_{i\in s}\pi _{i}^{-1}\mathbf{u}_{i}\left( 
\mathbf{\hat{\theta}}_{\pi }\right) \right] =\sum\limits_{i\in
s}\sum\limits_{j\in s}\frac{\pi _{ij}-\pi _{i}\pi _{j}}{\pi _{i}\pi _{j}}
\left[ \mathbf{u}_{i}\left( \mathbf{\hat{\theta}}_{\pi }\right) \right]
\left[ \mathbf{u}_{j}\left( \mathbf{\hat{\theta}}_{\pi }\right) \right]
^{^{\prime }}  
\label{eq:modpar14}
\end{equation}

e

\begin{equation}
\hat{J}\left( \mathbf{\hat{\theta}}_{\pi }\right) =\left. \frac{\partial 
\widehat{T}\left( \mathbf{\theta }\right) }{\partial \mathbf{\theta }}
\right| _{\mathbf{\theta }=\mathbf{\hat{\theta}}_{\pi }}=\sum\limits_{i\in
s}\pi _{i}^{-1}\left. \frac{\partial \mathbf{u}_{i}\left( \mathbf{\theta }
\right) }{\partial \mathbf{\theta }}\right| _{\mathbf{\theta }=\mathbf{\hat{
\theta}}_{\pi }}\;\;.  
\label{eq:modpar15}
\end{equation}

\begin{enumerate}
\def\labelenumi{\arabic{enumi}.}
\setcounter{enumi}{2}
\tightlist
\item
  Usar \(\mathbf{\hat{\theta}}_{\pi }\) e
  \(\hat{V}_{p}\left( \mathbf{\hat{\theta}}_{\pi }\right)\) para
  calcular regiões ou intervalos de confiança e/ou estatísticas de teste
  baseadas na distribuição normal e utilizá-las para fazer inferência
  sobre os componentes de \(\mathbf{\theta}\).
\end{enumerate}

\BeginKnitrBlock{remark}
\iffalse {Observação } \fi No Método de Máxima Pseudo-Verossimilhança,
os pesos amostrais são incorporados na análise através das equações de
estimação dos parâmetros \eqref{eq:modpar7} e através das equações de
estimação da matriz de covariância dos estimadores
\eqref{eq:modpar13}-\eqref{eq:modpar15}.
\EndKnitrBlock{remark}

\BeginKnitrBlock{remark}
\iffalse {Observação } \fi O plano amostral é também incorporado no
método de estimação MPV através da expressão para a variância do total
dos escores sob o plano amostral \eqref{eq:modpar14}, onde as propriedades
do plano amostral estão resumidas nas probabilidades de inclusão de
primeira e segunda ordem, isto é, os \(\pi _{i}\) e os \(\pi _{ij}\)
respectivamente.
\EndKnitrBlock{remark}

\BeginKnitrBlock{remark}
\iffalse {Observação } \fi Sob probabilidades de seleção iguais, os
pesos \(\pi _{i}^{-1}\) serão constantes e o estimador pontual
\(\hat{\theta}_{\pi }\) será idêntico ao estimador de Máxima
Verossimilhança (MV) ordinário para uma amostra de observações IID com
distribuição \(f\left(\mathbf{y;\theta }\right)\). Entretanto, o mesmo
não ocorre em se tratando da variância do estimador
\(\hat{\theta}_{\pi }\) , que difere da variância sob o modelo do
estimador usual de MV.\medskip
\EndKnitrBlock{remark}

\textbf{Vantagens do procedimento de MPV}

O procedimento MPV proporciona estimativas
\texttt{baseadas\ no\ plano\ amostral} para a variância assintótica dos
estimadores dos parâmetros, as quais são razoavelmente simples de
calcular e são consistentes sob \texttt{condições\ fracas} no plano
amostral e na especificação do modelo. Mesmo quando o estimador pontual
de MPV coincide com o estimador usual de Máxima Verossimilhança, a
estimativa da variância obtida pelo procedimento de MPV pode ser
preferível aos estimadores usuais da variância baseados no modelo, que
ignoram o plano amostral.

O procedimento MPV fornece estimativas \texttt{robustas}, no sentido de
que em muitos casos a quantidade \(\mathbf{\theta }_{U}\) da população
finita permanece um alvo válido para inferência, mesmo quando o modelo
especificado por \(f\left( \mathbf{y};\mathbf{\theta }\right)\) não
proporciona uma descrição adequada para a distribuição das variáveis de
pesquisa na população.

\textbf{Desvantagens do método de MPV}

Este procedimento requer conhecimento de informações detalhadas sobre os
elementos da amostra, tais como pertinência a estratos e conglomerados
ou unidades primárias de amostragem, e suas probabilidades de inclusão
ou pesos. Tais informações nem sempre estão disponíveis para usuários de
dados de pesquisas amostrais, seja por razões operacionais ou devido às
regras de proteção do sigilo de informações individuais.

As propriedades dos estimadores MPV não são conhecidas para pequenas
amostras. Este problema pode não ser importante em análises que usam os
dados de pesquisas feitas pelas agências oficiais de estatística, desde
que em tais análises seja utilizada a amostra inteira, ou no caso de
subdomínios estudados separadamente, que as amostras usadas sejam
suficientemente grandes nestes domínios.

Outra dificuldade é que métodos usuais de diagnóstico de ajuste de
modelos (tais como gráficos de resíduos) e outros procedimentos da
inferência clássica (tais como testes estatísticos de Razões de
Verossimilhança) não podem ser utilizados.

\section{Robustez do Procedimento
MPV}\label{robustez-do-procedimento-mpv}

Nesta seção vamos examinar a questão da robustez dos estimadores obtidos
pelo procedimento MPV. é essa robustez que justifica o emprego desses
estimadores frente aos estimadores usuais de MV, pois nas situações
práticas da análise de dados amostrais complexos as hipóteses usuais de
modelo IID para as observações amostrais raramente são verificadas.

Vamos agora analisar com mais detalhes a terceira abordagem para a
inferência analítica. Nela, postulamos um modelo como na primeira
abordagem e a inferência é direcionada aos parâmetros do modelo. Porém,
em vez de acharmos um estimador ótimo sob o modelo, achamos um estimador
na classe dos estimadores consistentes para a QDPC, onde a consistência
é referida à distribuição de aleatorização do estimador. Por que usar a
QDPC? A resposta é exatamente para obter maior robustez. Para entender
porque essa abordagem oferece maior robustez, vamos considerar dois
casos.

\begin{itemize}
\tightlist
\item
  Caso 1: o modelo para a população é adequado.
\end{itemize}

Então quando \(N\rightarrow \infty\) a QDPC \(\mathbf{\theta }_{U}\)
converge para o parâmetro \(\mathbf{\theta }\), isto é,
\(\mathbf{\theta }_{U}-\mathbf{\theta }\rightarrow \mathbf{0}\) em
probabilidade, segundo a distribuição de probabilidades do modelo \(M\).
Se \(\mathbf{\hat{\theta}}_{MPV}\) for consistente, então quando
\(n\rightarrow \infty\) temos que
\(\mathbf{\hat{\theta}}_{MPV}-\mathbf{\theta }_{U}\rightarrow\mathbf{0}\)
em probabilidade, segundo a distribuição de aleatorização \(p\).
Juntando essas condições obtemos que

\[
\mathbf{\hat{\theta}}_{MPV}\stackrel{P}{\rightarrow }\mathbf{\theta } 
\] em probabilidade segundo a mistura \(Mp\). Esse resultado segue
porque

\begin{eqnarray*}
\mathbf{\hat{\theta}}_{MPV}-\mathbf{\theta } &=&(\mathbf{\hat{\theta}}_{MPV}-
\mathbf{\theta }_{U})+\left( \mathbf{\theta }_{U}-\mathbf{\theta }\right) \\
&=&O_{p}(n^{-1/2})+O_{p}(N^{-1/2})=O_{p}(n^{-1/2})\;.
\end{eqnarray*}

\begin{itemize}
\tightlist
\item
  Caso 2: o modelo para a população não é válido.
\end{itemize}

Nesse caso, o parâmetro \(\mathbf{\theta }\) do modelo não tem
interpretação substantiva significante, porém a QDPC
\(\mathbf{\theta }_{U}\) é uma entidade definida na população finita
(real) com interpretação clara, independente da validade do modelo. Como
\(\mathbf{\hat{\theta}}_{MPV}\) é consistente para a QDPC
\(\mathbf{\theta}_{U}\), a inferência baseada no procedimento MPV segue
válida para este pseudo-parâmetro, independente da inadequação do modelo
para a população. \citep{Sk89b}, p.~81, discute essa situação, mostrando
que \(\mathbf{\theta }_{U}\) pode ainda ser um alvo válido para
inferência mesmo quando o modelo
\(f\left( \mathbf{y};\mathbf{\theta }\right)\) especificado para a
população é inadequado, ao menos no sentido de que
\(f\left( \mathbf{y};\mathbf{\theta}_{U}\right)\) forneceria a
\texttt{melhor\ aproximação\ possível} (em certo sentido) para o
verdadeiro modelo que gera as observações populacionais
(\(f^{*}\left( \mathbf{y};\mathbf{\eta }\right)\), digamos).
Skinner(1989b) reconhece que a \texttt{melhor\ aproximação\ possível}
entre um conjunto de aproximações ruins ainda seria uma aproximação
ruim, e portanto que a escolha do elenco de modelos especificados pela
distribuição \(f\left( \mathbf{y};\mathbf{\theta }\right)\) deve seguir
os cuidados necessários para garantir que esta escolha forneça uma
aproximação razoável da realidade.

\BeginKnitrBlock{remark}
\iffalse {Observação } \fi Consistência referente à distribuição de
aleatorização.
\EndKnitrBlock{remark}

Consistência na teoria clássica tem a ver com comportamento limite de um
estimador quando o tamanho da amostra cresce, isto é, quando
\(n\rightarrow \infty\). No caso de populações finitas, temos que
considerar o que ocorre quando crescem o tamanho da amostra e também o
tamanho da população, isto é, quando \(n\rightarrow \infty\) e
\(N\rightarrow \infty\). Neste caso, é preciso definir a maneira pela
qual \(N\uparrow\) e \(n\uparrow\) preservando a estrutura do plano
amostral. Para evitar um desvio indesejado que a discussão deste
problema traria, vamos supor que \(N\uparrow\) e \(n\uparrow\) de uma
forma bem definida. Os leitores interessados poderão consultar:
\citep{SSW92}, p.~166, \citep{brewer}, \citep{Isaki}, \citep{Robin},
\citep{hajek} e \citep{SHS89}, p.~18-19.

\section{Desvantagens da Inferência de
Aleatorização}\label{desvantagens-da-inferencia-de-aleatorizacao}

Se o modelo postulado para os dados amostrais for correto, o uso de
estimadores ponderados pode resultar em perda substancial de eficiência
comparado com o estimador ótimo, sob o modelo. Em geral, a perda de
eficiência aumenta quando diminui o tamanho da amostra e aumenta a varia
ção dos pesos. Há casos onde a ponderação é a única alternativa. Por
exemplo, se os dados disponíveis já estão na forma de estimativas
amostrais ponderadas, então o uso de pesos é inevitável. Um exemplo
clássico é discutido a seguir.

\BeginKnitrBlock{example}
\protect\hypertarget{ex:Analisec}{}{\label{ex:Analisec}}Análise secundária
de tabelas de contingência.
\EndKnitrBlock{example}

A pesquisa \texttt{Canada\ Health\ Survey} usa um plano amostral
estratificado com vários estágios de seleção. Nessa pesquisa, a
estimativa de contagem na cela \(k\) de uma tabela de contingência
qualquer é dada por \[
\widehat{N}_{k}=\sum_{a}\left( N_{a}/\widehat{N}_{a}\right) \left[
\sum_{h}\sum_{i}\sum_{j}w_{hij}Y_{ka\left( hij\right) }\right]
=\sum_{a}\left( N_{a}/\widehat{N}_{a}\right) \widehat{N}_{ka} 
\] onde \(Y_{ka\left( hij\right)}=1\) se a \(j\)-ésima unidade da UPA
\(i\) do estrato \(h\) pertence à \(k\)-ésima cela e ao \(a\)-ésimo
grupo de idade-sexo, e \(0\) (zero) caso contrário;

\(N_{a}/\widehat{N}_{a}-\) são fatores de ajustamento de
pós-estratificação que usam contagens censitárias \(N_{a}\) de
idade-sexo para diminuir as variâncias dos estimadores.

Quando as contagens \texttt{expandidas} \(\widehat{N}_{k}\) são usadas,
os testes de homogeneidade e de qualidade de ajuste de modelos
loglineares baseados em amostragem Multinomial e Poisson independentes
não são mais válidos. A estatística clássica \(X^{2}\) não tem mais
distribuição \(\chi ^{2}\) e sim uma soma ponderada
\(\sum_{k}\delta _{k}X_{k}\) de variáveis \(X_{k}\) IID com distribuição
\(\chi ^{2}\left( 1\right)\). Esse exemplo será rediscutido com mais
detalhes na Seção \ref{raoscott}.

A importância desse exemplo é ilustrar que mesmo quando o usuário pensa
estar livre das complicações causadas pelo plano amostral e pesos, ele
precisa estar atento à forma como foram gerados os dados que pretende
modelar ou analisar, sob pena de realizar inferências incorretas. Este
exemplo tem também grande importância prática, pois um grande número de
pesquisas domiciliares por amostragem produz como principal resultado
conjunto de tabelas com contagens e proporções, as quais foram obtidas
mediante ponderação pelas agências produtoras. Este é o caso, por
exemplo, da PNAD, da amostra do Censo Demográfico e de inúmeras outras
pesquisas do IBGE e de agências estatísticas congêneres.

\section{Laboratório de R}\label{laboratorio-de-r}

Usar função svymle da library survey para incluir exemplo de estimador
MPV?

Possibilidade: explorar o exemplo 2.1?

\chapter{Modelos de Regressão}\label{modreg}

\chapter{Testes de Qualidade de Ajuste}\label{testqualajust}

\chapter{Testes em Tabelas de Duas Entradas}\label{testetab2}

\chapter{Estimação de densidades}\label{estimacao-de-densidades}

\chapter{Modelos Hierárquicos}\label{modelos-hierarquicos}

\chapter{Não-Resposta}\label{nao-resposta}

\chapter{Diagnóstico de ajuste de
modelo}\label{diagnostico-de-ajuste-de-modelo}

\chapter{Agregação vs.~Desagregação}\label{agregdesag}

\chapter{Pacotes para Analisar Dados Amostrais}\label{pacotes}

\chapter{Placeholder}\label{placeholder}

\bibliography{packages,book}


\end{document}
