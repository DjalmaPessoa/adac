\documentclass[]{book}
\usepackage{lmodern}
\usepackage{amssymb,amsmath}
\usepackage{ifxetex,ifluatex}
\usepackage{fixltx2e} % provides \textsubscript
\ifnum 0\ifxetex 1\fi\ifluatex 1\fi=0 % if pdftex
  \usepackage[T1]{fontenc}
  \usepackage[utf8]{inputenc}
\else % if luatex or xelatex
  \ifxetex
    \usepackage{mathspec}
  \else
    \usepackage{fontspec}
  \fi
  \defaultfontfeatures{Ligatures=TeX,Scale=MatchLowercase}
\fi
% use upquote if available, for straight quotes in verbatim environments
\IfFileExists{upquote.sty}{\usepackage{upquote}}{}
% use microtype if available
\IfFileExists{microtype.sty}{%
\usepackage{microtype}
\UseMicrotypeSet[protrusion]{basicmath} % disable protrusion for tt fonts
}{}
\usepackage[margin=1in]{geometry}
\usepackage{hyperref}
\hypersetup{unicode=true,
            pdftitle={Análise de Dados Amostrais Complexos},
            pdfauthor={Djalma Pessoa e Pedro Nascimento Silva},
            pdfborder={0 0 0},
            breaklinks=true}
\urlstyle{same}  % don't use monospace font for urls
\usepackage{natbib}
\bibliographystyle{apalike}
\usepackage{color}
\usepackage{fancyvrb}
\newcommand{\VerbBar}{|}
\newcommand{\VERB}{\Verb[commandchars=\\\{\}]}
\DefineVerbatimEnvironment{Highlighting}{Verbatim}{commandchars=\\\{\}}
% Add ',fontsize=\small' for more characters per line
\usepackage{framed}
\definecolor{shadecolor}{RGB}{248,248,248}
\newenvironment{Shaded}{\begin{snugshade}}{\end{snugshade}}
\newcommand{\KeywordTok}[1]{\textcolor[rgb]{0.13,0.29,0.53}{\textbf{{#1}}}}
\newcommand{\DataTypeTok}[1]{\textcolor[rgb]{0.13,0.29,0.53}{{#1}}}
\newcommand{\DecValTok}[1]{\textcolor[rgb]{0.00,0.00,0.81}{{#1}}}
\newcommand{\BaseNTok}[1]{\textcolor[rgb]{0.00,0.00,0.81}{{#1}}}
\newcommand{\FloatTok}[1]{\textcolor[rgb]{0.00,0.00,0.81}{{#1}}}
\newcommand{\ConstantTok}[1]{\textcolor[rgb]{0.00,0.00,0.00}{{#1}}}
\newcommand{\CharTok}[1]{\textcolor[rgb]{0.31,0.60,0.02}{{#1}}}
\newcommand{\SpecialCharTok}[1]{\textcolor[rgb]{0.00,0.00,0.00}{{#1}}}
\newcommand{\StringTok}[1]{\textcolor[rgb]{0.31,0.60,0.02}{{#1}}}
\newcommand{\VerbatimStringTok}[1]{\textcolor[rgb]{0.31,0.60,0.02}{{#1}}}
\newcommand{\SpecialStringTok}[1]{\textcolor[rgb]{0.31,0.60,0.02}{{#1}}}
\newcommand{\ImportTok}[1]{{#1}}
\newcommand{\CommentTok}[1]{\textcolor[rgb]{0.56,0.35,0.01}{\textit{{#1}}}}
\newcommand{\DocumentationTok}[1]{\textcolor[rgb]{0.56,0.35,0.01}{\textbf{\textit{{#1}}}}}
\newcommand{\AnnotationTok}[1]{\textcolor[rgb]{0.56,0.35,0.01}{\textbf{\textit{{#1}}}}}
\newcommand{\CommentVarTok}[1]{\textcolor[rgb]{0.56,0.35,0.01}{\textbf{\textit{{#1}}}}}
\newcommand{\OtherTok}[1]{\textcolor[rgb]{0.56,0.35,0.01}{{#1}}}
\newcommand{\FunctionTok}[1]{\textcolor[rgb]{0.00,0.00,0.00}{{#1}}}
\newcommand{\VariableTok}[1]{\textcolor[rgb]{0.00,0.00,0.00}{{#1}}}
\newcommand{\ControlFlowTok}[1]{\textcolor[rgb]{0.13,0.29,0.53}{\textbf{{#1}}}}
\newcommand{\OperatorTok}[1]{\textcolor[rgb]{0.81,0.36,0.00}{\textbf{{#1}}}}
\newcommand{\BuiltInTok}[1]{{#1}}
\newcommand{\ExtensionTok}[1]{{#1}}
\newcommand{\PreprocessorTok}[1]{\textcolor[rgb]{0.56,0.35,0.01}{\textit{{#1}}}}
\newcommand{\AttributeTok}[1]{\textcolor[rgb]{0.77,0.63,0.00}{{#1}}}
\newcommand{\RegionMarkerTok}[1]{{#1}}
\newcommand{\InformationTok}[1]{\textcolor[rgb]{0.56,0.35,0.01}{\textbf{\textit{{#1}}}}}
\newcommand{\WarningTok}[1]{\textcolor[rgb]{0.56,0.35,0.01}{\textbf{\textit{{#1}}}}}
\newcommand{\AlertTok}[1]{\textcolor[rgb]{0.94,0.16,0.16}{{#1}}}
\newcommand{\ErrorTok}[1]{\textcolor[rgb]{0.64,0.00,0.00}{\textbf{{#1}}}}
\newcommand{\NormalTok}[1]{{#1}}
\usepackage{longtable,booktabs}
\usepackage{graphicx,grffile}
\makeatletter
\def\maxwidth{\ifdim\Gin@nat@width>\linewidth\linewidth\else\Gin@nat@width\fi}
\def\maxheight{\ifdim\Gin@nat@height>\textheight\textheight\else\Gin@nat@height\fi}
\makeatother
% Scale images if necessary, so that they will not overflow the page
% margins by default, and it is still possible to overwrite the defaults
% using explicit options in \includegraphics[width, height, ...]{}
\setkeys{Gin}{width=\maxwidth,height=\maxheight,keepaspectratio}
\IfFileExists{parskip.sty}{%
\usepackage{parskip}
}{% else
\setlength{\parindent}{0pt}
\setlength{\parskip}{6pt plus 2pt minus 1pt}
}
\setlength{\emergencystretch}{3em}  % prevent overfull lines
\providecommand{\tightlist}{%
  \setlength{\itemsep}{0pt}\setlength{\parskip}{0pt}}
\setcounter{secnumdepth}{5}
% Redefines (sub)paragraphs to behave more like sections
\ifx\paragraph\undefined\else
\let\oldparagraph\paragraph
\renewcommand{\paragraph}[1]{\oldparagraph{#1}\mbox{}}
\fi
\ifx\subparagraph\undefined\else
\let\oldsubparagraph\subparagraph
\renewcommand{\subparagraph}[1]{\oldsubparagraph{#1}\mbox{}}
\fi

%%% Use protect on footnotes to avoid problems with footnotes in titles
\let\rmarkdownfootnote\footnote%
\def\footnote{\protect\rmarkdownfootnote}

%%% Change title format to be more compact
\usepackage{titling}

% Create subtitle command for use in maketitle
\newcommand{\subtitle}[1]{
  \posttitle{
    \begin{center}\large#1\end{center}
    }
}

\setlength{\droptitle}{-2em}
  \title{Análise de Dados Amostrais Complexos}
  \pretitle{\vspace{\droptitle}\centering\huge}
  \posttitle{\par}
  \author{Djalma Pessoa e Pedro Nascimento Silva}
  \preauthor{\centering\large\emph}
  \postauthor{\par}
  \predate{\centering\large\emph}
  \postdate{\par}
  \date{2016-12-27}

\usepackage{booktabs}

\begin{document}
\maketitle

{
\setcounter{tocdepth}{1}
\tableofcontents
}
\begin{Shaded}
\begin{Highlighting}[]
\KeywordTok{library}\NormalTok{(survey)}
\end{Highlighting}
\end{Shaded}

\begin{verbatim}
## Loading required package: grid
\end{verbatim}

\begin{verbatim}
## Loading required package: Matrix
\end{verbatim}

\begin{verbatim}
## Loading required package: survival
\end{verbatim}

\begin{verbatim}
## 
## Attaching package: 'survey'
\end{verbatim}

\begin{verbatim}
## The following object is masked from 'package:graphics':
## 
##     dotchart
\end{verbatim}

\begin{Shaded}
\begin{Highlighting}[]
\KeywordTok{source}\NormalTok{(}\StringTok{"~}\CharTok{\textbackslash{}\textbackslash{}}\StringTok{GitHub}\CharTok{\textbackslash{}\textbackslash{}}\StringTok{adac}\CharTok{\textbackslash{}\textbackslash{}}\StringTok{data}\CharTok{\textbackslash{}\textbackslash{}}\StringTok{ppv1.R"}\NormalTok{)}
\end{Highlighting}
\end{Shaded}

\chapter*{Prefácio}\label{prefacio}
\addcontentsline{toc}{chapter}{Prefácio}

Uma preocupação básica de toda instituição produtora de informações
estatísticas é com a utilização `'correta'' de seus dados. Isso pode ser
intrepretado de várias formas, algumas delas com reflexos até na
confiança do público e na própria sobrevivência do órgão. Do nosso ponto
de vista, como técnicos da área de metodologia do IBGE, enfatizamos um
aspecto técnico particular, mas nem por isso menos importante para os
usuários dos dados.

A revolução da informática com a resultante facilidade de acesso ao
computador, criou condições extremamente favoráveis à utilização de
dados estatísticos, produzidos por órgãos como o IBGE. Algumas vezes
esses dados são utilizados para fins puramente descritivos. Outras
vezes, porém, sua utilização é feita para fins analíticos, envolvendo a
construção de modelos, quando o objetivo é extrair conclusões aplicáveis
também a populações distintas daquela da qual se extraiu a amostra.
Neste caso, é comum empregar, sem grandes preocupações, pacotes
computacionais padrões disponíveis para a seleção e ajuste de modelos. é
neste ponto que entra a nossa preocupação com o uso adequado dos dados
produzidos pelo IBGE.

O que torna tais dados especiais para quem pretende usá-los para fins
analíticos? Esta é a questão básica que será amplamente discutida ao
longo deste texto. A mensagem principal que pretendemos transmitir é que
certos cuidados precisam ser tomados para utilização correta dos dados
de pesquisas amostrais como as que o IBGE realiza.

O que torna especiais dados como os produzidos pelo IBGE é que estes são
obtidos através de pesquisas amostrais complexas de populações finitas
que envolvem: \textbf{probabilidades distintas de seleção,
estratificação e conglomeração das unidades, ajustes para compensar
não-resposta e outros ajustes}. Os pacotes tradicionais de análise
ignoram estes aspectos, podendo produzir estimativas incorretas tanto
dos parâmetros como para as variâncias destas estimativas. Quando
utilizamos a amostra para estudos analíticos, as opções disponíveis nos
pacotes estatísticos usuais para levar em conta os pesos distintos das
observações são apropriadas somente para observações independentes e
identicamente distribuídas (IID). Além disso, a variabilidade dos pesos
produz impactos tanto na estimação pontual quanto na estimação das
variâncias dessas estimativas, que sofre ainda influência da
estratificação e conglomeração.

O objetivo deste livro é analisar o impacto das simplificações feitas ao
utilizar procedimentos e pacotes usuais de análise de dados, e
apresentar os ajustes necessários desses procedimentos de modo a
incorporar na análise, de forma apropriada, os aspectos aqui
ressaltados. Para isto serão apresentados exemplos de análises de dados
obtidos em pesquisas amostrais complexas, usando pacotes clássicos e
também pacotes estatísticos especializados. A comparação dos resultados
das análises feitas das duas formas permitirá avaliar o impacto de
ignorar o plano amostral na análise dos dados resultantes de pesquisas
amostrais complexas.

\section*{Agradecimentos}\label{agradecimentos}
\addcontentsline{toc}{section}{Agradecimentos}

A elaboração de um texto como esse não se faz sem a colaboração de
muitas pessoas. Em primeiro lugar, agradecemos à Comissão Organizadora
do SINAPE por ter propiciado a oportunidade ao selecionar nossa proposta
de minicurso. Agradecemos também ao IBGE por ter proporcionado as
condições e os meios usados para a produção da monografia, bem como o
acesso aos dados detalhados e identificados que utilizamos em vários
exemplos.

No plano pessoal, agradecemos a Zélia Bianchini pela revisão do
manuscrito e sugestões que o aprimoraram. Agradecemos a Marcos Paulo de
Freitas e Renata Duarte pela ajuda com a computação de vários exemplos.
Agradecemos a Waldecir Bianchini, Luiz Pessoa e Marinho Persiano pela
colaboração na utilização do processador de textos. Aos demais colegas
do Departamento de Metodologia do IBGE, agradecemos o companheirismo e
solidariedade nesses meses de trabalho na preparação do manuscrito.

Finalmente, agradecemos a nossas famílias pela aceitação resignada de
nossas ausências e pelo incentivo à conclusão da empreitada.

\chapter{Introdução}\label{introducao}

\chapter{Referencial para Inferência}\label{referencial-para-inferencia}

\chapter{Estimação Baseada no Plano Amostral}\label{planamo}

\chapter{Efeitos do Plano Amostral}\label{efeitos-do-plano-amostral}

\chapter{Ajuste de Modelos
Paramétricos}\label{ajuste-de-modelos-parametricos}

\chapter{Modelos de Regressão}\label{modelos-de-regressao}

\chapter{Testes de Qualidade de
Ajuste}\label{testes-de-qualidade-de-ajuste}

\chapter{Testes em Tabelas de Duas
Entradas}\label{testes-em-tabelas-de-duas-entradas}

\chapter{Agregação vs.~Desagregação}\label{agregacao-vs.desagregacao}

\chapter{Pacotes para Analisar Dados
Amostrais}\label{pacotes-para-analisar-dados-amostrais}

\chapter{Placeholder}\label{placeholder}

\bibliography{packages,book}


\end{document}
